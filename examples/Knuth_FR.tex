Plutôt qu'un épigraphe:

\begin{framed}
\begin{quotation}

\textbf{Seibel:} Comment vous attaquez-vous à la lecture de code source? Même lire
quelque chose dans un langage de programmation que vous connaissez déjà est un problème
délicat.

\textbf{Knuth:} Mais ça vaut vraiment la peine pour ce que ça construit dans votre
cerveau. Donc, comment est-ce que je fais? Il y avait une machine appelée le Bunker
Ramo 300 et quelqu'un m'avait dit que le compilateur ForTran pour cette machine était
incroyablement rapide, mais personne n'avait la moindre idée de pourquoi il fonctionnait.
Je me suis procuré une copie du listing de son code source. Je n'avais pas de manuel
pour la machine, donc je n'étais même pas sûr de ce qu'était son langage machine.

Mais j'ai pris ça comme un défi intéressant. J'ai pu découvrir BEGIN et j'ai alors
commencé à décoder. Les codes opération avaient des sortes de mnémoniques sur deux
lettres et donc j'ai pu commencer à comprendre que "Ceci était probablement une
instruction de chargement, ceci probablement un branchement". Et je savais qu'il
s'agissait d'un compilateur ForTran, donc à un moment donné j'ai regardé la colonne
sept d'une carte, et c'était où ça disait s'il s'agissait d'un commentaire
ou non.

Après trois heures, j'en avais découvert un peu à propos de cette machine. Alors,
j'ai trouvé cette grosse table de branchement. Donc, c'était un puzzle et j'ai continué
à faire des petits graphiques comme si je travaillais dans un un organisme de sécurité
essayant de décoder un code secret. Mais je savais que ça fonctionnait et je savais
que c'était un compilateur ForTran-ce n'était pas chiffré dans le sens où ça serait
volontairement opaque; c'était seulement du code car je n'avais pas reçu le manuel
de cette machine.

Enfin j'ai réussi à comprendre pourquoi ce compilateur était si rapide.
Malheureusement ce n'était pas parce que son algorithme était brillant; c'était
seulement parce qu'ils avaient utilisé une programmation non structurée et optimisé
le code manuellement. 

C'était simplement la façon de résoudre une énigme inconnue-faire des tableaux et
des graphiques et y obtenir un peu plus d'informations et faire une hypothèse. En
général lorsque je lis un papier technique, c'est le même défi. J'essaye de me mettre
dans l'esprit de l'auteur, pour essayer de comprendre ce qu'est le concept. Plus
vous apprenez à lire les trucs des autres, plus vous serez capable d'inventer les
votre dans le futur, il me semble.

\end{quotation}
\end{framed}

( Peter Seibel --- Coders at Work: Reflections on the Craft of Programming )
