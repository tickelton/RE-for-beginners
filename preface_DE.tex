\section*{Vorwort}

\subsection*{Warum zwei Titel?}
\label{TwoTitles}

Dieses Buch hieß von 2014-2018 ``Reverse Engineering for Beginners'', jedoch hatte ich immer die Befürchtung, dass es den Leserkreis zu sehr einengen würde.

Infosec Leute kennen sich mit ``Reverse Engineering'' aus, jedoch hörte ich selten das Wort ``Assembler'' von Ihnen.

Desweiteren ist der Begriff ``Reverse Engineering'' etwas zu kryptisch für den Großteil von Programmierern, diesen ist jedoch ``Assembler'' geläufig.

Im Juli 2018 änderte ich als Experiment den Titel zu ``Assembly Language for Beginners'' und veröffentlichte den Link auf der Hacker News-Website\footnote{\url{https://news.ycombinator.com/item?id=17549050}}. Das Buch kam allgemein gut an.

Aus diesem Grund hat das Buch nun zwei Titel.

Ich habe den zweiten Titel zu ``Understanding Assembly Language'' geändert, da es bereits eine Erscheinung mit dem Titel ``Assembly Language for Beginners'' gab. Einige Leute sind der Meinung, dass ``for Beginners'' etwas sarkastisch klingt, für ein Buch mit \textasciitilde{}1000 Seiten.

Die beiden Bücher unterscheiden sich lediglich im Titel, dem Dateinamen (UAL-XX.pdf beziehungsweise RE4B-XX.pdf), URL und ein paar der einleitenden Seiten.

\subsection*{Über Reverse Engineering}

Es gibt verschiedene verbreitete Interpretationen des Begriffs Reverse Engineering:\\
1) Reverse Engineering von Software: Rückgewinnung des Quellcodes bereits kompilierter Programme;\\
2) Das Erfassen von 3D Strukturen und die digitalen Manipulationen die zur Duplizierung notwendig sind;\\
3) Nachbilden von \ac{DBMS}-Strukturen.\\
Dieses Buch behandelt die erste Interpretation.

\subsection*{Voraussetzungen}

Grundlegende Kenntnisse der Programmiersprache C.
Empfohlene Literatur: \myref{CCppBooks}.

\subsection*{Übungen und Aufgaben}
\dots 
befinden sich nun alle auf der Website: \url{http://challenges.re}.

\subsection*{Über den Autor}
\begin{tabularx}{\textwidth}{ l X }

\raisebox{-\totalheight}{
\includegraphics[scale=0.60]{Dennis_Yurichev.jpg}
}

&
Dennis Yurichev ist ein erfahrener Reverse Engineer und Programmierer.
Er kann per E-Mail kontaktiert werden: \textbf{\EMAIL{}}.

% FIXME: no link. \tablefootnote doesn't work
\end{tabularx}

% subsections:
\subsection*{%
	\RU{Отзывы об этой книге}%
	\EN{Praise for this book}%
	\ES{Elogios para}% TBT
	\PTBRph{}%
	\DE{Lob für}% TBT
	\PLph{}%
	\IT{Elogi per questo libro}%
	\FR{Éloges de ce livre}%
	\JA{賛辞}% TBT
}

\begin{itemize}

\item \q{Now that Dennis Yurichev has made this book free (libre), it is a contribution to the world of free knowledge and free education.} Richard M. Stallman,
\EN{GNU founder, software freedom activist.}
\RU{Основатель GNU, активист в области свободного ПО.}
\FR{Fondateur de GNU, militant pour la liberté des logiciels}
\IT{Fondatore di GNU, attivista del software libero.}
\JA{GNU創設者、自由なソフトウェアの活動家}

% expanded URLs to make it more robust for printouts. In electronic editions people will click anyway, so tracking will keep working
\item \q{It's very well done .. and for free .. amazing.}\footnote{\href{http://go.yurichev.com/17095}{twitter.com/daniel\_bilar/status/436578617221742593}} Daniel Bilar, Siege Technologies, LLC.

\item \q{... excellent and free}\footnote{\href{http://go.yurichev.com/17096}{twitter.com/petefinnigan/status/400551705797869568}} Pete Finnigan,%
	\RU{гуру по безопасности}%
	\ES{gur\'u de seguridad en}%
	\PTBRph{}%
	\DE{Security-Guru}%
	\PLph{}%
	\IT{Guru della sicurezza}
	\FR{gourou de la sécurité}
    \JA{セキュリティグル}
\oracle
	\EN{security guru}.

\item \q{... [the] book is interesting, great job!} Michael Sikorski,
	\RU{автор книги}%
	\EN{author of}%
	\ES{autor de}%
	\PTBRph{}%
	\DE{Autor von}%
	\PLph{}%
	\IT{Autore di}
	\FR{auteur de}
	\JA{以下の著作の著者です}
\emph{Practical Malware Analysis: The Hands-On Guide to Dissecting Malicious Software}.

\item \q{... my compliments for the very nice tutorial!} Herbert Bos,
	\RU{профессор университета}%
	\EN{full professor at the}%
	\ES{catedr\'atico de tiempo completo en la}%
	\PTBRph{}%
	\DE{Professor an der}%
	\PLph{}%
	\IT{professore presso la}
	\FR{professeur à temps complet à}
	\JA{教授}
Vrije Universiteit Amsterdam,
	\RU{соавтор}%
	\EN{co-author of}%
	\ES{coautor de}%
	\PTBRph{}%
	\DE{Co-Autor von}%
	\PLph{}%
	\IT{coautore di}
	\FR{co-auteur de}
	\JA{共著者}
\emph{Modern Operating Systems (4th Edition)}.

\item \q{... It is amazing and unbelievable.} Luis Rocha, CISSP / ISSAP, Technical Manager, Network \& Information Security at Verizon Business.

\item \q{Thanks for the great work and your book.} Joris van de Vis,
	\RU{специалист по}%
	\ES{especialista en}%
	\PTBRph{}%
	\DE{Spezialist bei}%
	\PLph{}%
	\IT{specialista di}
	\FR{spécialiste}
SAP Netweaver \& Security
	\EN{specialist}.
	\JA{スペシャリスト}

\item \q{... [a] reasonable intro to some of the techniques.}\footnote{\href{http://go.yurichev.com/17099}{reddit}} Mike Stay,
	\RU{преподаватель в}%
	\EN{teacher at the}%
	\ES{profesor en el}%
	\PTBRph{}%
	\DE{Professor an der}%
	\PLph{}%
	\IT{docente presso}
	\FR{professeur au}
	\JA{教授}
Federal Law Enforcement Training Center, Georgia, US.

\item \q{I love this book! I have several students reading it at the moment, [and] plan to use it in graduate course.}\footnote{\href{http://go.yurichev.com/17097}{twitter.com/sergeybratus/status/505590326560833536}}
	\RU{Сергей Братусь}%
	\EN{Sergey Bratus}%
	\ES{Sergey Bratus}%
	\PTBRph{}%
	\DE{Sergey Bratus}%
	\PLph{}%
	\IT{Sergey Bratus}
	\FR{Sergey Bratus},
	\JA{セルゲイブラウス}
Research Assistant Professor
	\RU{в отделе Computer Science в}%
	\EN{at the Computer Science Department at}%
	\ES{en el Departamento de Ciencias de la Computaci\'on en}%
	\PTBRph{}%
	\DE{an der Fakultät für Computer Science}
	\PLph{}%
	\IT{presso il dipartimento di Informatica del}
	\FR{dans le Département Informatique du}
Dartmouth College
	\JA{コンピュータサイエンス学部}

\item \q{Dennis @Yurichev has published an impressive (and free!) book on reverse engineering}\footnote{\href{http://go.yurichev.com/17098}{twitter.com/TanelPoder/status/524668104065159169}} Tanel Poder,
	\RU{эксперт по настройке производительности Oracle RDBMS}%
	\EN{Oracle RDBMS performance tuning expert}%
	\ES{experto en afinaci\'on de rendimiento de Oracle RDBMS}%
	\PTBRph{}%
	\DE{Oracle RDBMS Performacence-Tuning Experte}%
	\PLph{}
	\IT{esperto di ottimizzazione di Oracle RDBMS}
	\FR{expert en optimisation des performances Oracle RDBMS}
	\JA{オラクルRDBMSパフォーマンスチューニングエキスパート}.

\item \q{This book is a kind of Wikipedia to beginners...} Archer, Chinese Translator, IT Security Researcher.

\RU{\item \q{Прочел Вашу книгу~--- отличная работа, рекомендую на своих курсах студентам
в качестве учебного пособия}. Николай Ильин, преподаватель в ФТИ НТУУ \q{КПИ} и DefCon-UA}

\item \q{[A] first-class reference for people wanting to learn reverse engineering. And it's free for all.} Mikko Hyppönen, F-Secure.

\end{itemize}

\input{thanks}
\input{FAQ_DE}

\subsection*{Programmieren lernen}

Viele Leute stellen hierzu Fragen.

Es gibt keinen ``goldenen Weg'', aber einige effiziente Möglichkeiten.

Aus meiner eigenen Erfahrung ist dies: lösen von Übungen aus:

\begin{itemize}
	\item \KRBook
	\item Harold Abelson, Gerald Jay Sussman, Julie Sussman -- Structure and Interpretation of Computer Programs
	\item \TAOCP
	\item Niklaus Wirth's books
	\item \RobPikePractice
\end{itemize}

... in C und LISP.
Möglicherweise benutzt du diese Programmiersprachen in der Zukunft nicht mehr, zumindest die meisten professionellen Programmierer tun es nicht. Aber Programmiererfahrung in C und LISP wird langfristig eine wichtige Bedeutung haben.

Das Lesen der Bücher kann übersprungen und die Bücher nur zu Rate gezogen werden wenn das Gefühl aufkommt, dass einige Grundkenntnisse zum Lösen der Aufgaben fehlen.

Dies mag Jahre dauern, vielleicht auch ein Leben lang. Trotzdem ist es noch schneller als Modeerscheinungen hinterherzulaufen.

Der Erfolg dieser Bücher ist vielleicht in der Tatsache begründet, dass die Autoren Lehrer sind und das komplette Material zunächst von Studenten genutzt wurde.

Bezüglich LISP, empfehle ich persönlich Racket (einen Scheme-Dialekt). Dies ist aber stark abhängig vom Geschmack jedes Einzelnen.

Manche sagen, dass das Verständnis von Assembler auch hilfreich ist, wenn man es niemals nutzen wird.
Das ist wahr.
Aber dies ist der Weg für die meisten ``geeks'' und kann erst mal aufgeschoben werden.

Selbstlerner (inklusive dem Autoren dieses Buchs) stehen oft vor dem Problem, sich erst mit schwierigen Aufgaben zu beschäftigen, die einfachen jedoch zu überspringen.
Das ist ein großer Fehler.
Als Vergleich mit Sport oder Musik -- niemand startet mit 100Kg-Gewichten oder Paganini's Caprices.
Ich würde sagen -- man kann die Lösung eines Problems angehen, wenn man diese schon grob im Kopf skizzieren kann.

\begin{framed}
\begin{quotation}
I think the art of doing research consists largely of asking questions,
and sometimes answering them. Learn how to repeatedly pose miniquestions
that represent special cases of the big questions you are hoping to solve.

When you begin to explore some area, you take baby steps at first, building
intuition about that territory. Play with many small examples, trying to
get a complete understanding of particular parts of the general situation.

In that way you learn many properties that are true and many properties
that are false. That gives guidance about directions that are fruitful
versus directions to avoid.

Eventually your brain will have learned how to take larger and larger steps.
And shazam, you’ll be ready to take some giant steps and solve the big problem.

But don’t stop there! At this point you’ll be one of very few people in the
world who have ever understood your problem area so well. It will therefore
be your responsibility to discover what else is true, in the neighborhood
of that problem, using the same or similar methods to what your brain
can now envision. Take your results to their “natural boundary” (in a sense
analogous to the natural boundary where a function of a complex variable
ceases to be analytic).

My little book Surreal Numbers provides an authentic example of research
as it is happening. The characters in that story make false starts and
useful discoveries in exactly the same order as I myself made those false starts
and useful discoveries, when I first studied John Conway’s fascinating
axioms about number systems — his amazingly simple axioms that go
significantly beyond real-valued numbers.

(One of the characters in that book tends to succeed or fail by brute force
and patience; the other is more introspective, and able to see a bigger
picture. Both of them represent aspects of my own activities while doing
research. With that book I hoped to teach research skills “by osmosis”,
as readers observe a detailed case study.)

Surreal Numbers deals with a purely mathematical topic, not especially close
to computer science; it features algebra and logic, not algorithms.
When algorithms become part of the research, a beautiful new dimension
also comes into play: Algorithms can be implemented on computers!

I strongly recommend that you look for every opportunity to write programs
that carry out all or a part of whatever algorithms relate to your research.
In my experience the very act of writing such a program never fails to
deepen my understanding of the problem area.
\end{quotation}
\end{framed}

( Donald E. Knuth -- \url{https://theorydish.blog/2018/02/01/donald-knuth-on-doing-research/} )

Viel Glück!

\subsection*{Über die koreanische Übersetzung}

Im Januar 2015 hat die Acorn Publishing Company (\href{http://www.acornpub.co.kr}{www.acornpub.co.kr}) in Süd-Korea
viel Aufwand in die Übersetzung und Veröffentlichung meines Buchs ins Koreanische (mit Stand August 2014) investiert.

Es ist jetzt unter dieser \href{http://go.yurichev.com/17343}{Webseite} verfügbar.

\iffalse
\begin{figure}[H]
\centering
\includegraphics[scale=0.3]{acorn_cover.jpg}
\end{figure}
\fi

Der Übersetzer ist Byungho Min (\href{http://go.yurichev.com/17344}{twitter/tais9}).
Die Cover-Gestaltung wurde von meinem künstlerisch begabten Freund Andy Nechaevsky erstellt:
\href{http://go.yurichev.com/17023}{facebook/andydinka}.
Die Acorn Publishing Company besetzt die Urheberrechte an der koreanischen Übersetzung.

Wenn Sie also ein \emph{echtes} Buch in Ihrem Buchregal auf koreanisch haben und 
mich bei meiner Arbeit unterstützen wollen, können Sie das Buch nun kaufen.

\subsection*{Über die persische Übersetzung (Farsi)}

In 2016 wurde das Buch von Mohsen Mostafa Jokar übersetzt, der in der iranischen Community
auch für die Übersetzung des Radare
Handbuchs\footnote{\url{http://rada.re/get/radare2book-persian.pdf}} bekannt ist).
Es ist auf der Homepage des Verlegers\footnote{\url{http://goo.gl/2Tzx0H}} (Pendare Pars)
verfügbar.

Hier ist ein Link zu einem 40-seitigen Auszug: \url{https://beginners.re/farsi.pdf}.

National Library of Iran registration information: \url{http://opac.nlai.ir/opac-prod/bibliographic/4473995}.

\subsection*{Über die chinesische Übersetzung}

Im April 2017 wurde die chinesische Übersetzung von Chinese PTPress fertiggestellt, bei denen
auch das Copyright liegt.

Die chinesische Version kann hier bestellt werden: \url{http://www.epubit.com.cn/book/details/4174}.
Ein Auszug und die Geschichte der Übersetzung kann hier gefunden werden: \url{http://www.cptoday.cn/news/detail/3155}.

Der Hauptübersetzer ist Archer, dem der Autor sehr iel verdankt.
Archer war extrem akribisch (im positiven Sinne) und meldete
den Großteil der bekannten Fehler, was für Literatur wie dieses Buch extrem wichtig ist.
Der Autor empfiehlt diesen Service jedem anderen Autor!

Die Mitarbeiter von \href{http://www.antiy.net/}{Antiy Labs} halfen ebenfalls bei der Übersetzung.
\href{http://www.epubit.com.cn/book/onlinechapter/51413}{Hier ist das Vorwort} von ihnen.
