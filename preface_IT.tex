\section*{Prefazione}

\subsection*{Da cosa derivano i due titoli?}
\label{TwoTitles}

Il libro era chiamato ``Reverse Engineering for Beginners'' nel periodo 2014-2018, ma ho sempre sospettato che questo restringesse troppo i potenziali lettori.

Nel campo Infosec le persone conoscono il ``reverse engineering'', ma raramente ho sentito la parola ``assembler'' da parte loro.

Similmente, il termine ``reverse engineering'' è in qualche modo criptico per il resto dei programmatori, ma sanno cos'è l'``assembler''.

A luglio 2018, per esperimento, ho cambiato il titolo in ``Assembly Language for Beginners''
e postato il link sul sito Hacker News \footnote{\url{https://news.ycombinator.com/item?id=17549050}}, ed il libro ha avuto un buon successo.

Quindi è così, il libro adesso ha due titoli.

Tuttavia, ho modificato il secondo titolo in ``Understanding Assembly Language'', perchè qualcuno aveva già scritto un libro ``Assembly Language for Beginners''.
Inoltre lla gente dice che ``for Beginners'' sembra un po' sarcastico per un libro di \textasciitilde{}1000 pagine.

I due libri sono differenti solo per il titolo, il nome del file (UAL-XX.pdf versus RE4B-XX.pdf), l'URL ed un paio di pagine iniziali.

\subsection*{Sul reverse engineering}

Esistono diversi significati per il termine \q{\gls{reverse engineering}}:

1) Il reverse engineering del software; riguardo la ricerca su programmi compilati

2) La scansione di strutture 3D e la successiva manipolazione digitale necessaria alla loro riproduzione

3) Ricreare strutture in \ac{DBMS}

Questo libro riguarda il primo significato.

\subsection*{Prerequisiti}

Conoscenza di base del C \ac{PL}.
Letture raccomandate: \myref{CCppBooks}.

\subsection*{Esercizi e compiti}

\dots
possono essere trovati su: \url{http://challenges.re}.

\subsection*{L'Autore}
\begin{tabularx}{\textwidth}{ l X }

\raisebox{-\totalheight}{
\includegraphics[scale=0.60]{Dennis_Yurichev.jpg}
}

&
Dennis Yurichev è un reverse engineer e programmatore.
Può essere contattato via mail: \textbf{\EMAIL{}}. % o Skype: \textbf{dennis.yurichev}.

% FIXME: no link. \tablefootnote doesn't work
\end{tabularx}

% subsections:
\subsection*{%
	\RU{Отзывы об этой книге}%
	\EN{Praise for this book}%
	\ES{Elogios para}% TBT
	\PTBRph{}%
	\DE{Lob für}% TBT
	\PLph{}%
	\IT{Elogi per questo libro}%
	\FR{Éloges de ce livre}%
	\JA{賛辞}% TBT
}

\begin{itemize}

\item \q{Now that Dennis Yurichev has made this book free (libre), it is a contribution to the world of free knowledge and free education.} Richard M. Stallman,
\EN{GNU founder, software freedom activist.}
\RU{Основатель GNU, активист в области свободного ПО.}
\FR{Fondateur de GNU, militant pour la liberté des logiciels}
\IT{Fondatore di GNU, attivista del software libero.}
\JA{GNU創設者、自由なソフトウェアの活動家}

% expanded URLs to make it more robust for printouts. In electronic editions people will click anyway, so tracking will keep working
\item \q{It's very well done .. and for free .. amazing.}\footnote{\href{http://go.yurichev.com/17095}{twitter.com/daniel\_bilar/status/436578617221742593}} Daniel Bilar, Siege Technologies, LLC.

\item \q{... excellent and free}\footnote{\href{http://go.yurichev.com/17096}{twitter.com/petefinnigan/status/400551705797869568}} Pete Finnigan,%
	\RU{гуру по безопасности}%
	\ES{gur\'u de seguridad en}%
	\PTBRph{}%
	\DE{Security-Guru}%
	\PLph{}%
	\IT{Guru della sicurezza}
	\FR{gourou de la sécurité}
    \JA{セキュリティグル}
\oracle
	\EN{security guru}.

\item \q{... [the] book is interesting, great job!} Michael Sikorski,
	\RU{автор книги}%
	\EN{author of}%
	\ES{autor de}%
	\PTBRph{}%
	\DE{Autor von}%
	\PLph{}%
	\IT{Autore di}
	\FR{auteur de}
	\JA{以下の著作の著者です}
\emph{Practical Malware Analysis: The Hands-On Guide to Dissecting Malicious Software}.

\item \q{... my compliments for the very nice tutorial!} Herbert Bos,
	\RU{профессор университета}%
	\EN{full professor at the}%
	\ES{catedr\'atico de tiempo completo en la}%
	\PTBRph{}%
	\DE{Professor an der}%
	\PLph{}%
	\IT{professore presso la}
	\FR{professeur à temps complet à}
	\JA{教授}
Vrije Universiteit Amsterdam,
	\RU{соавтор}%
	\EN{co-author of}%
	\ES{coautor de}%
	\PTBRph{}%
	\DE{Co-Autor von}%
	\PLph{}%
	\IT{coautore di}
	\FR{co-auteur de}
	\JA{共著者}
\emph{Modern Operating Systems (4th Edition)}.

\item \q{... It is amazing and unbelievable.} Luis Rocha, CISSP / ISSAP, Technical Manager, Network \& Information Security at Verizon Business.

\item \q{Thanks for the great work and your book.} Joris van de Vis,
	\RU{специалист по}%
	\ES{especialista en}%
	\PTBRph{}%
	\DE{Spezialist bei}%
	\PLph{}%
	\IT{specialista di}
	\FR{spécialiste}
SAP Netweaver \& Security
	\EN{specialist}.
	\JA{スペシャリスト}

\item \q{... [a] reasonable intro to some of the techniques.}\footnote{\href{http://go.yurichev.com/17099}{reddit}} Mike Stay,
	\RU{преподаватель в}%
	\EN{teacher at the}%
	\ES{profesor en el}%
	\PTBRph{}%
	\DE{Professor an der}%
	\PLph{}%
	\IT{docente presso}
	\FR{professeur au}
	\JA{教授}
Federal Law Enforcement Training Center, Georgia, US.

\item \q{I love this book! I have several students reading it at the moment, [and] plan to use it in graduate course.}\footnote{\href{http://go.yurichev.com/17097}{twitter.com/sergeybratus/status/505590326560833536}}
	\RU{Сергей Братусь}%
	\EN{Sergey Bratus}%
	\ES{Sergey Bratus}%
	\PTBRph{}%
	\DE{Sergey Bratus}%
	\PLph{}%
	\IT{Sergey Bratus}
	\FR{Sergey Bratus},
	\JA{セルゲイブラウス}
Research Assistant Professor
	\RU{в отделе Computer Science в}%
	\EN{at the Computer Science Department at}%
	\ES{en el Departamento de Ciencias de la Computaci\'on en}%
	\PTBRph{}%
	\DE{an der Fakultät für Computer Science}
	\PLph{}%
	\IT{presso il dipartimento di Informatica del}
	\FR{dans le Département Informatique du}
Dartmouth College
	\JA{コンピュータサイエンス学部}

\item \q{Dennis @Yurichev has published an impressive (and free!) book on reverse engineering}\footnote{\href{http://go.yurichev.com/17098}{twitter.com/TanelPoder/status/524668104065159169}} Tanel Poder,
	\RU{эксперт по настройке производительности Oracle RDBMS}%
	\EN{Oracle RDBMS performance tuning expert}%
	\ES{experto en afinaci\'on de rendimiento de Oracle RDBMS}%
	\PTBRph{}%
	\DE{Oracle RDBMS Performacence-Tuning Experte}%
	\PLph{}
	\IT{esperto di ottimizzazione di Oracle RDBMS}
	\FR{expert en optimisation des performances Oracle RDBMS}
	\JA{オラクルRDBMSパフォーマンスチューニングエキスパート}.

\item \q{This book is a kind of Wikipedia to beginners...} Archer, Chinese Translator, IT Security Researcher.

\RU{\item \q{Прочел Вашу книгу~--- отличная работа, рекомендую на своих курсах студентам
в качестве учебного пособия}. Николай Ильин, преподаватель в ФТИ НТУУ \q{КПИ} и DefCon-UA}

\item \q{[A] first-class reference for people wanting to learn reverse engineering. And it's free for all.} Mikko Hyppönen, F-Secure.

\end{itemize}

\input{thanks}
\input{FAQ_IT}

\subsection*{Come imparare a programmare}

Molte persone se lo chiedono.

Non esiste una via unica preferenziale, ma esistono alcuni modi efficaci.

Per la mia esperienza, basta risolvere gli esercizi su:

\begin{itemize}
\item \KRBook
\item Harold Abelson, Gerald Jay Sussman, Julie Sussman -- Structure and Interpretation of Computer Programs
\item \TAOCP
\item Niklaus Wirth's books
\item \RobPikePractice
\end{itemize}

... in C puro e LISP.
Potresti non utilizzare mai questi linguaggi in futuro.
La maggiorparte dei programmatori commerciali non lo fanno. Ma l'esperienza nella programmazione in C e LISP ti aiuterà enormemente a lungo termine.

Inoltre, puoi saltare la lettura dei libri in se,
salta semplicemente alle sezioni in cui pensi di aver bisogno per capire qualcosa che ti manca per risolvere l'esercizio che stai facendo.

Può richiedere anni nel migliore dei casi, o una vita, ma è comunque più veloce che correre dietro alle mode del momento.

Il successo di questi libri dipende probabilmente dal fatto che i loro autori sono insegnanti
e tutto il materiale è stato verificato sugli studenti.

Riguardo al LISP, personalmente consiglierei Racket (un dialetto). Ma questa è una questione di preferenze.

Secondo alcune persone anche la comprensione del linguaggio assembly è molto utile, anche se non lo utilizzi mai.
Questo è vero.
Ma questa è una strada per i geek più dedicati, e inizialmente può essere rimandata.

Inoltre, gli autodidatti (incluso l'autore di queste righe) spesso hanno la caratteristica di perdere troppo tempo sui problemi più complessi,
saltando quelli più semplici.
Questo è un grande errore.
Pensa allo sport o alla musica -- nessuno inizia con sollvare pesi di 100kg, o a suonare i Capricci di Paganini.
Io direi che -- puoi tentare di affrontare un problema se sei in grado di abbozzare mentalmente una soluzione.

\begin{framed}
\begin{quotation}
I think the art of doing research consists largely of asking questions,
and sometimes answering them. Learn how to repeatedly pose miniquestions
that represent special cases of the big questions you are hoping to solve.

When you begin to explore some area, you take baby steps at first, building
intuition about that territory. Play with many small examples, trying to
get a complete understanding of particular parts of the general situation.

In that way you learn many properties that are true and many properties
that are false. That gives guidance about directions that are fruitful
versus directions to avoid.

Eventually your brain will have learned how to take larger and larger steps.
And shazam, you’ll be ready to take some giant steps and solve the big problem.

But don’t stop there! At this point you’ll be one of very few people in the
world who have ever understood your problem area so well. It will therefore
be your responsibility to discover what else is true, in the neighborhood
of that problem, using the same or similar methods to what your brain
can now envision. Take your results to their “natural boundary” (in a sense
analogous to the natural boundary where a function of a complex variable
ceases to be analytic).

My little book Surreal Numbers provides an authentic example of research
as it is happening. The characters in that story make false starts and
useful discoveries in exactly the same order as I myself made those false starts
and useful discoveries, when I first studied John Conway’s fascinating
axioms about number systems — his amazingly simple axioms that go
significantly beyond real-valued numbers.

(One of the characters in that book tends to succeed or fail by brute force
and patience; the other is more introspective, and able to see a bigger
picture. Both of them represent aspects of my own activities while doing
research. With that book I hoped to teach research skills “by osmosis”,
as readers observe a detailed case study.)

Surreal Numbers deals with a purely mathematical topic, not especially close
to computer science; it features algebra and logic, not algorithms.
When algorithms become part of the research, a beautiful new dimension
also comes into play: Algorithms can be implemented on computers!

I strongly recommend that you look for every opportunity to write programs
that carry out all or a part of whatever algorithms relate to your research.
In my experience the very act of writing such a program never fails to
deepen my understanding of the problem area.
\end{quotation}
\end{framed}

( Donald E. Knuth -- \url{https://theorydish.blog/2018/02/01/donald-knuth-on-doing-research/} )

Buona fortuna!

\subsection*{La traduzione in Coreano}

A gennaio 2015, la Acorn publishing company (\href{http://www.acornpub.co.kr}{www.acornpub.co.kr}) in Corea del Sud ha compiuto un enorme lavoro traducendo e pubblicando
questo libro (aggiornato ad agosto 2014) in Coreano.

Adesso è disponibile sul \href{http://go.yurichev.com/17343}{loro sito}.

\iffalse
\begin{figure}[H]
\centering
\includegraphics[scale=0.3]{acorn_cover.jpg}
\end{figure}
\fi

Il traduttore è Byungho Min (\href{http://go.yurichev.com/17344}{twitter/tais9}).
La copertina è stata creata dall'artistico Andy Nechaevsky, un amico dell'autore:
\href{http://go.yurichev.com/17023}{facebook/andydinka}.
Acorn detiene inoltre i diritti della traduzione in Coreano.

Quindi se vuoi un \emph{vero} libro in Coreano nella tua libreria e
vuoi supportare questo lavoro, è disponibile per l'acquisto.

\subsection*{La traduzione in Persiano/Farsi}

Nel 2016 il libro è stato tradotto da Mohsen Mostafa Jokar (che è anche conosciuto nell comunità iraniana per la traduzione del manuale di Radare \footnote{\url{http://rada.re/get/radare2book-persian.pdf}}).
Potete trovarlo sul sito dell'editore\footnote{\url{http://goo.gl/2Tzx0H}} (Pendare Pars).

Qua c'è il link ad un estratto di 40 pagine: \url{https://beginners.re/farsi.pdf}.

Informazioni nella Libreria Nazionale dell'Iran: \url{http://opac.nlai.ir/opac-prod/bibliographic/4473995}.

\subsection*{La traduzione Cinese}

Ad aprile 2017, la traduzione in Cinese è stata completata da Chinese PTPress. Possiedono inoltre i diritti della traduzione in Cinese.

La versione cinese può essere ordinata a questo indirizzo: \url{http://www.epubit.com.cn/book/details/4174}. Una recensione parziale, con informazioni sulla traduzione è disponibile qua: \url{http://www.cptoday.cn/news/detail/3155}.

Il traduttore principale è Archer, al quale l'autore deve molto. E' stato estremamente meticolosos (in senso buono) ed ha segnalato buona parte degli errori e bug, il che è molto importante in un libro come questo.
L'autore raccomanderebbe i suoi servizi a qualsiasi altro autore!

I ragazzi di \href{http://www.antiy.net/}{Antiy Labs} hanno inoltre aiutato nella traduzione. \href{http://www.epubit.com.cn/book/onlinechapter/51413}{Qua c'è la prefazione} scritta da loro.
