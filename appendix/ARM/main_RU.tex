\mysection{ARM}
\myindex{ARM}

\subsection{Терминология}

ARM изначально разрабатывался как 32-битный \ac{CPU}, 
поэтому \emph{слово} здесь, в отличие от x86, 32-битное.

\begin{description}
	\item[byte] 8-бит.
		Для определения переменных и массива байт используется директива ассемблера DCB.
	\item[halfword] 16-бит. \dittoclosing директива ассемблера DCW.
	\item[word] 32-бит. \dittoclosing директива ассемблера DCD.
	\item[doubleword] 64-бит.
	\item[quadword] 128-бит.
\end{description}

\subsection{Версии}

\begin{itemize}
\item ARMv4: появился режим Thumb.

\item ARMv6: использовался в iPhone 1st gen., iPhone 3G 
(Samsung 32-bit RISC ARM 1176JZ(F)-S поддерживающий Thumb-2)

\item ARMv7: появился Thumb-2 (2003).
Использовался в iPhone 3GS, iPhone 4, iPad 1st gen. (ARM Cortex-A8), iPad 2 (Cortex-A9),
iPad 3rd gen.

\item ARMv7s: Добавлены новые инструкции.
Использовался в iPhone 5, iPhone 5c, iPad 4th gen. (Apple A6).

\item ARMv8: 64-битный процессор, \ac{AKA} ARM64 \ac{AKA} AArch64.
Использовался в iPhone 5S, iPad Air (Apple A7).
В 64-битном режиме, режима Thumb больше нет, только режим ARM (4-байтные инструкции).
\end{itemize}

% sections
\input{appendix/ARM/registers}
\input{appendix/ARM/instructions}
