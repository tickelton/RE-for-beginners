\myindex{x86!\Instructions!RET}
\myindex{MS-DOS}
\item[RET] Revient d'une sous-routine: \TT{POP tmp; JMP tmp}.

En fait, RET
est une macro du langage d'assemblage, sous les environnements Windows et *NIX, elle
est traduite en
RETN (\q{return near})
ou, du temps de MS-DOS, où la mémoire était adressée différemment
(\myref{8086_memory_model}), en RETF (\q{return far}).

\TT{RET} peut avoir un opérande.
Alors il fonctionne comme ceci: \\
\TT{POP tmp; ADD ESP op1; JMP tmp}.
\TT{RET} avec un opérande termine en général les fonctions avec la convention d'appel
\emph{stdcall}, voir aussi: \myref{sec:stdcall}.

