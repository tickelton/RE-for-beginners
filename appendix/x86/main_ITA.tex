\mysection{x86}

\subsection{Terminologia}

Comune per 16-bit (8086/80286), 32-bit (80386, etc.), 64-bit.

\myindex{IEEE 754}
\myindex{MS-DOS}
\begin{description}
	\item[byte] 8-bit.
		La direttiva DB in assembly è utilizzata per definire variabili ed array di byte.
		I byte sono passati nella parte ad 8-bit dei registri: \TT{AL/BL/CL/DL/AH/BH/CH/DH/SIL/DIL/R*L}.
	\item[word] 16-bit.
		La direttiva DW in assembly \dittoclosing.
		Le Word sono passate nella parte a 16-bit dei registri:\\
			\TT{AX/BX/CX/DX/SI/DI/R*W}.
	\item[double word] (\q{dword}) 32-bit.
		La direttiva DD in assembly \dittoclosing.
		Le Double words sono passate nei registri (x86) o nelle parti a 32-bit dei registri (x64).
		Nel codice a 16-bit, le double words sono passate in coppie di registri a 16-bit.
	\item[quad word] (\q{qword}) 64-bit.
		La direttiva DQ in assembly \dittoclosing.
		Negli ambienti a 32-bit, le quad words sono passate in coppie di registri a 32-bit.
	\item[tbyte] (10 bytes) 80-bit o 10 bytes (usati per i registri FPU IEEE 754).
	\item[paragraph] (16 bytes)---termine popolare nell'ambiente MS-DOS. % TODO link to a part about 8086 memory model...
\end{description}

\myindex{Windows!API}

Tipi di dati della stessa dimensione (BYTE, WORD, DWORD) sono gli stessi nelle Windows \ac{API}.

%TBT
\input{appendix/x86/registers} % subsection
\input{appendix/x86/instructions} % subsection
\input{appendix/x86/npad} % subsection
