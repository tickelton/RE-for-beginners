\subsection{\CapitalPICcode}
\myindex{\PICcode}
\myindex{Linux}
\label{sec:PIC}

Wenn der Code von Shared Libraries (.so) unter Linux analysiert wird, findet
man häufig das folgende Code-Muster:

\lstinputlisting[style=customasmx86,caption=libc-2.17.so x86]{OS/tmp1.asm}

Alle Zeiger auf Zeichenketten sind durch Konstanten und den Wert in \EBX korrigiert,
welcher zu Beginn jeder Funktion berechnet wird.

Dies ist sogenannter \ac{PIC}, und hat den Zweck an jeder beliebigen Stelle im Speicher
ausführbar zu sein. Aus diesem Grund können keine absoluten Speicheradressen verwendet werden.

\ac{PIC} war entscheidend in früheren Computer-Systemen und ist es immer noch in Eingebetteten
Systeme ohne virtuelle Speicherverwaltung in denen sich alle Prozesse in einem einzigen durchgängigen
Speicherbereich befinden.

Diese Technik wird auch heute noch in *NIX-Systemen für Shared Libraries verwendet
da diese von mehreren Prozessen genutzt, aber nur einmal in den Speicher geladen werden.
Jeder Prozess kann jedoch die gleiche Bibliothek an verschiedene Adressen \q{mappen}.
Aus diesem Grund muss diese Bibliothek auch ohne Verwendung absoluter Adressen funktionieren.

Machen wir ein sehr einfaches Experiment:

\begin{lstlisting}
#include <stdio.h>

int global_variable=123;

int f1(int var)
{
    int rt=global_variable+var;
    printf ("returning %d\n", rt);
    return rt;
};
\end{lstlisting}

Nachfolgende die kompilierte .so-Datei von GCC 4.7.3. in \IDA:

\begin{lstlisting}
gcc -fPIC -shared -O3 -o 1.so 1.c
\end{lstlisting}

\lstinputlisting[style=customasmx86,caption=GCC 4.7.3]{OS/tmp2.asm}

\newcommand{\retstring}{\emph{<<returning \%d\textbackslash{}n>>}}
\newcommand{\globvar}{\emph{global\_variable}}

Das ist es: die Zeiger auf \retstring{} und \globvar{} werden bei jedem Funktionsaufruf korrigiert.

\par Die \TT{\_\_x86\_get\_pc\_thunk\_bx()}-Funktion gibt in \EBX die Adresse auf eine Stelle nach einen
Aufruf von sich selbst zurück (hier \TT{0x57C}).

Dies ist eine einfache Möglichkeit um den Wert des Programmzählers (\EIP) an einer beliebigen Stelle zu erhalten.
Die Konstante \TT{0x1A84} gehört zu dem Unterschied des Funktionsanfangs und der sogenannten
\emph{Global Offset Table Procedure Linkage Table} (GOT PLT), die Sektion direkt hinter der \emph{Global Offset Table}
(GOT), an der der Zeiger auf \globvar{}.
\IDA zeigt diese Offsets aus Gründen des einfacheren Verständnisses in einer verarbeiteten Form an,
der Code ist aber wie folgt:

\begin{lstlisting}
.text:00000577                 call    __x86_get_pc_thunk_bx
.text:0000057C                 add     ebx, 1A84h
.text:00000582                 mov     [esp+1Ch+var_4], esi
.text:00000586                 mov     eax, [ebx-0Ch]
.text:0000058C                 mov     esi, [eax]
.text:0000058E                 lea     eax, [ebx-1A30h]
\end{lstlisting}

Hier zeigt \EBX auf die \TT{GOT PLT}-Sektion und um den Zeiger auf \globvar{} zu berechnen
(welcher in der \TT{GOT} gesichert ist) muss \TT{0xC} subtrahiert werden.

Um den Zeiger auf die Zeichenkette \retstring{} zu berechnen muss \TT{0x1A30} abgezogen werden.

\myindex{x86-64}
\myindex{x86!\Registers!RIP}

Übrigens ist dies der Grund warum die AMD64-Anweisungen RIP\footnote{Programmzähler in AMD64}-relative Adressierung
unterstützen: sie vereinfachen den PIC-Code.

Nachfolgend der selbe C-Code mit der gleichen GCC-Version, jedoch für x64 kompiliert.

\myindex{objdump}
\IDA würde den resultierenden Code vereinfachen aber auch die Details zur RIP-relativen Adressierung
unterdrücken. Um alles sehen zu können wird hier \emph{objdump} anstatt IDA genutzt.

\lstinputlisting[style=customasmx86]{OS/tmp3.asm}

\TT{0x2008b9} ist der Unterschied zwischen der Adresse der Anweisung an \TT{0x720} und \globvar{},
und \TT{0x20} ist der Unterschied zwischen der Adresse der Anweisung an \TT{0x72A} und der
Zeichenkette \retstring{}.

Wie zu sehen ist, die Notwendigkeit die Adressen regelmäßig neu zu berechnen macht die Ausführung
etwas langsamer (auch wenn dies in x64 etwas besser ist).

Es mag also von Vorteil sein, statisch zu linken wenn Geschwindigkeit eine Rolle spielt \InSqBrackets{siehe: \AgnerFogCPP}.

\subsubsection{Windows}
\myindex{Windows!Win32}

Der PIC-Mechanismus wird in Windows-DLLs nicht genutzt. Wenn der Windows-Loader eine DLL an
eine andere Basisadresse laden muss, \q{patched} er diese im Speicher (and den \emph{FIXUP}-Platz)
um alle Adressen zu korrigieren.

Dies bedeutet, dass mehrere Windows-Prozesse eine einmal geladene DLL nicht teilen können
wenn diese an verschiedenen Adressen in verschiedenen Prozess-Speichern sein muss, da
jede Instanz im Speicher die Funktionen an einer festen Adresse erwartet.
