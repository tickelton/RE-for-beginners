\subsubsection{値の集合としての構造体}

構造体が1つの場所に並んでいる変数であることを説明するために、
\emph{tm}構造体の定義を再度見ながら、この例を再作ってみましょう:\lstref{struct_tm}

\lstinputlisting[style=customc]{patterns/15_structs/3_tm_linux/as_array/GCC_tm2.c}

\myindex{\CStandardLibrary!localtime\_r()}
注意:
\TT{tm\_sec}フィールドへのポインタは、\TT{localtime\_r}、
すなわち\q{構造体}の最初の要素に渡されます。

コンパイラは警告します。

\begin{lstlisting}[caption=GCC 4.7.3]
GCC_tm2.c: In function 'main':
GCC_tm2.c:11:5: warning: passing argument 2 of 'localtime_r' from incompatible pointer type [enabled by default]
In file included from GCC_tm2.c:2:0:
/usr/include/time.h:59:12: note: expected 'struct tm *' but argument is of type 'int *'
\end{lstlisting}

それにもかかわらず、これを生成します。

\lstinputlisting[caption=GCC 4.7.3,style=customasmx86]{patterns/15_structs/3_tm_linux/as_array/GCC_tm2.asm}

このコードはこれまで見てきたものと同じもので、
元のソースコードでは構造体であったか、単なる変数の集合であったかは言えません。

そしてこれは動きます。
ただし、これを実際に行うことはお勧めしません。

通常、最適化されていないコンパイラは、
関数内で宣言されたのと同じ順序で変数をローカルスタックに割り当てます。

とはいえ、何の保証もありません。

ちなみに、\TT{tm\_year}、 \TT{tm\_mon}、 \TT{tm\_mday}、
\TT{tm\_hour}、 \TT{tm\_min}変数については、
他のコンパイラが警告することがありますが、\TT{tm\_sec}は初期化されずに使用されます。

実際、コンパイラは、これらが\TT{localtime\_r()}
関数で初期化されることを認識していません。

すべての構造体フィールドが \Tint 型であるため、この例を選択しました。

\TT{SYSTEMTIME}構造体の場合のように、構造体フィールドが16ビット(\TT{WORD})の場合、これは機能しません。
\TT{GetSystemTime()}は、(ローカル変数が32ビット境界で整列されているため)、それらを正しく入力しません。
次のセクションでそれについてもっと読む:
\q{\StructurePackingSectionName} (\myref{structure_packing})

したがって、構造体は、1つの場所に並べて配置された単なる変数の集合です。 
構造体はコンパイラへの命令であり、変数を1つの場所に保持するように指示することができます。 
ところで、非常に初期のCバージョン(1972年以前)には、構造体がまったく存在しませんでした。 \RitchieDevC

デバッガの例はありません。すでに見たのとまったく同じです。

\subsubsection{32ビットワードの配列としての構造体}

\lstinputlisting[style=customc]{patterns/15_structs/3_tm_linux/as_array/GCC_tm3.c}

構造体へのポインタを \Tint{} の配列に\emph{キャスト}します。 
そして、それは動作します! 
例は2014年7月26日23:51:45に実行されます。

\begin{lstlisting}[label=GCC_tm3_output]
0x0000002D (45)
0x00000033 (51)
0x00000017 (23)
0x0000001A (26)
0x00000006 (6)
0x00000072 (114)
0x00000006 (6)
0x000000CE (206)
0x00000001 (1)
\end{lstlisting}

ここでの変数は、
\myref{struct_tm} 構造体の定義で列挙されているのと同じ順序です。

コンパイル方法は次のとおりです。

\lstinputlisting[caption=\Optimizing GCC 4.8.1,style=customasmx86]{patterns/15_structs/3_tm_linux/as_array/GCC_tm3_JA.lst}

実際には、ローカルスタック内のスペースは最初に構造体として扱われ、その後で配列として扱われます。

このポインタを介して構造体のフィールドを変更することも可能です。

そして、この怪しいハッカーっぽいやり方は、プロダクトコードでの使用はお勧めしません。

\mysubparagraph{\Exercise}

練習として、構造体を配列として扱い、現在の月の番号を変更(1つ増やしてください)してみてください。

\subsubsection{配列オブジェクトとしての構造体}

さらに進むことができます。 ポインタをバイトの配列に\emph{キャスト}して、それをダンプしましょう。

\lstinputlisting[style=customc]{patterns/15_structs/3_tm_linux/as_array/GCC_tm4.c}

\begin{lstlisting}
0x2D 0x00 0x00 0x00 
0x33 0x00 0x00 0x00 
0x17 0x00 0x00 0x00 
0x1A 0x00 0x00 0x00 
0x06 0x00 0x00 0x00 
0x72 0x00 0x00 0x00 
0x06 0x00 0x00 0x00 
0xCE 0x00 0x00 0x00 
0x01 0x00 0x00 0x00 
\end{lstlisting}

この例は、2014年7月26日 23:51:45にも実行されています。
\footnote{デモの目的で、日時は同じです。 バイト値は固定されています。}
値は以前のダンプと同じですが、
(\myref{GCC_tm3_output})もちろんこれはリトルエンディアンアーキテクチャなので、一番下のバイトが先になります。
(\myref{sec:endianness})

\lstinputlisting[caption=\Optimizing GCC 4.8.1,style=customasmx86]{patterns/15_structs/3_tm_linux/as_array/GCC_tm4_JA.lst}
