\subsubsection{MIPS}
\label{MIPS_structure_big_endian}

\lstinputlisting[caption=\Optimizing GCC 4.4.5 (IDA),numbers=left,style=customasmMIPS]{patterns/15_structs/4_packing/MIPS_O3_IDA_JA.lst}

構造体フィールドはレジスタ \$A0..\$A3 に入ってから \printf のために \$A1..\$A3 に再整理され、
4番目のフィールド(\$A3 から)は\INS{SW}を使ってローカルスタックを経由して渡されます。

しかし、2つのSRA(\q{Shift Word Right Arithmetic})命令があり、これは \Tchar フィールドを準備します。 
なぜでしょうか?

MIPSはデフォルトではビッグエンディアンアーキテクチャです\myref{sec:endianness}。私たちが動かすDebian Linuxもビッグエンディアンです。

したがって、バイト変数が32ビット構造のスロットに格納されるとき、それらは高位31~24ビットを占有します。

また、\Tchar 変数を32ビット値に拡張する必要がある場合は、それを24ビット右にシフトする必要があります。

\Tchar は符号付き型なので、ここでは論理シフトの代わりに算術シフトが使用されます。
