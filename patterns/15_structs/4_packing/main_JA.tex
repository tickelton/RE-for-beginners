\subsection{\StructurePackingSectionName}
\label{structure_packing}

1つ重要なことは、構造内のフィールドのパッキングです\footnote{参照: \URLWPDA}。

簡単な例を考えてみましょう:

\lstinputlisting[style=customc]{patterns/15_structs/4_packing/packing.c}

見てきたように、2つの \Tchar フィールド(それぞれ1バイト)と2つの \Tint(それぞれ4バイト)があります。

% subsections:
\subsubsection{x86}

このようにコンパイルされます。

\lstinputlisting[caption=MSVC 2012 /GS- /Ob0,label=src:struct_packing_4,numbers=left,style=customasmx86]{patterns/15_structs/4_packing/packing_JA.asm}

構造全体を渡しますが、実際には、構造体は
一時的な領域にコピーされて、(スタック内の領域は10行目に割り当てられ、
次に4つのフィールドはすべて1つずつ、12行目から19行目にコピーされます)
そのポインタ(アドレス)が渡されます。

\ttf{} 関数が構造体を変更するかどうかわからないため、
構造体がコピーされます。 
それが変更された場合、 \main の構造体はそのままでいなければなりません。

私たちは \CCpp ポインタを使うことができました。結果のコードはほぼ同じですが、
コピーは行いません。

次に見るように、各フィールドのアドレスは4バイトの境界に揃えられています。 
だからこそ、各 \Tchar が( \Tint のように)4バイトを占めるのです。なぜでしょうか? 
CPUが整列したアドレスでメモリにアクセスし、メモリからデータをキャッシュする方が簡単であるためです。

しかし、あまり経済的ではありません。

オプション(\TT{/Zp1})(nバイト境界で構造体をパックする \emph{/Zp[n]})で
コンパイルしてみましょう。

\lstinputlisting[caption=MSVC 2012 /GS- /Zp1,label=src:struct_packing_1,numbers=left,style=customasmx86]{patterns/15_structs/4_packing/packing_msvc_Zp1_JA.asm}

Now the structure takes only 10 bytes and each \Tchar value takes 1 byte. What does it give to us?
Size economy. And as drawback~---the CPU accessing these fields slower than it could.

構造体は10バイトしかなく、各 \Tchar 値は1バイト必要です。それは私たちに何を与えるのですか?
サイズ経済。そして欠点として、CPUはこれらのフィールドにアクセスするのが遅くなります。

\label{short_struct_copying_using_MOV}

構造体も \main にコピーされます。フィールド単位ではなく、3つの \MOV ペアを使用して直接10バイトをコピーします。
なぜ4ではないのでしょうか?

コンパイラは、3つの \MOV ペアを使用して10バイトをコピーする方が、2つの32ビットワードと
4つの \MOV ペアを使用して2バイトをコピーするよりも優れていると判断しました。

ちなみに、\TT{memcpy()}関数を呼び出す代わりに \MOV を使用するようなコピーの実装は、
\TT{memcpy()}の呼び出しよりも速いため、広く使用されています。
\myref{copying_short_blocks}

簡単に推測できるように、構造体が多くのソースファイルとオブジェクトファイルで使用されている場合、
構造体パッキングについてはすべて同じ規則でコンパイルする必要があります。

\newcommand{\FNURLMSDNZP}{\footnote{\href{http://go.yurichev.com/17067}
{MSDN: Working with Packing Structures}}}
\newcommand{\FNURLGCCPC}{\footnote{\href{http://go.yurichev.com/17068}
{Structure-Packing Pragmas}}}

各構造体フィールドの配置方法を設定するMSVC \TT{/Zp}オプションの他に、
\TT{\#pragma pack}コンパイラオプションもあります。このオプションはソースコード内で直接定義できます。 
MSVC\FNURLMSDNZP と GCC\FNURLGCCPC{} の両方で利用できます。

16ビットのフィールドで構成される\TT{SYSTEMTIME}構造体に戻りましょう。
私たちのコンパイラは、1バイト境界でパックすることをどうやって知っていますか?

\TT{WinNT.h}ファイルはこれを持っています:

\begin{lstlisting}[caption=WinNT.h,style=customc]
#include "pshpack1.h"
\end{lstlisting}

そしてこれを。

\begin{lstlisting}[caption=WinNT.h,style=customc]
#include "pshpack4.h"                   // 4バイトパッキングがデフォルト
\end{lstlisting}

PshPack1.h ファイルはこのようになっています。

\lstinputlisting[caption=PshPack1.h,style=customc]{patterns/15_structs/4_packing/tmp1.c}

コンパイラは \TT{\#pragma pack} の後で定義される構造体をパックする方法を知らせます。

\clearpage
\myparagraph{\Optimizing MSVC + \olly}
\myindex{\olly}

\olly でこの(最適化された)例を試すことができます。ここに最初のイテレーションがあります:

\begin{figure}[H]
\centering
\myincludegraphics{patterns/10_strings/1_strlen/olly1.png}
\caption{\olly: 最初のイテレーションの開始}
\label{fig:strlen_olly_1}
\end{figure}

\olly がループを見つけ、便宜上、その指示を括弧で\emph{囲んだ}ことがわかります。 
\EAX の右ボタンをクリックして、\q{Follow in Dump}を選択すると、メモリウィンドウが正しい場所にスクロールします。
ここではメモリ内に \q{hello!}という文字列があります。
その後に少なくとも1つのゼロバイトがあり、次にランダムなごみがあります。

\olly が有効なアドレスを持つレジスタを見ると、それは文字列を指しており、
文字列として表示されます。

\clearpage
F8(\stepover)を数回押して、ループの本体の先頭に移動しましょう:

\begin{figure}[H]
\centering
\myincludegraphics{patterns/10_strings/1_strlen/olly2.png}
\caption{\olly: 2回目のイテレーションの開始}
\label{fig:strlen_olly_2}
\end{figure}

\EAX には文字列中の2番目の文字のアドレスが含まれていることがわかります。

\clearpage

我々はループから脱出するためにF8を十分な回数押す必要があります:

\begin{figure}[H]
\centering
\myincludegraphics{patterns/10_strings/1_strlen/olly3.png}
\caption{\olly: 計算すべきポインタの差}
\label{fig:strlen_olly_3}
\end{figure}

% FIXME:
\EAX には文字列の直後に0バイトのアドレスが含まれることがわかりました。一方、
\EDX は変更されていないので、まだ文字列の先頭を指しています。

これらの2つのアドレスの差がここで計算されています。

\clearpage
\SUB 命令が実行されました。

\begin{figure}[H]
\centering
\myincludegraphics{patterns/10_strings/1_strlen/olly4.png}
\caption{\olly: \EAX がデクリメントされる}
\label{fig:strlen_olly_4}
\end{figure}

ポインタの違いは \EAX レジスタにあります(値は7)。
確かに、 \q{hello!}文字列の長さは6ですが、7にはゼロバイトが含まれています。
しかし、\TT{strlen()}は文字列中のゼロ以外の文字数を返さなければなりません。
したがって、デクリメントが実行され、関数が戻ります。


\subsubsection{ARM + \OptimizingKeilVI (\ARMMode)}

\begin{lstlisting}[caption=\OptimizingKeilVI (\ARMMode),style=customasmARM]
02 0C C0 E3          BIC     R0, R0, #0x200
01 09 80 E3          ORR     R0, R0, #0x4000
1E FF 2F E1          BX      LR
\end{lstlisting}

\myindex{ARM!\Instructions!BIC}
\INS{BIC} (\emph{BItwise bit Clear})は特定のビットをクリアする命令です。
\AND 命令に似ていますが、反転したオペランドを使用します。
つまり、 \NOT+\AND 命令ペアに類似しています。

\myindex{ARM!\Instructions!ORR}
\INS{ORR} is \q{logical or}, analogous to \OR in x86.

\INS{ORR} は\q{論理OR}です。x86の \OR に類似しています。

ここまでは簡単です。

\subsubsection{ARM + \OptimizingKeilVI (\ThumbMode)}

\begin{lstlisting}[caption=\OptimizingKeilVI (\ThumbMode),style=customasmARM]
01 21 89 03          MOVS    R1, 0x4000
08 43                ORRS    R0, R1
49 11                ASRS    R1, R1, #5   ; 0x200を生成しR1に配置する
88 43                BICS    R0, R1
70 47                BX      LR
\end{lstlisting}

KeilはThumbモードのコードが\TT{0x4000}から\TT{0x200}になり、
\TT{0x200}を任意のレジスタに書き込むコードよりも
コンパクトであると判断したようです。
% TODO1 пример, как компилятор при помощи сдвигов оптизирует такое: a=0x1000; b=0x2000; c=0x4000, etc

\myindex{ARM!\Instructions!ASRS}

したがって、\INS{ASRS}(\ASRdesc)の助けを借りて、この値は$\TT{0x4000} \gg 5$として計算されます。

\subsubsection{ARM + \OptimizingXcodeIV (\ARMMode)}
\label{anomaly:LLVM}
\myindex{\CompilerAnomaly}

\begin{lstlisting}[caption=\OptimizingXcodeIV (\ARMMode),label=ARM_leaf_example3,style=customasmARM]
42 0C C0 E3          BIC             R0, R0, #0x4200
01 09 80 E3          ORR             R0, R0, #0x4000
1E FF 2F E1          BX              LR
\end{lstlisting}

LLVMが生成したコードは、次のようになるかもしれません。

\begin{lstlisting}[style=customc]
    REMOVE_BIT (rt, 0x4200);
    SET_BIT (rt, 0x4000);
\end{lstlisting}

そして、これはまさに必要としているものです。
しかしなぜ\TT{0x4200}なのでしょうか。
おそらく、LLVMのオプチマイザが生成した生成物でしょう。

\footnote{Apple Xcode 4.6.3にバンドルされたLLVM build 2410.2.00です}

コンパイラのオプチマイザのエラーかもしれませんが、生成されたコードはともあれ正しく動作します。

コンパイラのアノマリについての詳細はこちら~(\myref{anomaly:Intel})

Thumbモードでの \OptimizingXcodeIV は同じコードを生成します。

\subsubsection{ARM: \INS{BIC} 命令についての詳細}
\myindex{ARM!\Instructions!BIC}

例を少し改変してみましょう。

\begin{lstlisting}[style=customc]
int f(int a)
{
    int rt=a;

    REMOVE_BIT (rt, 0x1234);

    return rt;
};
\end{lstlisting}

ARMモードの最適化Keil 5.03 の結果は
以下のようになります。

\begin{lstlisting}[style=customasmARM]
f PROC
        BIC      r0,r0,#0x1000
        BIC      r0,r0,#0x234
        BX       lr
        ENDP
\end{lstlisting}

\INS{BIC}命令が2つあります。すなわち、ビット\TT{0x1234}は2パスでクリアされます。

なぜなら1つの\INS{BIC}命令では\TT{0x1234}をエンコードすることが不可能だからです。
しかし、\TT{0x1000} と \TT{0x234}をエンコードすることはできます。

\subsubsection{ARM64: \Optimizing GCC (Linaro) 4.9}

\Optimizing GCCコンパイラでARM64をコンパイルするなら\INS{BIC}の代わりに \AND 命令を使用できます。

\begin{lstlisting}[caption=\Optimizing GCC (Linaro) 4.9,style=customasmARM]
f:
	and	w0, w0, -513	; 0xFFFFFFFFFFFFFDFF
	orr	w0, w0, 16384	; 0x4000
	ret
\end{lstlisting}

\subsubsection{ARM64: \NonOptimizing GCC (Linaro) 4.9}

\NonOptimizing GCC はもっと冗長なコードを生成しますが、最適化されたように動作します。

\begin{lstlisting}[caption=\NonOptimizing GCC (Linaro) 4.9,style=customasmARM]
f:
	sub	sp, sp, #32
	str	w0, [sp,12]
	ldr	w0, [sp,12]
	str	w0, [sp,28]
	ldr	w0, [sp,28]
	orr	w0, w0, 16384	; 0x4000
	str	w0, [sp,28]
	ldr	w0, [sp,28]
	and	w0, w0, -513	; 0xFFFFFFFFFFFFFDFF
	str	w0, [sp,28]
	ldr	w0, [sp,28]
	add	sp, sp, 32
	ret
\end{lstlisting}

\subsubsection{MIPS}
\label{MIPS_structure_big_endian}

\lstinputlisting[caption=\Optimizing GCC 4.4.5 (IDA),numbers=left,style=customasmMIPS]{patterns/15_structs/4_packing/MIPS_O3_IDA_JA.lst}

構造体フィールドはレジスタ \$A0..\$A3 に入ってから \printf のために \$A1..\$A3 に再整理され、
4番目のフィールド(\$A3 から)は\INS{SW}を使ってローカルスタックを経由して渡されます。

しかし、2つのSRA(\q{Shift Word Right Arithmetic})命令があり、これは \Tchar フィールドを準備します。 
なぜでしょうか?

MIPSはデフォルトではビッグエンディアンアーキテクチャです\myref{sec:endianness}。私たちが動かすDebian Linuxもビッグエンディアンです。

したがって、バイト変数が32ビット構造のスロットに格納されるとき、それらは高位31~24ビットを占有します。

また、\Tchar 変数を32ビット値に拡張する必要がある場合は、それを24ビット右にシフトする必要があります。

\Tchar は符号付き型なので、ここでは論理シフトの代わりに算術シフトが使用されます。


\subsubsection{もう一言}

関数の引数として構造体を渡すのは(構造体へのポインタを渡すのではなく)構造体のフィールドを
1つ1つ渡すのと同じです。

構造体のフィールドがデフォルトでパックされる場合、f()関数は以下のように書き換える可能です。

\begin{lstlisting}[style=customc]
void f(char a, int b, char c, int d)
{
    printf ("a=%d; b=%d; c=%d; d=%d\n", a, b, c, d);
};
\end{lstlisting}

そして同じコードになります。
