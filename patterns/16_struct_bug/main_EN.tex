\mysection{The classic \emph{struct} bug}

This is a classic \emph{struct} bug.

Here is a definition:

\lstinputlisting[style=customc]{patterns/16_struct_bug/old.h}

And then C files:

\lstinputlisting[style=customc]{patterns/16_struct_bug/setter.c}

\lstinputlisting[style=customc]{patterns/16_struct_bug/printer.c}

So far so good.

Now you add a third field into the structure, some place between two fields:

\lstinputlisting[style=customc]{patterns/16_struct_bug/new.h}

And you probably modify setter() function, but forget about printer():

\lstinputlisting[style=customc]{patterns/16_struct_bug/setter_new.c}

You compile your project, but the C file where printer() is residing, isn't recompiling, because your \ac{IDE} or build system
has no idea that module depends on a \emph{test} struct definition.
Maybe because \verb|#include <new.h>| is omitted.
Or maybe, \verb|new.h| header file is included \verb|in printer.c| via some other header file.
The object file remains untouched (\ac{IDE} thinks it doesn't need to be recompiled),
while \verb|setter()| function is already a new version.
These two object files (old and new) eventually linked into an executable file.

Then you run it, and the \verb|setter()| sets 3 fields at +0, +4 and +8 offsets.
However, the \verb|printer()| only knows about 2 fields, and gets them from +0 and +4 offsets during printing.

This leads to very obscure and nasty bugs.
The reason is that \ac{IDE} or build system or Makefile doesn't know the fact that both C files (or modules) depends on the header
file with \emph{test} definition.
A popular remedy is to clean everything and recompile.

This is true for C++ classes as well, since they works just like structures: \ref{CppClasses}.

This is a C/C++'s malady, and a source of criticism, yes.
Many newer \ac{PL}s has better support of modules and interfaces.
But keep in mind, when C compiler was created: 1970s, on old PDP computers.
So everything was simplified down to this.

