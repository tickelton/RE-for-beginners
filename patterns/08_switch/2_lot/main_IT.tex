\subsection{Molti casi}

Se uno statement \TT{switch()} contiene molti casi, per il compilatore non è molto conveniente enettere codice troppo lungo con un sacco di 
istruzioni \JE/\JNE.

\lstinputlisting[label=switch_lot_c,style=customc]{patterns/08_switch/2_lot/lot.c}

\input{patterns/08_switch/2_lot/lot_x86_IT}
\input{patterns/08_switch/2_lot/lot_ARM_IT}
\input{patterns/08_switch/2_lot/lot_MIPS_IT}

\subsubsection{\Conclusion{}}

Stuttura approssimativa di \emph{switch()}:

% TODO: ARM, MIPS skeleton
\lstinputlisting[caption=x86,style=customasmx86]{patterns/08_switch/2_lot/skel1_IT.lst}

Il salto agli indirizzi nella tabella di jump può anche essere implementato usando questa istruzione: \\
\TT{JMP jump\_table[REG*4]}.
oppure \TT{JMP jump\_table[REG*8]} in x64.

Una \emph{jumptable} è semplicemente un array di puntatori, come quello descritto più avanti: \myref{array_of_pointers_to_strings}.
