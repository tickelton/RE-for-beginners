\subsubsection{Address patching (Win64)}

Se il nostro esempio venisse compilato in MSVC 2013 utilizzando lo switch \TT{\textbackslash{}MD}
(che significa un eseguibile più piccolo a causa del link dei file \TT{MSVCR*.DLL}), la funzione \main verrebbe prima, e può essere trovata facilmente:

\begin{figure}[H]
\centering
\myincludegraphics{patterns/01_helloworld/hiew_incr1.png}
\caption{Hiew}
\label{}
\end{figure}

Come esperimento, possiamo \glslink{increment}{incrementare} l'indirizzo di 1:

\begin{figure}[H]
\centering
\myincludegraphics{patterns/01_helloworld/hiew_incr2.png}
\caption{Hiew}
\label{}
\end{figure}

Hiew mostra \q{ello, world}.
E quando lanciamo l'eseguibile modificato, viene stampata proprio questa stringa.

\subsubsection{Scegliere un'altra stringa dall'immagine binaria (Linux x64)}

Il file binario che si ottiene compilando il nostro esempio tramite GCC 5.4.0 su Linux x64 contiene molte altre stringhe di testo.
Si tratta principalmente di nomi di funzioni e librerie importate.

Esegui objdump per ottenere il contenuto di tutte le sezioni del file compilato:

\begin{lstlisting}[basicstyle=\ttfamily, mathescape]
$\$$ objdump -s a.out

a.out:     file format elf64-x86-64

Contents of section .interp:
 400238 2f6c6962 36342f6c 642d6c69 6e75782d  /lib64/ld-linux-
 400248 7838362d 36342e73 6f2e3200           x86-64.so.2.
Contents of section .note.ABI-tag:
 400254 04000000 10000000 01000000 474e5500  ............GNU.
 400264 00000000 02000000 06000000 20000000  ............ ...
Contents of section .note.gnu.build-id:
 400274 04000000 14000000 03000000 474e5500  ............GNU.
 400284 fe461178 5bb710b4 bbf2aca8 5ec1ec10  .F.x[.......^...
 400294 cf3f7ae4                             .?z.

...
\end{lstlisting}

Non è un problema passare l'indirizzo della stringa di test \q{/lib64/ld-linux-x86-64.so.2} a \TT{printf()}:

\begin{lstlisting}[style=customc]
#include <stdio.h>

int main()
{
    printf(0x400238);
    return 0;
}
\end{lstlisting}

E' difficile da credere, ma questo codice stampa la stringa citata prima.

Se cambiassi l'indirizzo a \TT{0x400260}, verrebbe stampata la stringa \q{GNU}.
Questo indirizzo è corretto per la mia specifica versione di GCC, GNU toolset, etc.
Sul tuo sistema, l'eseguibile potrebbe essere leggermente differente, e anche tutti gli indirizzi sarebbero differenti.
Inoltre, aggiungendo o rimuovendo del codice in/da questo codice sorgente probabilmente sposterebbe tutti gli indirizzi in avanti o indietro.
