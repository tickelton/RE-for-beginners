\myparagraph{MSVC}

MSVC 2010でコンパイルしましょう。

\lstinputlisting[caption=MSVC 2010: \ttf{},style=customasmx86]{patterns/12_FPU/1_simple/MSVC_JA.asm}

\FLD はスタックから8バイトを取り出し、その数値を\ST{0}レジスタにロードし、内部80ビットフォーマット
(\emph{拡張精度})に自動的に変換します。

\myindex{x86!\Instructions!FDIV}

\FDIV は、\ST{0}の値をアドレス\GTT{\_\_real@40091eb851eb851f}~に格納された数値で除算します。
値3.14はそこにエンコードされます。
アセンブリ構文は浮動小数点数をサポートしていないので、64ビットIEEE 754形式での3.14の16進表現です。

\FDIV \ST{0}の実行後に\gls{quotient}が保持されます。

\myindex{x86!\Instructions!FDIVP}

ちなみに、 \FDIVP 命令もあります。これは、\ST{1}を\ST{0}で除算し、
これらの値をスタックからポップし、その結果をプッシュします。
あなたがForth言語\FNURLFORTH を知っていれば、
すぐにこれがスタックマシン\FNURLSTACK であることがわかります。

後続の \FLD 命令は、 $b$ の値をスタックにプッシュします。

その後、商は\ST{1}に置かれ、\ST{0}は $b$ の値を持ちます。

\myindex{x86!\Instructions!FMUL}

次の \FMUL 命令は乗算を行います。\ST{0}の $b$ は\GTT{\_\_real@4010666666666666}
(そこには4.1が入る)の値で乗算され、結果は\ST{0}レジスタに残ります。

\myindex{x86!\Instructions!FADDP}

最後の \FADDP 命令は、スタックの先頭に2つの値を加算し、結果を\ST{1}に格納した後、
\ST{0}の値をポップし、\ST{0}のスタックの先頭に結果を残します。

関数はその結果を\ST{0}レジスタに戻す必要があるため、
\FADDP 後の関数エピローグ以外の命令はありません。

\clearpage
\myparagraph{\Optimizing MSVC + \olly}
\myindex{\olly}

\olly でこの(最適化された)例を試すことができます。ここに最初のイテレーションがあります:

\begin{figure}[H]
\centering
\myincludegraphics{patterns/10_strings/1_strlen/olly1.png}
\caption{\olly: 最初のイテレーションの開始}
\label{fig:strlen_olly_1}
\end{figure}

\olly がループを見つけ、便宜上、その指示を括弧で\emph{囲んだ}ことがわかります。 
\EAX の右ボタンをクリックして、\q{Follow in Dump}を選択すると、メモリウィンドウが正しい場所にスクロールします。
ここではメモリ内に \q{hello!}という文字列があります。
その後に少なくとも1つのゼロバイトがあり、次にランダムなごみがあります。

\olly が有効なアドレスを持つレジスタを見ると、それは文字列を指しており、
文字列として表示されます。

\clearpage
F8(\stepover)を数回押して、ループの本体の先頭に移動しましょう:

\begin{figure}[H]
\centering
\myincludegraphics{patterns/10_strings/1_strlen/olly2.png}
\caption{\olly: 2回目のイテレーションの開始}
\label{fig:strlen_olly_2}
\end{figure}

\EAX には文字列中の2番目の文字のアドレスが含まれていることがわかります。

\clearpage

我々はループから脱出するためにF8を十分な回数押す必要があります:

\begin{figure}[H]
\centering
\myincludegraphics{patterns/10_strings/1_strlen/olly3.png}
\caption{\olly: 計算すべきポインタの差}
\label{fig:strlen_olly_3}
\end{figure}

% FIXME:
\EAX には文字列の直後に0バイトのアドレスが含まれることがわかりました。一方、
\EDX は変更されていないので、まだ文字列の先頭を指しています。

これらの2つのアドレスの差がここで計算されています。

\clearpage
\SUB 命令が実行されました。

\begin{figure}[H]
\centering
\myincludegraphics{patterns/10_strings/1_strlen/olly4.png}
\caption{\olly: \EAX がデクリメントされる}
\label{fig:strlen_olly_4}
\end{figure}

ポインタの違いは \EAX レジスタにあります(値は7)。
確かに、 \q{hello!}文字列の長さは6ですが、7にはゼロバイトが含まれています。
しかし、\TT{strlen()}は文字列中のゼロ以外の文字数を返さなければなりません。
したがって、デクリメントが実行され、関数が戻ります。

