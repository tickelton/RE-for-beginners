\subsubsection{MIPS}

\myindex{MIPS!\Registers!FCCR}
MIPSプロセッサのコプロセッサには条件ビットがあり、これをFPUにセットしてCPUでチェックすることができます。

以前のMIPSには1つの条件ビット(FCC0と呼びます)があり、後のモデルには8つのビット(FCC7-FCC0と呼びます)があります。

このビット(または複数のビット)はFCCRと呼ばれるレジスタに配置されています。

\lstinputlisting[caption=\Optimizing GCC 4.4.5 (IDA),style=customasmMIPS]{patterns/12_FPU/3_comparison/MIPS_O3_IDA_JA.lst}

\myindex{MIPS!\Instructions!C.LT.D}
\INS{C.LT.D}は2つの値を比較します。 
\GTT{LT}は\q{Less Than}の条件です。
\GTT{D}は \Tdouble 型の値を意味します。 
比較の結果に応じて、FCC0条件ビットはセットまたはクリアされます。

\myindex{MIPS!\Instructions!BC1T}
\myindex{MIPS!\Instructions!BC1F}
\INS{BC1T} checks the FCC0 bit and jumps if the bit is set.
\GTT{T} means that the jump is to be taken if the bit is set (\q{True}).
There is also the instruction \INS{BC1F} which jumps if the bit is cleared (\q{False}).

\INS{BC1T}はFCC0ビットをチェックし、ビットがセットされていればジャンプします。 
\GTT{T}は、ビットがセット(\q{True})されている場合にジャンプが行われることを意味します。 
ビットがクリアされるとジャンプする\INS{BC1F}命令もあります。(\q{False})

ジャンプに応じて、関数引数の1つが \$F0 に配置されます。
