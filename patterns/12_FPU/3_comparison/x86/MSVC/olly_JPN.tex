\clearpage
\mysubparagraph{最初の \olly の例: a=1.2 と b=3.4}
\myindex{\olly}

\olly で例をロードしてみましょう。

\begin{figure}[H]
\centering
\myincludegraphics{patterns/12_FPU/3_comparison/x86/MSVC/olly1_1.png}
\caption{\olly: 最初の \FLD が実行される}
\label{fig:FPU_comparison_case1_olly1}
\end{figure}

関数の引数:$a=1.2$ と $b=3.4$ (スタックで見ることができます。32ビット値の2組のペアです)
$b$ (3.4) は \ST{0}にすでにロードされています。
そして \FCOMP が実行されます。
\olly は次の \FCOMP の引数を表示します。スタックにあります。

\clearpage
\FCOMP が実行されます。

\begin{figure}[H]
\centering
\myincludegraphics{patterns/12_FPU/3_comparison/x86/MSVC/olly1_2.png}
\caption{\olly: \FCOMP が実行される}
\label{fig:FPU_comparison_case1_olly2}
\end{figure}

\ac{FPU}条件フラグの状態を見ることができます。すべて0です。
ポップされた値は\ST{7}に反映されます。この理由については前に書きました:
\myref{FPU_is_rather_circular_buffer}.

\clearpage
\FNSTSW が実行されます。
\begin{figure}[H]
\centering
\myincludegraphics{patterns/12_FPU/3_comparison/x86/MSVC/olly1_3.png}
\caption{\olly: \FNSTSW が実行される}
\label{fig:FPU_comparison_case1_olly3}
\end{figure}

\GTT{AX}レジスタが0であるのが見えます。実際、条件フラグはすべてゼロです。
( \olly は \FNSTSW 命令を\INS{FSTSW}としてディスアセンブルします。これは同義語です)

\clearpage
\TEST が実行されます。

\begin{figure}[H]
\centering
\myincludegraphics{patterns/12_FPU/3_comparison/x86/MSVC/olly1_4.png}
\caption{\olly: \TEST が実行される}
\label{fig:FPU_comparison_case1_olly4}
\end{figure}

\GTT{PF}フラグが1にセットされます。

実際、0にセットされるビットの数は0で0は偶数です。
\olly は\INS{JP}を\ac{JPE}としてディスアセンブルします。これは同義語です。
そして、実行されます。

\clearpage
\ac{JPE}が実行され、 \FLD は $b$ (3.4)の値を \ST{0}にロードします。

\begin{figure}[H]
\centering
\myincludegraphics{patterns/12_FPU/3_comparison/x86/MSVC/olly1_5.png}
\caption{\olly: 次の \FLD が実行される}
\label{fig:FPU_comparison_case1_olly5}
\end{figure}

関数が終了します。

\clearpage
\mysubparagraph{次の \olly の例: a=5.6 と b=-4}

例を \olly にロードしてみましょう。

\begin{figure}[H]
\centering
\myincludegraphics{patterns/12_FPU/3_comparison/x86/MSVC/olly2_1.png}
\caption{\olly: 最初の \FLD が実行される}
\label{fig:FPU_comparison_case2_olly1}
\end{figure}

関数の引数:$a=5.6$ と $b=-4$
$b$ (-4) はすでに \ST{0}にロードされています。
\FCOMP は実行されます。
\olly は次の \FCOMP の引数を表示します。スタックにあります。

\clearpage
\FCOMP が実行されます。

\begin{figure}[H]
\centering
\myincludegraphics{patterns/12_FPU/3_comparison/x86/MSVC/olly2_2.png}
\caption{\olly: \FCOMP が実行される}
\label{fig:FPU_comparison_case2_olly2}
\end{figure}

\ac{FPU}条件フラグの状態を見ることができます。 \Czero を除いてすべてゼロです。

\clearpage
\FNSTSW が実行されます。

\begin{figure}[H]
\centering
\myincludegraphics{patterns/12_FPU/3_comparison/x86/MSVC/olly2_3.png}
\caption{\olly: \FNSTSW が実行される}
\label{fig:FPU_comparison_case2_olly3}
\end{figure}

\GTT{AX}レジスタが\GTT{0x100}であるのが見えます。 \Czero フラグは8番目のビットです。

\clearpage
\TEST が実行されます。

\begin{figure}[H]
\centering
\myincludegraphics{patterns/12_FPU/3_comparison/x86/MSVC/olly2_4.png}
\caption{\olly: \TEST が実行される}
\label{fig:FPU_comparison_case2_olly4}
\end{figure}

\GTT{PF}フラグがクリアされます。
実際、

\GTT{0x100}にセットされるビットの数は1で、1は奇数です。
\ac{JPE}はスキップされます。

\clearpage
\ac{JPE}は実行されず、 \FLD は $a$ (5.6)の値を\ST{0}にロードします。

\begin{figure}[H]
\centering
\myincludegraphics{patterns/12_FPU/3_comparison/x86/MSVC/olly2_5.png}
\caption{\olly: 次の \FLD が実行される}
\label{fig:FPU_comparison_case2_olly5}
\end{figure}

関数が終了します。
