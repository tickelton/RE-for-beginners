\subsubsection{x86}

(MSVC 2010)で見てみましょう

\lstinputlisting[caption=MSVC 2010,style=customasmx86]{patterns/12_FPU/2_passing_floats/MSVC_JA.asm}

\myindex{x86!\Instructions!FLD}
\myindex{x86!\Instructions!FSTP}

\FLD および \FSTP は、データセグメントとFPUスタックとの間の変数を移動します。
\GTT{pow()} \footnote{標準的なC関数であり、与えられたべき乗(指数関数)}はスタックから両方の値をとり、
その結果を\ST{0}レジスタに返します。  \printf はローカルスタックから8バイトを取り出し、double型の変数として解釈します。

ちなみに、メモリ内の値はIEEE 754形式で格納され、pow()もこの形式で格納されているため、
値をメモリからスタックに移動するための一対の \MOV 命令を使用でき、変換は不要です。
これはARMのための次の例で行われます:\myref{FPU_passing_floats_ARM}
