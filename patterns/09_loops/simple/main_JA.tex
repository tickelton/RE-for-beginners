\subsection{単純な例}

% subsections
\EN{\input{patterns/04_scanf/3_checking_retval/x86_EN}}
\RU{\input{patterns/04_scanf/3_checking_retval/x86_RU}}
\PTBR{\input{patterns/04_scanf/3_checking_retval/x86_PTBR}}
\ITA{\input{patterns/04_scanf/3_checking_retval/x86_ITA}}
\FR{\subsection{x86}

\dots est compilée de manière très prédictive:

\lstinputlisting[caption=MSVC,style=customasmx86]{\CURPATH/11_1_msvc_FR.asm}

\myindex{x86!\Instructions!IDIV}

\IDIV divise le nombre 64-bit stocké dans la paire de registres \TT{EDX:EAX} par
la valeur dans \ECX.
Comme résultat, \EAX contiendra le \gls{quotient}, et \EDX --- le reste.
Le résultat de la fonction \ttf est renvoyé dans le registre \EAX, donc la valeur
n'est pas déplacée après la division, elle est déjà à la bonne place.

Puisque \IDIV utilise la valeur dans la paire de registres \TT{EDX:EAX}, l'instruction
\TT{CDQ} (avant \IDIV) étend la valeur dans \EAX en une valeur 64-bit, en tenant
compte du signe, tout comme \MOVSX le fait.

Si nous mettons l'optimisation (\Ox), nous obtenons:

\lstinputlisting[caption=MSVC \Optimizing,style=customasmx86]{\CURPATH/11_1_msvc_Ox.asm}

Ceci est la division par la multiplication. L'opération de multiplication est bien
plus rapide.
Et il possible d'utiliser cette astuce
\footnote{En savoir plus sur la division par la multiplication dans \InSqBrackets{\HenryWarren 10-3}}
pour produire du code effectivement équivalent et plus rapide.

Ceci est aussi appelé \q{strength reduction} dans les optimisations du compilateur.

GCC 4.4.1 génère presque le même code, même sans flag d'optimisation, tout comme
MSVC avec l'optimisation:

\lstinputlisting[caption=GCC 4.4.1 \NonOptimizing,style=customasmx86]{\CURPATH/11_2_gcc.asm}
}
\DE{\input{patterns/04_scanf/3_checking_retval/x86_DE}}
\JPN{\subsubsection{x86}

\myparagraph{MSVC}

コンパイルしてみましょう。

\lstinputlisting[caption=MSVC 2008,style=customasmx86]{patterns/13_arrays/1_simple/simple_msvc.asm}

\myindex{x86!\Instructions!SHL}

% TBT:
特別なことは何もなくて、2つのループだけです。1つめは配列に値を詰めるループで2つめは値を表示するループです。
\TT{shl ecx, 1}命令は \ECX の値を2倍するのに使用されます。詳細はこちら~\myref{SHR}

80バイトは4バイトの20要素分の配列用としてスタック上に確保されます。

\clearpage
\olly でこの例を試してみましょう。
\myindex{\olly}

配列がどのように埋まるのか見ていきます。

各要素は32ビットの \Tint 型で値はインデックスを2倍したものです。

\begin{figure}[H]
\centering
\myincludegraphics{patterns/13_arrays/1_simple/olly.png}
\caption{\olly: 要素を埋めた後}
\label{fig:array_simple_olly}
\end{figure}

この配列はスタックに位置しているので、20要素すべてを見ることができます。

\myparagraph{GCC}

GCC 4.4.1ではこのようになります。

\lstinputlisting[caption=GCC 4.4.1,style=customasmx86]{patterns/13_arrays/1_simple/simple_gcc.asm}

なお、変数 $a$ は\emph{int*}型です
(\Tint{} へのポインタ)---別の関数に配列へのポインタを渡すことができます。
しかし、もっと正確には、配列の最初の要素へのポインタが渡されます。
(要素の残りのアドレスは明確なやり方で計算されます)

もしこのポインタを\emph{a[idx]}としてインデックスするなら、\emph{idx}はポインタに加算されるだけで、
配置されている要素(計算されたポインタが示されている)がリターンされます。

面白い例:\emph{string}のような文字列は(1)文字の配列で\emph{const char[]}の型を持ちます。

インデックスもこのポインタに適用されます。

そしてこれが\TT{\q{string}[i]}のように書き込みが可能な理由です。これは \CCpp の正しい表現です!
}

\EN{\mysection{More about pointers}
\myindex{\CLanguageElements!\Pointers}
\label{label_pointers}

\epigraph{The way C handles pointers, for example, was a brilliant innovation;
it solved a lot of problems that we had before in data structuring and
made the programs look good afterwards.}{Donald Knuth, interview (1993)}

For those, who still have hard time understanding \CCpp pointers, here are more examples.
Some of them are weird and serves only demonstration purpose:
use them in production code only if you really know what you're doing.

% subsections:
\subsection{Weird loop optimization}

This is a simplest ever memcpy() function implementation:

\begin{lstlisting}[style=customc]
void memcpy (unsigned char* dst, unsigned char* src, size_t cnt)
{
	size_t i;
	for (i=0; i<cnt; i++)
		dst[i]=src[i];
};
\end{lstlisting}

At least MSVC 6.0 from the end of 1990s till MSVC 2013 can produce a really weird code (this listing is generated by MSVC 2013 x86):

\lstinputlisting[style=customasmx86]{advanced/500_loop_optimizations/1_1_EN.lst}

This is weird, because how humans work with two pointers? They store two addresses in two registers or two memory cells.
MSVC compiler in this case stores two pointers as one pointer (\emph{sliding dst} in \EAX) 
and difference between \emph{src} and \emph{dst} pointers (left unchanged over the span of loop body execution in \ESI).
\myindex{\CLanguageElements!ptrdiff\_t}
(By the way, this is a rare case when ptrdiff\_t data type can be used.)
When it needs to load a byte from \emph{src}, it loads it at \emph{diff + sliding dst} and stores byte
at just \emph{sliding dst}.

This has to be some optimization trick. But I've rewritten this function to:

\lstinputlisting[style=customasmx86]{advanced/500_loop_optimizations/1_2.lst}

\dots and it works as efficient as the \emph{optimized} version on my Intel Xeon E31220 @ 3.10GHz.
Maybe, this optimization was targeted some older x86 CPUs of 1990s era, since this trick is used at least by ancient MS VC 6.0?

Any idea?

\myindex{Hex-Rays}
Hex-Rays 2.2 have a hard time recognizing patterns like that (hopefully, temporary?):

\begin{lstlisting}[style=customc]
void __cdecl f1(char *dst, char *src, size_t size)
{
  size_t counter; // edx@1
  char *sliding_dst; // eax@2
  char tmp; // cl@3

  counter = size;
  if ( size )
  {
    sliding_dst = dst;
    do
    {
      tmp = (sliding_dst++)[src - dst];         // difference (src-dst) is calculated once, at the beginning
      *(sliding_dst - 1) = tmp;
      --counter;
    }
    while ( counter );
  }
}
\end{lstlisting}

Nevertheless, this optimization trick is often used by MSVC (not just in \ac{DIY} homebrew \emph{memcpy()} routines,
but in many loops which uses two or more arrays),
so it's worth for reverse engineers to keep it in mind.

% <!-- As of why writting occurred after <b>dst</b> incrementing? -->


\subsection{Another loop optimization}

If you process all elements of some array which happens to be located in global memory, compiler can optimize it.
For example, let's calculate a sum of all elements of array of 128 \emph{int}'s:

\begin{lstlisting}[style=customc]
#include <stdio.h>

int a[128];

int sum_of_a()
{
	int rt=0;
	
	for (int i=0; i<128; i++)
		rt=rt+a[i];

	return rt;
};

int main()
{
	// initialize
	for (int i=0; i<128; i++)
		a[i]=i;
	
	// calculate the sum
	printf ("%d\n", sum_of_a());
};
\end{lstlisting}

Optimizing GCC 5.3.1 (x86) can produce this (\IDA):

\lstinputlisting[style=customasmx86]{advanced/500_loop_optimizations/tmp1.lst}

What the heck is \TT{\_\_libc\_start\_main@@GLIBC\_2\_0} at \TT{0x080484C5}?
This is a label just after end of \TT{a[]} array.
The function can be rewritten like this:

\begin{lstlisting}[style=customc]
int sum_of_a_v2()
{
	int *tmp=a;
	int rt=0;
	
	do
	{
		rt=rt+(*tmp);
		tmp++;
	}
	while (tmp<(a+128));

	return rt;
};
\end{lstlisting}

First version has \emph{i} counter, and the address of each element of array is to be calculated at each iteration.
The second version is more optimized: the pointer to each element of array is always ready and is sliding 4 bytes forward at each iteration.
How to check if the loop is ended?
Just compare the pointer with the address just behind array's end, which is, in our case, is happens to be address of imported \TT{\_\_libc\_start\_main()} function from Glibc 2.0.
Sometimes code like this is confusing, and this is very popular optimizing trick, so that's why I made this example.

My second version is very close to what GCC did, and when I compile it, the code is almost the same as in first version, but two first instructions are swapped:

\lstinputlisting[style=customasmx86]{advanced/500_loop_optimizations/tmp2.lst}

Needless to say, this optimization is possible if the compiler can calculate address of the end of array during compilation time.
This happens if the array is global and it's size is fixed.

However, if the address of array is unknown during compilation, but size is fixed, address of the label just behind array's end can be calculated at the beginning of the loop.
% FIXME <!-- \ref{} to example -->


\subsection{Pointers abuse in Windows kernel}

The resource section of PE executable file in Windows OS is a section containing pictures, icons, strings, etc.
Early Windows versions allowed to address resources only by IDs, but then Microsoft added a way to address them using strings.

So then it would be possible to pass ID or string to 
\href{https://msdn.microsoft.com/en-us/library/windows/desktop/ms648042%28v=vs.85%29.aspx}{FindResource()} function.
Which is declared like this:

\myindex{win32!FindResource()}

\begin{lstlisting}[style=customc]
HRSRC WINAPI FindResource(
  _In_opt_ HMODULE hModule,
  _In_     LPCTSTR lpName,
  _In_     LPCTSTR lpType
);
\end{lstlisting}

\emph{lpName} and \emph{lpType} has \emph{char*} or \emph{wchar*} types, and when someone still wants to pass ID,
he/she have to use
\href{https://msdn.microsoft.com/en-us/library/windows/desktop/ms648029%28v=vs.85%29.aspx}{MAKEINTRESOURCE} macro, like this:

\myindex{win32!MAKEINTRESOURCE()}

\begin{lstlisting}[style=customc]
result = FindResource(..., MAKEINTRESOURCE(1234), ...);
\end{lstlisting}

It's interesting fact that MAKEINTRESOURCE is merely casting integer to pointer.
In MSVC 2013, in the file\\
\emph{Microsoft SDKs\textbackslash{}Windows\textbackslash{}v7.1A\textbackslash{}Include\textbackslash{}Ks.h} we can find this:

\begin{lstlisting}[style=customc]
...

#if (!defined( MAKEINTRESOURCE )) 
#define MAKEINTRESOURCE( res ) ((ULONG_PTR) (USHORT) res)
#endif

...
\end{lstlisting}

Sounds insane. Let's peek into ancient leaked Windows NT4 source code.
In \emph{private/windows/base/client/module.c} we can find \emph{FindResource()} source code:

\begin{lstlisting}[style=customc]
HRSRC
FindResourceA(
    HMODULE hModule,
    LPCSTR lpName,
    LPCSTR lpType
    )

...

{
    NTSTATUS Status;
    ULONG IdPath[ 3 ];
    PVOID p;

    IdPath[ 0 ] = 0;
    IdPath[ 1 ] = 0;
    try {
        if ((IdPath[ 0 ] = BaseDllMapResourceIdA( lpType )) == -1) {
            Status = STATUS_INVALID_PARAMETER;
            }
        else
        if ((IdPath[ 1 ] = BaseDllMapResourceIdA( lpName )) == -1) {
            Status = STATUS_INVALID_PARAMETER;
...
\end{lstlisting}

Let's proceed to \emph{BaseDllMapResourceIdA()} in the same source file:

\begin{lstlisting}[style=customc]
ULONG
BaseDllMapResourceIdA(
    LPCSTR lpId
    )
{
    NTSTATUS Status;
    ULONG Id;
    UNICODE_STRING UnicodeString;
    ANSI_STRING AnsiString;
    PWSTR s;

    try {
        if ((ULONG)lpId & LDR_RESOURCE_ID_NAME_MASK) {
            if (*lpId == '#') {
                Status = RtlCharToInteger( lpId+1, 10, &Id );
                if (!NT_SUCCESS( Status ) || Id & LDR_RESOURCE_ID_NAME_MASK) {
                    if (NT_SUCCESS( Status )) {
                        Status = STATUS_INVALID_PARAMETER;
                        }
                    BaseSetLastNTError( Status );
                    Id = (ULONG)-1;
                    }
                }
            else {
                RtlInitAnsiString( &AnsiString, lpId );
                Status = RtlAnsiStringToUnicodeString( &UnicodeString,
                                                       &AnsiString,
                                                       TRUE
                                                     );
                if (!NT_SUCCESS( Status )){
                    BaseSetLastNTError( Status );
                    Id = (ULONG)-1;
                    }
                else {
                    s = UnicodeString.Buffer;
                    while (*s != UNICODE_NULL) {
                        *s = RtlUpcaseUnicodeChar( *s );
                        s++;
                        }

                    Id = (ULONG)UnicodeString.Buffer;
                    }
                }
            }
        else {
            Id = (ULONG)lpId;
            }
        }
    except (EXCEPTION_EXECUTE_HANDLER) {
        BaseSetLastNTError( GetExceptionCode() );
        Id =  (ULONG)-1;
        }
    return Id;
}
\end{lstlisting}

\emph{lpId} is ANDed with \emph{LDR\_RESOURCE\_ID\_NAME\_MASK}. \\
Which we can find in \emph{public/sdk/inc/ntldr.h}:

\begin{lstlisting}[style=customc]
...

#define LDR_RESOURCE_ID_NAME_MASK 0xFFFF0000

...
\end{lstlisting}

So \emph{lpId} is ANDed with \emph{0xFFFF0000} and if some bits beyond lowest 16 bits are still present,
first half of function is executed (\emph{lpId} is treated as an address of string).
Otherwise---second half (\emph{lpId} is treated as 16-bit value).

Still, this code can be found in Windows 7 kernel32.dll file:

\lstinputlisting[style=customasmx86]{advanced/450_more_ptrs/tmp1.lst}

If value in input pointer is greater than 0x10000, jump to string processing is occurred.
Otherwise, input value of \emph{lpId} is returned as is.
\emph{0xFFFF0000} mask is not used here any more, because this is 64-bit code after all, but still, \emph{0xFFFFFFFFFFFF0000} could work here.

Attentive reader may ask, what if address of input string is lower than 0x10000?
This code relied on the fact that in Windows there are nothing on addresses below 0x10000, at least in Win32 realm.

Raymond Chen \href{https://blogs.msdn.microsoft.com/oldnewthing/20130925-00/?p=3123}{writes} about this:

\begin{framed}
\begin{quotation}
How does MAKE­INT­RESOURCE work? It just stashes the integer in the bottom 16 bits of a pointer, leaving the upper bits zero. This relies on the convention that the first 64KB of address space is never mapped to valid memory, a convention that is enforced starting in Windows 7.
\end{quotation}
\end{framed}

In short words, this is dirty hack and probably one should use it only if there is a real necessity.
Perhaps, \emph{FindResource()} function in past had \emph{SHORT} type for its arguments, and then Microsoft has added a way to pass strings there,
but older code must also be supported.

Now here is my short distilled example:

\begin{lstlisting}[style=customc]
#include <stdio.h>
#include <stdint.h>

void f(char* a)
{
	if (((uint64_t)a)>0x10000)
		printf ("Pointer to string has been passed: %s\n", a);
	else
		printf ("16-bit value has been passed: %d\n", (uint64_t)a);
};

int main()
{
	f("Hello!"); // pass string
	f((char*)1234); // pass 16-bit value
};
\end{lstlisting}

It works!

\subsubsection{Pointers abuse in Linux kernel}

As it has been noted in \href{https://news.ycombinator.com/item?id=11823647}{comments on Hacker News}, Linux kernel also has something like that.

For example, this function can return both error code and pointer:

\begin{lstlisting}[style=customc]
struct kernfs_node *kernfs_create_link(struct kernfs_node *parent,
				       const char *name,
				       struct kernfs_node *target)
{
	struct kernfs_node *kn;
	int error;

	kn = kernfs_new_node(parent, name, S_IFLNK|S_IRWXUGO, KERNFS_LINK);
	if (!kn)
		return ERR_PTR(-ENOMEM);

	if (kernfs_ns_enabled(parent))
		kn->ns = target->ns;
	kn->symlink.target_kn = target;
	kernfs_get(target);	/* ref owned by symlink */

	error = kernfs_add_one(kn);
	if (!error)
		return kn;

	kernfs_put(kn);
	return ERR_PTR(error);
}
\end{lstlisting}

( \url{https://github.com/torvalds/linux/blob/fceef393a538134f03b778c5d2519e670269342f/fs/kernfs/symlink.c#L25} )

\emph{ERR\_PTR} is a macro to cast integer to pointer:

\begin{lstlisting}[style=customc]
static inline void * __must_check ERR_PTR(long error)
{
	return (void *) error;
}
\end{lstlisting}

( \url{https://github.com/torvalds/linux/blob/61d0b5a4b2777dcf5daef245e212b3c1fa8091ca/tools/virtio/linux/err.h} )

This header file also has a macro helper to distinguish error code from pointer:

\begin{lstlisting}[style=customc]
#define IS_ERR_VALUE(x) unlikely((x) >= (unsigned long)-MAX_ERRNO)
\end{lstlisting}

This means, error codes are the ``pointers'' which are very close to -1 and, hopefully, there are nothing in kernel memory
on the addresses like 0xFFFFFFFFFFFFFFFF, 0xFFFFFFFFFFFFFFFE, 0xFFFFFFFFFFFFFFFD, etc.

Much more popular solution is to return \emph{NULL} in case of error and to pass error code via additional argument.
Linux kernel authors don't do that, but everyone who use these functions must always keep in mind that returning pointer
must always be checked with \emph{IS\_ERR\_VALUE} before dereferencing.

For example:

\begin{lstlisting}[style=customc]
	fman->cam_offset = fman_muram_alloc(fman->muram, fman->cam_size);
	if (IS_ERR_VALUE(fman->cam_offset)) {
		dev_err(fman->dev, "%s: MURAM alloc for DMA CAM failed\n",
			__func__);
		return -ENOMEM;
	}
\end{lstlisting}

( \url{https://github.com/torvalds/linux/blob/aa00edc1287a693eadc7bc67a3d73555d969b35d/drivers/net/ethernet/freescale/fman/fman.c#L826} )

\subsubsection{Pointers abuse in UNIX userland}

\myindex{UNIX!mmap()}
mmap() function returns -1 in case of error (or \TT{MAP\_FAILED}, which equals to -1).
Some people say, mmap() can map a memory at zeroth address in rare situations, so it can't use 0 or NULL as error code.


\input{advanced/450_more_ptrs/4_EN}
\input{advanced/450_more_ptrs/5_EN}
\input{advanced/450_more_ptrs/6_EN}
\input{advanced/450_more_ptrs/7_EN}
\subsection{Oracle RDBMS and a simple garbage collector for C/C++}

There was a time, when the author of these lines tried to learn more about Oracle RDBMS, searching for vulnerabilities, etc.
This is a huge piece of software, and a typical function can take very large nested objects as arguments.
And I wanted to dump these objects, as trees (or graphs).

Also, I tracked all memory allocations/deallocations by intercepting memory allocating/deallocating functions.
And when a function to be intercepted getting a pointer to a block in memory, I search for the block in a list of blocks allocated.
I'm getting its size + short name of block
(this is like "tagging" in Windows OS kernel\footnote{Read more about comments in allocated blocks: \CNotes{} \url{http://yurichev.com/C-book.html}}).

Given a block, I can scan it for 32-bit words (on 32-bit OS) or for 64-bit words (on 64-bit OS).
Each word can be a pointer to another block.
And if it is so (I find this another block in my records), I can process it recursively.

\myindex{GraphViz}
And then, using GraphViz, I could render such a diagrams:

\begin{figure}[H]
\centering
\includegraphics[scale=0.55]{advanced/450_more_ptrs/oracle2_crop.png}
\end{figure}

Bigger pictures:
\href{https://raw.githubusercontent.com/DennisYurichev/RE-for-beginners/master/advanced/450_more_ptrs/oracle1.png}{1},
\href{https://raw.githubusercontent.com/DennisYurichev/RE-for-beginners/master/advanced/450_more_ptrs/oracle2.png}{2}.

This is quite impressive, given the fact that I had no information about data types of all these structures.
But I could get some information from it.

\subsubsection{Now the garbage collector for C/C++: Boehm GC}

\myindex{Garbage collector}
If you use a block allocated in memory, its address has to be present somewhere, as a pointer in some structure/array in another allocated block,
or in globally allocated structure, or in local variable in stack.
If there are no pointer to a block, you can call it "orphan", and it will be a reason of memory leak.

And this is what \ac{GC} does.
It scans all blocks (because it keep tabs on all blocks allocated) for pointers.
It's important to understand, that it has no idea of data types of all these structure fields in blocks---this is important, \ac{GC} has no information about types.
It just scans blocks for 32-bit of 64-bit words and see, if they could be a pointers to another block(s).
It also scans stack.
It treats allocated blocks and stack as arrays of words, some of which may be pointers.
And if it found a block allocated, which is "orphaned", i.e., there are no pointer(s) to it from another block(s) or stack, this block considered unneeded, to be freed.
Scanning process takes time, and this is what for \ac{GC}s are criticized.

\myindex{Boehm garbage collector}
Also, \ac{GC} like Boehm GC\footnote{\url{https://www.hboehm.info/gc/}} (for pure C) has function like \verb|GC_malloc_atomic()|---using it, you declare that the block allocated
using this function will never contain any pointer(s) to other block(s).
Maybe this could be a text string, or other type of data.
(Indeed, \verb|GC_strdup()| calls \verb|GC_malloc_atomic()|.)
\ac{GC} will not scan it.

% Even more: if \ac{GC}'s memory allocator thinks it can find a better place for a block, it can \emph{move} it to another place, and then fix (rewrite) all addresses,
% pointing to it, in all other blocks and in stack.



}
\DE{\newcommand{\GlobalVarsSectionName}{Globale Variablen}
\subsection{\GlobalVarsSectionName}
\myindex{\GlobalVarsSectionName}
\label{scanf_global_variable}

Was passiert, wenn die \TT{x}-Variable aus dem letzten Beispiel nicht lokal sondern global ist?
In dem Fall wäre sie von jeder Stelle aus zugreifbar, nicht nur aus dem Funktions-Rumpf.
Globale Variablen gelten als \gls{anti-pattern}, aber für den Lerneffekt können wir folgendes tun:

\lstinputlisting[style=customc]{patterns/04_scanf/2_global/ex2_DE.c}

\input{patterns/04_scanf/2_global/ex2_global_vars_x86_DE}
\input{patterns/04_scanf/2_global/ex2_global_vars_ARM_DE}

\EN{\input{patterns/04_scanf/2_global/MIPS/main_EN}}
\RU{\input{patterns/04_scanf/2_global/MIPS/main_RU}}
\DE{\input{patterns/04_scanf/2_global/MIPS/main_DE}}
\IT{\input{patterns/04_scanf/2_global/MIPS/main_IT}}
\FR{\input{patterns/04_scanf/2_global/MIPS/main_FR}}
\JA{\input{patterns/04_scanf/2_global/MIPS/main_JA}}


}
\RU{\subsection{Функции проверки пароля в SAP 6.0}

\myindex{SAP}
Когда автор этой книги в очередной раз вернулся к своему SAP 6.0 IDES заинсталлированному в виртуальной машине VMware, он обнаружил что забыл пароль, впрочем, затем он вспомнил его, но теперь получаем такую ошибку:
 
\emph{<<Password logon no longer possible - too many failed attempts>>}, 
потому что были потрачены все попытки на то, чтобы вспомнить его.

\myindex{Windows!PDB}
Первая очень хорошая новость состоит в том, что с SAP поставляется полный \gls{PDB}-файл \emph{disp+work.pdb}, он содержит все: имена функций, структуры, типы, локальные переменные, имена аргументов, и~т.д. Какой щедрый подарок!

Существует утилита TYPEINFODUMP\footnote{\url{http://go.yurichev.com/17038}} для дампа содержимого \gls{PDB}-файлов во что-то более читаемое и grep-абельное.

Вот пример её работы: информация о функции + её аргументах + её локальных переменных:

\lstinputlisting{examples/SAP/pw/1.txt}

А вот пример дампа структуры:

\lstinputlisting{examples/SAP/pw/2.txt}

Вау!

Вторая хорошая новость: \emph{отладочные} вызовы, коих здесь очень много, очень полезны. 

Здесь вы можете увидеть глобальную переменную \emph{ct\_level}\footnote{Еще об уровне трассировки: \url{http://go.yurichev.com/17039}}, отражающую уровень трассировки.

В \emph{disp+work.exe} очень много таких отладочных вставок:

\lstinputlisting[style=customasmx86]{examples/SAP/pw/3.asm}

Если текущий уровень трассировки выше или равен заданному в этом коде порогу, 
отладочное сообщение будет записано в лог-файл вроде \emph{dev\_w0}, \emph{dev\_disp} 
и прочие файлы \emph{dev*}.

\myindex{\GrepUsage}
Попробуем grep-ать файл недавно полученный при помощи утилиты TYPEINFODUMP:

\begin{lstlisting}
cat "disp+work.pdb.d" | grep FUNCTION | grep -i password
\end{lstlisting}

Получаем:

\lstinputlisting{examples/SAP/pw/4.txt}

Попробуем так же искать отладочные сообщения содержащие слова \emph{<<password>>} и \emph{<<locked>>}.
Одна из таких это строка \emph{<<user was locked by subsequently failed password logon attempts>>} на которую есть ссылка в \\
функции \emph{password\_attempt\_limit\_exceeded()}.

Другие строки, которые эта найденная функция может писать в лог-файл это: 
\emph{<<password logon attempt will be rejected immediately (preventing dictionary attacks)>>}, \emph{<<failed-logon lock: expired (but not removed due to 'read-only' operation)>>}, \emph{<<failed-logon lock: expired => removed>>}.

Немного поэкспериментировав с этой функцией, мы быстро понимаем, что проблема именно в ней.
Она вызывается из функции \emph{chckpass()} --- одна из функций проверяющих пароль.

В начале, давайте убедимся, что мы на верном пути:

Запускаем \tracer:
\myindex{tracer}

\begin{lstlisting}
tracer64.exe -a:disp+work.exe bpf=disp+work.exe!chckpass,args:3,unicode
\end{lstlisting}

\begin{lstlisting}
PID=2236|TID=2248|(0) disp+work.exe!chckpass (0x202c770, L"Brewered1                               ", 0x41) (called from 0x1402f1060 (disp+work.exe!usrexist+0x3c0))
PID=2236|TID=2248|(0) disp+work.exe!chckpass -> 0x35
\end{lstlisting}

Функции вызываются так: \emph{syssigni()} -> \emph{DyISigni()} -> \emph{dychkusr()} -> \emph{usrexist()} -> \emph{chckpass()}.

Число 0x35 возвращается из \emph{chckpass()} в этом месте:

\lstinputlisting[style=customasmx86]{examples/SAP/pw/5.asm}

Отлично, давайте проверим:

\begin{lstlisting}
tracer64.exe -a:disp+work.exe bpf=disp+work.exe!password_attempt_limit_exceeded,args:4,unicode,rt:0
\end{lstlisting}

\begin{lstlisting}
PID=2744|TID=360|(0) disp+work.exe!password_attempt_limit_exceeded (0x202c770, 0, 0x257758, 0) (called from 0x1402ed58b (disp+work.exe!chckpass+0xeb))
PID=2744|TID=360|(0) disp+work.exe!password_attempt_limit_exceeded -> 1
PID=2744|TID=360|We modify return value (EAX/RAX) of this function to 0
PID=2744|TID=360|(0) disp+work.exe!password_attempt_limit_exceeded (0x202c770, 0, 0, 0) (called from 0x1402e9794 (disp+work.exe!chngpass+0xe4))
PID=2744|TID=360|(0) disp+work.exe!password_attempt_limit_exceeded -> 1
PID=2744|TID=360|We modify return value (EAX/RAX) of this function to 0
\end{lstlisting}

Великолепно! Теперь мы можем успешно залогиниться.

Кстати, мы можем сделать вид что вообще забыли пароль, заставляя \emph{chckpass()} всегда возвращать ноль, и этого достаточно для отключения проверки пароля:

\begin{lstlisting}
tracer64.exe -a:disp+work.exe bpf=disp+work.exe!chckpass,args:3,unicode,rt:0
\end{lstlisting}

\begin{lstlisting}
PID=2744|TID=360|(0) disp+work.exe!chckpass (0x202c770, L"bogus                                   ", 0x41) (called from 0x1402f1060 (disp+work.exe!usrexist+0x3c0))
PID=2744|TID=360|(0) disp+work.exe!chckpass -> 0x35
PID=2744|TID=360|We modify return value (EAX/RAX) of this function to 0
\end{lstlisting}

Что еще можно сказать, бегло анализируя функцию \\
\emph{password\_attempt\_limit\_exceeded()}, это то, что в начале
можно увидеть следующий вызов:

\lstinputlisting[style=customasmx86]{examples/SAP/pw/6.asm}

Очевидно, функция \emph{sapgparam()} используется чтобы узнать значение какой-либо переменной конфигурации. Эта функция может вызываться из 1768 разных мест.

Вероятно, при помощи этой информации, мы можем легко находить те места кода, на которые влияют определенные переменные конфигурации.

Замечательно! Имена функций очень понятны, куда понятнее чем в \oracle.
 
\myindex{\Cpp}
По всей видимости, процесс \emph{disp+work} весь написан на \Cpp. Должно быть, он был переписан не так давно?

}
\IT{\mysection{Ritornare valori}
\label{ret_val_func}

Un'altra funzione semplice è quella che ritorna un valore costante:

\lstinputlisting[caption=\EN{\CCpp Code},style=customc]{patterns/011_ret/1.c}

Compiliamola.

\subsection{x86}

Questo è quello che i compilatori GCC e MSVC producono (con ottimizzazione) per x86:

\lstinputlisting[caption=\Optimizing GCC/MSVC (\assemblyOutput),style=customasmx86]{patterns/011_ret/1.s}

\myindex{x86!\Instructions!RET}
Ci sono solo due istruzioni: la prima inserisce il valore 123 nel registro \EAX,
che per convenzione viene utilizzato per memorizzare i valori di ritorno,
e la seconda è \RET, che ritorna l'esecuzione al \gls{caller}.

La funzione chiamante troverà quindi il valore di ritorno nel registro \EAX.

\subsection{ARM}

Ci sono alcune differenze nella piattaforma ARM:

\lstinputlisting[caption=\OptimizingKeilVI (\ARMMode) ASM Output,style=customasmARM]{patterns/011_ret/1_Keil_ARM_O3.s}

ARM utilizza il registro \Reg{0} per ritornare i risultati delle funzioni, quindi 123 viene copiato in \Reg{0}.

\myindex{ARM!\Instructions!MOV}
\myindex{x86!\Instructions!MOV}
Occorre notare che \MOV è un nome di funzione fuorviante sia nella \ac{ISA} x86 che ARM.

I dati non vengono infatti \emph{spostati}, ma \emph{copiati}.

\subsection{MIPS}

\label{MIPS_leaf_function_ex1}

L'output in assembly di GCC assembly qua sotto chiama i registri per numero:

\lstinputlisting[caption=\Optimizing GCC 4.4.5 (\assemblyOutput),style=customasmMIPS]{patterns/011_ret/MIPS.s}

\dots mentre \IDA utilizza gli pseudonimi:

\lstinputlisting[caption=\Optimizing GCC 4.4.5 (IDA),style=customasmMIPS]{patterns/011_ret/MIPS_IDA.lst}

Il registro \$2 (o \$V0) viene utilizzato per memorizzare il valore di ritorno della funzione.
\myindex{MIPS!\Pseudoinstructions!LI}
\INS{LI} sta per ``Load Immediate'' ed è l'equivalente MIPS di \MOV.

\myindex{MIPS!\Instructions!J}
L'altra istruzione è il salto (J or JR) che ritorna il flusso di esecuzione al \gls{caller}.

\myindex{MIPS!Branch delay slot}
Potresti domandarti perchè le posizioni delle istruzioni Load Immediate (LI) ed il jump (J or JR) siano invertite. Questo è dovuto ad una funzionalità di \ac{RISC} chiamata``branch delay slot''.

Il motivo per cui accade è dovuto ad un problema nell'architettura di alcune \ac{ISA} RISC e non è importante per i nostri
scopi---dobbiamo semplicemente tenere a mente che in MIPS, l'istruzione che segue un jump o un'istruzione condizionale
viene eseguita \emph{prima} del salto/ramificazione stessi.

Come conseguenza, le istruzioni di ramificazione vengono sempre scambiate di posto con l'istruzione immediatamente precedente.

In pratica, le funzioni che ritornano semplicemente 1 (\emph{true}) o 0 (\emph{false}) sono molto frequenti.

Le più piccole utility UNIX in assoluto, \emph{/bin/true} e \emph{/bin/false} ritornano 0 ed 1 rispettivamente, come codice di uscita.
(Zero come codice di uscita generalmente indica successo, valori diversi da zero indicano errori.)
}
\PTBR{\input{patterns/04_scanf/2_global/main_PTBR}}
\PL{\input{patterns/04_scanf/2_global/main_PL}}
\FR{\mysection{Unions}

Les \emph{unions} en \CCpp sont utilisées principalement pour interpréter une variable
(ou un bloc de mémoire) d'un type de données comme une variable d'un autre type de données.

% sections
\EN{\input{patterns/04_scanf/2_global/main_EN}}
\DE{\input{patterns/04_scanf/2_global/main_DE}}
\RU{\input{patterns/04_scanf/2_global/main_RU}}
\IT{\input{patterns/04_scanf/2_global/main_IT}}
\PTBR{\input{patterns/04_scanf/2_global/main_PTBR}}
\PL{\input{patterns/04_scanf/2_global/main_PL}}
\FR{\input{patterns/04_scanf/2_global/main_FR}}
\JA{\input{patterns/04_scanf/2_global/main_JA}}

\EN{\input{patterns/04_scanf/2_global/main_EN}}
\DE{\input{patterns/04_scanf/2_global/main_DE}}
\RU{\input{patterns/04_scanf/2_global/main_RU}}
\IT{\input{patterns/04_scanf/2_global/main_IT}}
\PTBR{\input{patterns/04_scanf/2_global/main_PTBR}}
\PL{\input{patterns/04_scanf/2_global/main_PL}}
\FR{\input{patterns/04_scanf/2_global/main_FR}}
\JA{\input{patterns/04_scanf/2_global/main_JA}}

\input{patterns/17_unions/FSCALE_FR}

\subsection{\FRph{}}

Un autre algorithme connu où un \Tfloat est interprété comme un entier est celui
de calcul rapide de racine carrée.

\begin{lstlisting}[caption=Le code source provient de Wikipedia: \url{http://go.yurichev.com/17364},style=customc]
/* Assumes that float is in the IEEE 754 single precision floating point format
 * and that int is 32 bits. */
float sqrt_approx(float z)
{
    int val_int = *(int*)&z; /* Same bits, but as an int */
    /*
     * To justify the following code, prove that
     *
     * ((((val_int / 2^m) - b) / 2) + b) * 2^m = ((val_int - 2^m) / 2) + ((b + 1) / 2) * 2^m)
     *
     * where
     *
     * b = exponent bias
     * m = number of mantissa bits
     *
     * .
     */
 
    val_int -= 1 << 23; /* Subtract 2^m. */
    val_int >>= 1; /* Divide by 2. */
    val_int += 1 << 29; /* Add ((b + 1) / 2) * 2^m. */
 
    return *(float*)&val_int; /* Interpret again as float */
}
\end{lstlisting}

À titre d'exercice, vous pouvez essayez de compiler cette fonction et de comprendre
comme elle fonctionne.\\
\\
C'est un algorithme connu de calcul rapide de $\frac{1}{\sqrt{x}}$.
\myindex{Quake III Arena}
L'algorithme devînt connu, supposément, car il a été utilisé dans Quake III Arena.

La description de l'algorithme peut être trouvée sur Wikipédia: \url{http://go.yurichev.com/17360}.

}
\JA{\subsection{擬似乱数生成器の例}
\label{FPU_PRNG}

0と1の間の浮動小数点の乱数が必要な場合、最も簡単なのはメルセンヌツイスターのような
\ac{PRNG}を使うことです。
ランダムな符号なし32ビット値を生成します(つまり、ランダム32ビットを生成します)。
この値をfloatに変換し、
\GTT{RAND\_MAX}(ここでは\GTT{0xFFFFFFFF})で割ります。我々は0..1の間で値を取得します。

しかし知ってのとおり、除算は遅いです。
また、できるだけ少ないFPU演算で実行したいと考えています。
私たちは除算を取り除くことができるでしょうか?

\myindex{IEEE 754}

浮動小数点数が符号ビット、仮数ビット、指数ビットからなるものを思い出してみましょう。
ランダムな浮動小数点数を得るには、すべての仮数ビットにランダムなビットを格納するだけです。

指数部はゼロではありません(浮動小数点はこの場合非正規化されています)ので、
指数部に0b01111111
を格納しています。指数部が1であることを意味します。
次に、仮数部をランダムビットで埋め、符号ビットを0に設定する(正の数)と出来上がり。
生成される数は1と2の間にあるので、1を減算する必要があります。

\newcommand{\URLXOR}{\url{http://go.yurichev.com/17308}}

私の例では、非常に単純な線形合同乱数ジェネレータが使用され、
\footnote{アイデアは以下から取りました: \URLXOR} これは32ビットの数値を生成します。 
\ac{PRNG}はUNIXのタイムスタンプ形式で現在の時刻で初期化されます。

ここでは \Tfloat 型を\emph{union}として表します。これは、メモリの種類を
異なる型として解釈できる \CCpp 構造です。
私たちの場合、\emph{union}型の変数を作成し、
それを \Tfloat  または \emph{uint32\_t} のようにアクセスすることができます。
それはまさに汚いハックだと言えるでしょう。

% WTF?

整数\ac{PRNG}コードは、すでに検討しているものと同じです:\myref{LCG_simple}
このコードはコンパイルされた形式では省略されています。

\lstinputlisting[style=customc]{patterns/17_unions/FPU_PRNG/FPU_PRNG_JA.cpp}

\subsubsection{x86}

\lstinputlisting[caption=\Optimizing MSVC 2010,style=customasmx86]{patterns/17_unions/FPU_PRNG/MSVC2010_Ox_Ob0_JA.asm}

この例はC++としてコンパイルされており、これはC++での名前の変換であるため、ここでは関数名が非常に奇妙です。
これについては後で説明します:\myref{namemangling}
これをMSVC 2012でコンパイルすると、FPU用のSIMD命令が使用されます。詳細については、こちらを参照してください:\myref{FPU_PRNG_SIMD}

\iffalse
A BUG HERE
\subsubsection{MIPS}

\lstinputlisting[caption=\Optimizing GCC 4.4.5,style=customasmMIPS]{patterns/17_unions/FPU_PRNG/MIPS_O3_IDA_JA.lst}

いくつかの奇妙な理由のために追加された無駄な\INS{LUI}命令もあります。 
このアーティファクトを以前検討しました:\myref{MIPS_FPU_LUI}
\fi

\subsubsection{ARM (\ARMMode)}

\lstinputlisting[caption=\Optimizing GCC 4.6.3 (IDA),style=customasmARM]{patterns/17_unions/FPU_PRNG/raspberry_GCC_O3_IDA_JA.lst}

\myindex{objdump}
\myindex{binutils}
\myindex{IDA}

また、objdumpにダンプを作成し、FPU命令の名前が \IDA とは異なることを確認します。 
見たところ、IDAとbinutilsの開発者は異なるマニュアルを使ったのでしょうか? 
おそらく、両方の命令の変種を知っておくとよいでしょう。

\lstinputlisting[caption=\Optimizing GCC 4.6.3 (objdump),style=customasmARM]{patterns/17_unions/FPU_PRNG/raspberry_GCC_O3_objdump.lst}

\TT{float\_rand()}の0x5cと \main の0x38の命令は、(疑似)乱数ノイズです。
}

\EN{\input{patterns/07_jcc/simple/MIPS_EN}}
\RU{\input{patterns/07_jcc/simple/MIPS_RU}}
\DE{\input{patterns/07_jcc/simple/MIPS_DE}}
\FR{\subsection{MIPS}

Pour une raison quelconque, GCC 4.4.5 avec optimisation génère seulement une instruction
de division:

\lstinputlisting[caption=\Optimizing GCC 4.4.5 (IDA),style=customasmMIPS]{\CURPATH/MIPS_O3_IDA_FR.lst}

\myindex{MIPS!\Instructions!BREAK}
Ici, nous voyons une nouvelle instruction: BREAK. Elle lève simplement une exception.

Dans ce cas, une exception est levée si le diviseur est zéro (il n'est pas possible
de diviser par zéro dans les mathématiques conventionnelles).

Mais GCC n'a probablement pas fait correctement le travail d'optimisation et n'a
pas vu que \$V0 ne vaut jamais zéro.

Donc le test est laissé ici.
Donc, si \$V0 est zéro, BREAK est exécuté, signalant l'exception à l'\ac{OS}.

\myindex{MIPS!\Instructions!MFLO}
Autrement, MFLO s'exécute, qui prend le résultat de la division depuis le registre
LO et le copie dans \$V0.

\myindex{MIPS!\Registers!LO}
\myindex{MIPS!\Registers!HI}
À propos, comme on devrait le savoir, l'instruction MUL laisse les 32bits hauts du
résultat dans le registre HI et les 32 bits bas dans le registre LO.

DIV laisse le résultat dans le registre LO, et le reste dans le registre HI.

\myindex{MIPS!\Instructions!MFHI}
Si nous modifions la déclaration en \q{a \% 9},
l'instruction MFHI est utilisée au lieu de MFLO.
}
\IT{\input{patterns/07_jcc/simple/MIPS_IT}}
\JA{\subsubsection{MIPS}
\label{MIPS_structure_big_endian}

\lstinputlisting[caption=\Optimizing GCC 4.4.5 (IDA),numbers=left,style=customasmMIPS]{patterns/15_structs/4_packing/MIPS_O3_IDA_JA.lst}

構造体フィールドはレジスタ \$A0..\$A3 に入ってから \printf のために \$A1..\$A3 に再整理され、
4番目のフィールド(\$A3 から)は\INS{SW}を使ってローカルスタックを経由して渡されます。

しかし、2つのSRA(\q{Shift Word Right Arithmetic})命令があり、これは \Tchar フィールドを準備します。 
なぜでしょうか?

MIPSはデフォルトではビッグエンディアンアーキテクチャです\myref{sec:endianness}。私たちが動かすDebian Linuxもビッグエンディアンです。

したがって、バイト変数が32ビット構造のスロットに格納されるとき、それらは高位31~24ビットを占有します。

また、\Tchar 変数を32ビット値に拡張する必要がある場合は、それを24ビット右にシフトする必要があります。

\Tchar は符号付き型なので、ここでは論理シフトの代わりに算術シフトが使用されます。
}


\subsubsection{もう一つ}

生成されたコードでは、 
$i$ を初期化した後、iの条件が最初にチェックされ、
ループ本体が実行された後にのみ、ループの本体が実行されないことがわかります。 
それは正しいです。

ループの条件が最初に満たされない場合、
ループの本体は実行されてはならないからです。
これは次の場合に可能です:

\lstinputlisting[style=customc]{patterns/09_loops/simple/loops_3_JA.c}

\emph{total\_entries\_to\_process}が0の場合、ループの本体はまったく実行されてはなりません。

これは、実行前に条件がチェックされている理由です。

しかし、最適化されたコンパイラは、ここで説明した状況が不可能であることが確かな場合
(例えばKeil、Xcode(LLVM)、MSVCなどのコンパイラを最適化モードで使用する場合など)、
条件チェックとループ本体をスワップできます。
