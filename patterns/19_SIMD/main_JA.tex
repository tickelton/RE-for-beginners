\mysection{SIMD}

\label{SIMD_x86}
\ac{SIMD} は頭字語です:\emph{Single Instruction, Multiple Data}

名前の通り、複数のデータを1つの命令で処理します。

\ac{FPU}と同様に、\ac{CPU}サブシステムはx86内では独立したプロセッサのように見えます。

\myindex{x86!MMX}

SIMDはx86でMMXとして始まりました。 8つの新しい64ビットレジスタが登場しました:MM0-MM7

各MMXレジスタは、2つの32ビット値、4つの16ビット値、または8バイトを保持できます。
たとえば、MMXレジスタに2つの値を追加することで、8つの8ビット値(バイト)を同時に追加することができます。

簡単な例として、画像を2次元配列として表現するグラフィックエディタがあります。
ユーザが画像の明るさを変更すると、エディタは各ピクセル値に係数を加減する必要があります。
簡潔にするために、画像がグレースケールで各ピクセルが1つの8ビットバイトで定義されているとしたら、
8ピクセルの明るさを同時に変更することが可能です。

ところで、これが\emph{飽和}命令がSIMDに存在する理由です。

ユーザがグラフィックエディタで明るさを変更するとき、オーバーフローとアンダーフローは望ましくないので、
最大値に達すると何も加算しないという追加命令がSIMDにあります。

MMXが登場したとき、これらのレジスタは実際にはFPUのレジスタにありました。 
FPUまたはMMXを同時に使用することは可能でした。 Intelはトランジスタを節約したと思うかもしれませんが、
実際にはそのような共生の理由はより単純なものでした。追加のCPUレジスタを意識しない\ac{OS}は
コンテキストスイッチでそれらを保存せず、FPUレジスタを保存します。
したがって、MMX機能を利用したMMX対応CPU + 古い\ac{OS} + プロセスは依然として機能します。

\myindex{x86!SSE}
\myindex{x86!SSE2}
SSEはSIMDレジスタを128ビットに拡張したもので、現在はFPUとは別のものです。

\myindex{x86!AVX}
AVXは256ビットにした他の拡張です。

実用的な用途はどうでしょうか。

もちろん、これはメモリコピールーチン(\TT{memcpy})、メモリ比較(\TT{memcmp})などです。

\myindex{DES}

もう1つの例:DES暗号化アルゴリズムは64ビットブロックと56ビットキーを受け取り、ブロックを暗号化して64ビットの結果を生成します。 
DESアルゴリズムは、ワイヤおよびAND/OR/NOTゲートを有する非常に大きな電子回路と見なすことができます。

\label{bitslicedes}
\newcommand{\URLBS}{\url{http://go.yurichev.com/17329}}

Bitslice DES\footnote{\URLBS}は、ブロックとキーのグループを同時に処理するというアイデアです。
たとえば、x86の\emph{符号なし整数}型の変数は最大32ビットを保持できるため、
64個+56個の\emph{符号なし整数}型の変数を使用して、32個のブロックキーペアの中間結果を同時に格納できます。

\myindex{\oracle}
There is an utility to brute-force \oracle passwords/hashes (ones based on DES),
using slightly modified bitslice DES algorithm for SSE2 and AVX---now it is possible to encrypt 128 
or 256 block-keys pairs simultaneously.
SSE2およびAVX用にわずかに修正されたbitslice DESアルゴリズムを使用して、
Oracle RDBMSのパスワード/ハッシュ(DESに基づくもの)をブルートフォースするユーティリティがあります。
128または256のブロックキーペアを暗号化することが可能です。

\url{http://go.yurichev.com/17313}

% sections
\subsection{ベクトル化}

\newcommand{\URLVEC}{\href{http://go.yurichev.com/17080}{Wikipedia: vectorization}}

ベクトル化\footnote{\URLVEC}は、たとえば、入力用に2つの配列を取り、1つの配列を生成するループがある場合です。
ループ本体は入力配列から値を受け取り、何かを実行して結果を出力配列に入れます。
%It is important that there is only a single operation applied to each element.
ベクトル化は、いくつかの要素を同時に処理することです。

ベクトル化はそれほど新鮮なテクノロジではありません。この教科書の作者は、少なくとも
1988年のCray Y-MPスーパーコンピュータラインで、その\q{ライト}バージョンのCray Y-MP EL 179を使ったことを見ました。
\footnote{リモートから。それはスーパーコンピュータの博物館に設置されています: \url{http://go.yurichev.com/17081}}.

% FIXME! add assembly listing!
例えば:

\begin{lstlisting}[style=customc]
for (i = 0; i < 1024; i++)
{
    C[i] = A[i]*B[i];
}
\end{lstlisting}

このコード片は、AとBから要素を取り出し、それらを乗算して結果をCに保存します。

\myindex{x86!\Instructions!PLMULLD}
\myindex{x86!\Instructions!PLMULHW}
\newcommand{\PMULLD}{\emph{PMULLD} (\emph{Multiply Packed Signed Dword Integers and Store Low Result})}
\newcommand{\PMULHW}{\TT{PMULHW} (\emph{Multiply Packed Signed Integers and Store High Result})}

各配列要素が32ビット \Tint の場合、Aから128ビットのXMMレジスタへ、Bから別のXMMレジスタへ、
\PMULLD{} および \PMULHW{}を実行することで、一度に4つの64ビット\glspl{product}を取得できます。

したがって、ループ本体の実行数は1024ではなく$1024/4$となり、4倍少なくなり、もちろん高速になります。

\newcommand{\URLINTELVEC}{\href{http://go.yurichev.com/17082}{Excerpt: Effective Automatic Vectorization}}

\subsubsection{加算の例}

\myindex{Intel C++}

Intel C++\footnote{インテルC ++自動ベクトル化についての詳細: \URLINTELVEC}のように、
単純な場合には自動的にベクトル化を実行できるコンパイラーもあります。

これが小さな機能です。

\begin{lstlisting}[style=customc]
int f (int sz, int *ar1, int *ar2, int *ar3)
{
	for (int i=0; i<sz; i++)
		ar3[i]=ar1[i]+ar2[i];

	return 0;
};
\end{lstlisting}

\myparagraph{Intel C++}

それをIntel C++ 11.1.051 win32でコンパイルしましょう。

\begin{verbatim}
icl intel.cpp /QaxSSE2 /Faintel.asm /Ox
\end{verbatim}

(\IDA で)次の結果を得ました。

\lstinputlisting[style=customasmx86]{patterns/19_SIMD/18_1_JA.asm}

SSE2関連の命令は以下のとおりです。
\myindex{x86!\Instructions!MOVDQA}
\myindex{x86!\Instructions!MOVDQU}
\myindex{x86!\Instructions!PADDD}
\begin{itemize}
\item
\MOVDQU (\emph{Move Unaligned Double Quadword})---メモリから16バイトをXMMレジスタにロードします

\item
\PADDD (\emph{Add Packed Integers})---4対の32ビット数を加算し、その結果を最初のオペランドに残します。
ちなみに、オーバーフローが発生しても例外は発生せず、フラグも設定されません。
結果の下位32ビットだけが格納されます。 
\PADDD のオペランドの1つがメモリ内の値のアドレスである場合、
そのアドレスは16バイト境界に揃えられている必要があります。
整列されていない場合は、例外が発生します。
\footnote{データアラインメントについての詳細は: \URLWPDA}.

\item
\MOVDQA (\emph{Move Aligned Double Quadword})
はMOVDQUと同じですが、メモリ内の値のアドレスを16ビット境界に揃える必要があります。
整列されていないと、例外が発生します。 
\MOVDQA は \MOVDQU よりも高速に動作しますが、前述のものが必要です。

\end{itemize}

そのため、これらのSSE2命令は、作業するペアが4つ以上あり、
ポインタ\TT{ar3}が16バイト境界に整列している場合にのみ実行されます。

また、\TT{ar2}が16バイト境界にも揃えられている場合は、
次のコードが実行されます。

\begin{lstlisting}[style=customasmx86]
movdqu  xmm0, xmmword ptr [ebx+edi*4] ; ar1+i*4
paddd   xmm0, xmmword ptr [esi+edi*4] ; ar2+i*4
movdqa  xmmword ptr [eax+edi*4], xmm0 ; ar3+i*4
\end{lstlisting}

そうでなければ、\TT{ar2}からの値は、 \MOVDQU を使用して\XMM{0}にロードされます。
これは、位置合わせされたポインターを必要としませんが、遅くなる可能性があります。

\lstinputlisting[style=customasmx86]{patterns/19_SIMD/18_1_excerpt_JA.asm}

それ以外の場合は、SSE2以外のコードが実行されます。

\myparagraph{GCC}

\newcommand{\URLGCCVEC}{\url{http://go.yurichev.com/17083}}

\Othree オプションが使用され、SSE2サポートがオンになっている場合、
GCCは単純な場合にもベクトル化することがあります\footnote{GCCベクトル化サポートについての詳細は: \URLGCCVEC}

以下を得ます(GCC 4.4.1)。

\lstinputlisting[style=customasmx86]{patterns/19_SIMD/18_2_gcc_O3.asm}

しかし、ほぼ同じですが、Intel C++ほど細心の注意を払っていません。

\subsubsection{メモリコピーの例}
\label{vec_memcpy}

簡単なmemcpy()の例をもう一度見てみましょう。
(\myref{loop_memcpy}):

\lstinputlisting[style=customc]{memcpy.c}

GCC 4.9.1の最適化によるものです。

\lstinputlisting[caption=\Optimizing GCC 4.9.1 x64,style=customasmx86]{patterns/19_SIMD/memcpy_GCC49_x64_O3_JA.s}

\subsection{SIMD \strlen 実装}
\label{SIMD_strlen}

\newcommand{\URLMSDNSSE}{\href{http://go.yurichev.com/17262}{MSDN: MMX, SSE, and SSE2 Intrinsics}}

SIMD命令は、特別なマクロ\footnote{\URLMSDNSSE}を介して \CCpp コードに挿入できることに注意しなければなりません。 
MSVCの場合、それらのいくつかは\TT{intrin.h}ファイルにあります。

\newcommand{\URLSTRLEN}{http://go.yurichev.com/17330}

\myindex{\CStandardLibrary!strlen()}

SIMD命令を使用して \strlen 関数\footnote{strlen()~---文字列長を計算する標準Cライブラリ関数}
を実装することは、一般的な実装よりも2-2.5倍高速に実行できます。
この関数は16文字をXMMレジスタにロードし、それぞれをゼロと照合します。
\footnote{
この例は、以下のソースコードに基づいています。: \url{\URLSTRLEN}.}.

\lstinputlisting[style=customc]{patterns/19_SIMD/18_3.c}

\Ox オプションを付けてMSVC 2010でコンパイルしましょう。

\lstinputlisting[caption=\Optimizing MSVC 2010,style=customasmx86]{patterns/19_SIMD/18_4_msvc_Ox_JA.asm}

どう機能するでしょうか?
まず第一に、私達は機能の目的を理解しなければなりません。
これはC文字列の長さを計算しますが、別の用語を使用することもできます。タスクはゼロバイトを検索し、次に文字列の開始位置に対する位置を計算することです。

まず、\TT{str}ポインタが16バイト境界に揃っているかどうかを調べます。
そうでなければ、一般的な \strlen 実装を呼び出します。

次に、 \MOVDQA を使用して次の16バイトを\XMM{1}レジスタにロードします。

注意深い読者が尋ねるかもしれません、なぜポインターアライメントに関係なくメモリからデータをロードできるのに
\MOVDQU がここで使用できないのですか?

はい、それはこのようにして行われるかもしれません:もしポインタがアラインメントしていれば、MOVDQAを使用してデータをロードし、
そうでなければより遅い \MOVDQU を使用します。

しかし、ここで我々は別の警告を受けるかもしれません。

\myindex{Page (memory)}
\newcommand{\URLPAGE}{\href{http://go.yurichev.com/17136}{wikipedia}}

\gls{Windows NT}系列の\ac{OS}において(しかしそれに限定されない)、メモリは4KiB(4096バイト)のページによって割り当てられます。
各win32プロセスには4GiBの空き容量がありますが、実際には、アドレス空間の
一部のみが実際の物理メモリに接続されています。
プロセスが存在しないメモリブロックにアクセスしている場合は、例外が発生します。
それが\ac{VM}のしくみです\footnote{\URLPAGE}。

したがって、一度に16バイトをロードする関数は、割り当てられたメモリブロックの境界をまたぐことがあります。 
\ac{OS}がアドレス0x008c0000に8192(0x2000)バイトを割り当てたとしましょう。
したがって、ブロックはアドレス0x008c0000から始まり0x008c1fffまでを含むバイトです。

ブロックの後、つまりアドレス0x008c2000から始まり、そこには何もありません。 \ac{OS}は
そこにメモリを割り当てていません。
そのアドレスからメモリにアクセスしようとすると、例外が発生します。

プログラムがほぼブロックの最後に5文字を含む文字列を保持しているという例を考えてみましょう。
それは違法な行為ではありません。

\begin{center}
  \begin{tabular}{ | l | l | }
    \hline
        0x008c1ff8 & 'h' \\
        0x008c1ff9 & 'e' \\
        0x008c1ffa & 'l' \\
        0x008c1ffb & 'l' \\
        0x008c1ffc & 'o' \\
        0x008c1ffd & '\textbackslash{}x00' \\
        0x008c1ffe & ランダムノイズ \\
        0x008c1fff & ランダムノイズ \\
    \hline
  \end{tabular}
\end{center}

したがって、通常の状態では、プログラムは \strlen を呼び出して、アドレス0x008c1ff8のメモリに
配置された文字列\TT{'hello'}へのポインタを渡します。 
\strlen は0x008c1ffdまで1バイトずつ読み込みます。0x008c1ffdはバイトが0ですが、その後は停止します。

整列されているかどうかにかかわらず、任意のアドレスから始めて一度に16バイトを読み取る独自の \strlen を実装すると、
\MOVDQU はアドレス0x008c1ff8から最大16バイトを一度に0x008c2008までロードしようとして、
例外が発生します。
もちろん、そのような状況は避けるべきです。

そのため、私たちは16バイト境界に整列されたアドレスでのみ動作します。これは、\ac{OS}のページサイズが通常16バイト境界に
整列されているという知識と組み合わせると、ある程度の保証が得られます。
私たちの関数は、割り当てられていないメモリから読み込みません。

私たちの機能に戻りましょう。

\myindex{x86!\Instructions!PXOR}
\verb|_mm_setzero_si128()| は、pxor xmm0、xmm0 .itを生成するマクロで、XMM0レジスタをクリアするだけです。

\verb|_mm_load_si128()| はMOVDQAのマクロで、アドレスからXMM1レジスタに16バイトをロードするだけです。

\myindex{x86!\Instructions!PCMPEQB}
\verb|_mm_cmpeq_epi8()| は、2つのXMMレジスタをバイト単位で比較する命令であるPCMPEQB用のマクロです。

また、あるバイトが他のレジスタのバイトと等しい場合は、
結果のこの時点で\TT{0xff}になり、それ以外の場合は0になります。

例えば:

\begin{verbatim}
XMM1: 0x11223344556677880000000000000000
XMM0: 0x11ab3444007877881111111111111111
\end{verbatim}

\TT{pcmpeqb xmm1, xmm0}の実行後、\XMM{1}レジスターには以下が含まれます。

\begin{verbatim}
XMM1: 0xff0000ff0000ffff0000000000000000
\end{verbatim}

この例では、この命令は各16バイトブロックを16個のゼロバイトのブロックと比較します。
これは、\TT{pxor xmm0, xmm0}によって\XMM{0}レジスタに設定されています。

\myindex{x86!\Instructions!PMOVMSKB}

次のマクロは\TT{\_mm\_movemask\_epi8()}です。これは\TT{PMOVMSKB}命令です。

\PCMPEQB と一緒に使うととても便利です。

\TT{pmovmskb eax, xmm1}

この命令は、\XMM{1}の最初のバイトの最上位ビットが1の場合、最初のEAXビットを1に設定します。
つまり、\XMM{1}レジスタの最初のバイトが\TT{0xff}の場合、 \EAX の最初のビットも1になります。

\XMM{1}レジスタの2番目のバイトが\TT{0xff}の場合、 \EAX の2番目のビットは1に設定されます。
言い換えれば、命令は\q{\XMM{1}のどのバイトに最上位ビット(MBS)が設定されているか、0x7fより大きいか}
という質問に答えます。
そして、 \EAX レジスタに16ビットを返します。 
\EAX レジスタの他のビットはクリアされます。

ところで、私たちのアルゴリズムのこの風変わりなことを忘れないでください。
入力には16バイトあります。

\input{patterns/19_SIMD/strlen_hello_and_garbage}

これは、\TT{'hello'}文字列で、ゼロで終わり、メモリ内のランダムノイズです。

これらの16バイトを\XMM{1}にロードしてゼロ化された\XMM{0}と比較すると、
次のようになります。
\footnote{ここでは、\ac{MSB}から\ac{LSB}への順序が使用されています。}

\begin{verbatim}
XMM1: 0x0000ff00000000000000ff0000000000
\end{verbatim}

これは、命令が2つのゼロバイトを見つけたことを意味していますが、それは驚くことではありません。
 
この場合の\TT{PMOVMSKB}は \EAX を
\emph{0b0010000000100000}に設定します。

明らかに、私たちの関数は最初の0ビットだけを取り、残りを無視しなければなりません。

\myindex{x86!\Instructions!BSF}
\label{instruction_BSF}
次の命令は\TT{BSF} (\emph{Bit Scan Forward})です。

この命令は、1に設定された最初のビットを見つけ、その位置を最初のオペランドに格納します。

\begin{verbatim}
EAX=0b0010000000100000
\end{verbatim}

\TT{bsf eax, eax}の実行後、 \EAX は5を含み、
これは1が5番目のビット位置(ゼロから始まる)に見つかったことを意味します。

MSVCには、この命令用のマクロ\TT{\_BitScanForward}があります。

今は簡単です。ゼロバイトが見つかった場合は、その位置がすでに数えたものに追加され、
今度は結果が返されます。

Almost all.
ほとんど全て。

ちなみに、MSVCコンパイラは最適化のために2つのループ本体を一緒に発行していました。

ちなみに、SSE 4.2(Intel Core i7に登場)は、これらの文字列操作がさらに簡単になる可能性がある場合に、
より多くの命令を提供します:\url{http://go.yurichev.com/17331}


