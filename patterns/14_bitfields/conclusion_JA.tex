% TODO: add ROL/ROR
\subsection{\Conclusion{}}

\myindex{x86!\Instructions!SHR}
\myindex{x86!\Instructions!SHL}
\myindex{x86!\Instructions!SAR}

\CCpp のシフト演算子 \TT{$\ll$} および \TT{$\gg$} と同様に、
x86のシフト命令は \SHR/\SHL (符号なしの値)と \SAR/\SHL (符号付きの値)です。

\myindex{ARM!\Instructions!LSR}
\myindex{ARM!\Instructions!LSL}
\myindex{ARM!\Instructions!ASR}

ARMのシフト命令は、 \LSR/\LSL (符号なし値の場合)と \ASR/\LSL (符号付き値の場合)です。

シフトサフィックスをいくつかの命令
(\q{データ処理命令}と呼ばれます)に追加することもできます。
% FIXME: which instructions?

\subsubsection{特定のビットをチェックする(コンパイル段階で知られている)}

0b1000000ビット(0x40)がレジスタの値に存在するかどうかをテストします。

\lstinputlisting[caption=\CCpp,style=customc]{patterns/14_bitfields/c_snippet0.c}

\lstinputlisting[caption=x86,style=customasmx86]{patterns/14_bitfields/TEST_JNZ_JA.lst}

\lstinputlisting[caption=x86,style=customasmx86]{patterns/14_bitfields/TEST_JZ_JA.lst}

\lstinputlisting[caption=ARM (\ARMMode),style=customasmARM]{patterns/14_bitfields/TST_BNE_JA.lst}

\myindex{x86!\Instructions!AND}
\myindex{x86!\Instructions!TEST}

場合によっては、 \TEST の代わりに \AND が使用されますが、設定されるフラグは同じです。

\subsubsection{特定のビットをチェックする(実行時に指定する)}

これは通常、この \CCpp コードスニペットによって行われます( $n$ ビット右にシフトし、次に最下位ビットをカットします)。

\lstinputlisting[caption=\CCpp,style=customc]{patterns/14_bitfields/c_snippet1.c}

これは通常、x86コードで次のように実装されています。

\begin{lstlisting}[caption=x86,style=customasmx86]
; REG=input\_value
; CL=n
SHR REG, CL
AND REG, 1
\end{lstlisting}

または(1ビット左シフトを $n$ 回、入力値でこのビットを分離し、ゼロでないかどうかをチェックする)

\lstinputlisting[caption=\CCpp,style=customc]{patterns/14_bitfields/c_snippet2.c}

これは通常、x86コードで次のように実装されています。

\begin{lstlisting}[caption=x86,style=customasmx86]
; CL=n
MOV REG, 1
SHL REG, CL
AND input_value, REG
\end{lstlisting}

\subsubsection{特定のビットを設定する(コンパイル時に知られている)}

\begin{lstlisting}[caption=\CCpp]
value=value|0x40;
\end{lstlisting}

\begin{lstlisting}[caption=x86,style=customasmx86]
OR REG, 40h
\end{lstlisting}

\begin{lstlisting}[caption=ARM (\ARMMode) and ARM64,style=customasmARM]
ORR R0, R0, #0x40
\end{lstlisting}

\subsubsection{特定のビットを設定する(実行時に指定する)}

\lstinputlisting[caption=\CCpp,style=customc]{patterns/14_bitfields/c_snippet3.c}

これは通常、x86コードで次のように実装されています。

\begin{lstlisting}[caption=x86,style=customasmx86]
; CL=n
MOV REG, 1
SHL REG, CL
OR input_value, REG
\end{lstlisting}

\subsubsection{明確な特定のビット(コンパイル段階で知られている)}

逆の値で \AND 演算を適用するだけです:

\begin{lstlisting}[caption=\CCpp,style=customc]
value=value&(~0x40);
\end{lstlisting}

\begin{lstlisting}[caption=x86,style=customasmx86]
AND REG, 0FFFFFFBFh
\end{lstlisting}

\begin{lstlisting}[caption=x64,style=customasmx86]
AND REG, 0FFFFFFFFFFFFFFBFh
\end{lstlisting}

これは、実際には、1を除いてすべてのビットを設定しています。

\myindex{ARM!\Instructions!BIC}

ARMモードのARMには \BIC 命令があります。これは \NOT+\AND 命令ペアのように動作します。

\begin{lstlisting}[caption=ARM (\ARMMode),style=customasmARM]
BIC R0, R0, #0x40
\end{lstlisting}

\subsubsection{
特定のビットをクリア(実行時に指定)}

\lstinputlisting[caption=\CCpp,style=customc]{patterns/14_bitfields/c_snippet4.c}

\begin{lstlisting}[caption=x86,style=customasmx86]
; CL=n
MOV REG, 1
SHL REG, CL
NOT REG
AND input_value, REG
\end{lstlisting}
