\subsection{特定ビットのセットやクリア: \ac{FPU} の例}

\myindex{IEEE 754}

IEEE 754形式で \Tfloat 型がどのように配置されるのか見てみます。

\bigskip
% a hack used here! http://tex.stackexchange.com/questions/73524/bytefield-package
\begin{center}
\begin{bytefield}{32}
	\bitheader[endianness=big]{0,22,23,30,31} \\
	\bitbox{1}{S} &
	\bitbox{8}{%
		\RU{экспонента}%
		\EN{exponent}%
		\ES{exponente}%
		\PTBRph{}%
		\DEph{}\PLph{}%
		\IT{esponente}%
		\FR{exposant}%
		\JA{指数}
	} &
	\bitbox{23}{%
		\RU{мантисса}%
		\EN{mantissa or fraction}%
		\ES{mantisa o fracci\'on}%
		\PTBRph{}%
		\DEph{}
		\PLph{}%
		\IT{mantissa}%
		\FR{mantisse ou fraction}%
		\JA{対数または比}
	}
\end{bytefield}
\end{center}

\begin{center}
( S --- %
	\RU{знак}%
	\EN{sign}%
	\ES{signo}%
	\PTBRph{}%
	\DEph{}\PLph{}%
	\IT{segno}%
	\FR{signe}%
	\JA{記号}
)
\end{center}


数字の符号情報は\ac{MSB}にあります。
FPU命令なしで浮動小数点数の符号を変更することは可能でしょうか?

\lstinputlisting[style=customc]{patterns/14_bitfields/35_set_reset_FPU/abs.c}

実際の変換をせずに \Tfloat 値との間でコピーを行うには、 \CCpp でこのトリックが必要です。 
したがって、3つの関数があります。 my\_abs() は\ac{MSB}をリセットします。 \TT{set\_sign()}は\ac{MSB}を設定します。 negate()はそれを反転させます。

\XOR を使ってビットを反転することができます:\myref{XOR_property}

\subsubsection{x86}

コードはかなり簡単です。

\lstinputlisting[caption=\Optimizing MSVC 2012,style=customasmx86]{patterns/14_bitfields/35_set_reset_FPU/abs_MSVC2012_Ox.asm}

\Tfloat 型の入力値はスタックから取得されますが、整数値として扱われます。

\AND と \OR は望むビットをリセットそしてセットします。
\XOR は反転します。

最後に、変更された値が\TT{ST0}にロードされます。浮動小数点数はこのレジスタにリターンされるからです。

さて、x64向けの最適化MSVC 2012で試してみましょう。

\lstinputlisting[caption=\Optimizing MSVC 2012 x64,style=customasmx86]{patterns/14_bitfields/35_set_reset_FPU/abs_MSVC2012_x64_Ox.asm}

\myindex{x86!\Instructions!BTR}
\myindex{x86!\Instructions!BTS}
\myindex{x86!\Instructions!BTC}

入力値は\TT{XMM0}に渡され、そしてローカルスタックにコピーされて、
新しい命令がいくつか見られます。\BTR 、 \BTS 、 \BTC です。

各命令は特定のビットをリセット(\BTR)、セット(\BTS)そして反転(または補数: \BTC)するのに用いられます。
0から数えて31番目のビットは\ac{MSB}です。

最後に、結果は\TT{XMM0}にコピーされます。浮動小数点数の値はWin64の環境では\TT{XMM0}を通してリターンされる
からです。

\subsubsection{MIPS}

GCC 4.4.5でMIPS向けのコードはほとんど同じです。

\lstinputlisting[caption=\Optimizing GCC 4.4.5 (IDA),style=customasmMIPS]{patterns/14_bitfields/35_set_reset_FPU/MIPS_O3_IDA_JA.lst}

\myindex{MIPS!\Instructions!LUI}

単一の \LUI 命令が使用され、レジスタに0x80000000がロードされます。
\LUI は低位16ビットをクリアし、ゼロにするので、 後続に \ORI がなくても \LUI で十分です。

\subsubsection{ARM}

\myparagraph{\OptimizingKeilVI (\ARMMode)}

\lstinputlisting[caption=\OptimizingKeilVI (\ARMMode),style=customasmARM]{patterns/14_bitfields/35_set_reset_FPU/abs_Keil_ARM_O3_JA.s}

ここまでは順調です。
\myindex{ARM!\Instructions!BIC}
\myindex{ARM!\Instructions!EOR}

ARMは \BIC 命令があり、特定のビットを明示的にクリアします。
\EOR はARM命令で \XOR のことです
(\q{Exclusive OR})。

\myparagraph{\OptimizingKeilVI (\ThumbMode)}

\lstinputlisting[caption=\OptimizingKeilVI (\ThumbMode),style=customasmx86]{patterns/14_bitfields/35_set_reset_FPU/abs_Keil_thumb_O3.s}

ARMのThumbモードは16ビット命令を提供します。そんなに多くのデータをエンコードはできないので、
\INS{MOVS}/\INS{LSLS} 命令ペアが定数0x80000000を形づくるのに使用されます。
このように動作します:$1<<31 = 0x80000000$

\myindex{ARM!\Instructions!LSLS}
\myindex{ARM!\Instructions!LSRS}

\TT{my\_abs}のコードは奇妙で、この式のように効果的に機能します:$(i<<1)>>1$
この文は無意味に見えます。 しかし、 $input<<1$ が実行されると、\ac{MSB}(符号ビット)がドロップされるだけです。
後続の $result>>1$ 文が実行されると、すべてのビットが現在自分の場所にありますが、
シフト演算から出現するすべての\q{新しい}ビットは常にゼロであるため、\ac{MSB}はゼロになります。 
これが \LSLS/\LSRS 命令ペアが\ac{MSB}をクリアする方法です。

\myparagraph{\Optimizing GCC 4.6.3 (Raspberry Pi, \ARMMode)}

\lstinputlisting[caption=\Optimizing GCC 4.6.3 for Raspberry Pi (\ARMMode),style=customasmARM]{patterns/14_bitfields/35_set_reset_FPU/raspberry_GCC_O3_ARM_mode_JA.lst}

QEMUでRaspberry Pi Linuxを動かしてARM FPUをエミュレートしてみましょう。Rレジスタの代わりにSレジスタが浮動小数点数に
使用されます。

\myindex{ARM!\Instructions!FMRS}

\FMRS 命令は\ac{GPR}からFPUそして逆にもデータをコピーします。

\TT{my\_abs()} と \TT{set\_sign()} は期待通りに見えますが、negate()はどうでしょうか?
\XOR の代わりに \ADD があるのはどうしてでしょうか?

\myindex{ARM!\Instructions!XOR}
\myindex{ARM!\Instructions!ADD}
信じがたいかもしれませんが、命令
\INS{ADD register, 0x80000000}は
\INS{XOR register, 0x80000000}のように動作します。
まず、ゴールは何でしょうか?
ゴールは\ac{MSB}を反転させることなので、 \XOR 演算は忘れましょう。
学校レベルの数学からは他の値に1000を加算することは最後の3桁には影響しないと
思うかもしれません。
例えば:$1234567 + 10000 = 1244567$ (最後の4桁は影響を受けない)

しかし、ここではバイナリベースで操作すると、
0x80000000 は 0b100000000000000000000000000000000で、すなわち最高位のビットのみセットされます。

任意の値に0x80000000を加算すると低位の31ビットに影響しませんが、\ac{MSB}だけ影響します。
0に1を加算すると結果は1です。

1に1を加算すると結果は バイナリ形式で0b10になりますが、(0から数えて)32番目のビットはドロップします。
レジスタは32ビット幅なので、結果は0になります。
\XOR が \ADD で置き換え可能なのはそのためです。

GCCがなぜこうすると決定したかはわかりませんが、正しく動作します。
