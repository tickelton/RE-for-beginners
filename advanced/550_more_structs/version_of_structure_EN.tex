\subsection{Version of C structure}

Many Windows programmers have seen this in MSDN:

\begin{lstlisting}
SizeOfStruct
    The size of the structure, in bytes. This member must be set to sizeof(SYMBOL_INFO).
\end{lstlisting}

( \url{https://msdn.microsoft.com/en-us/library/windows/desktop/ms680686(v=vs.85).aspx} )

Some structures like \emph{SYMBOL\_INFO} has started with this field indeed. Why?
This is some kind of structure version.

Imagine you have a function which draws circle.
It takes a single argument---a pointer to a structure with only three fields: X, Y and radius.
And then color displays flooded a market, sometimes in 1980s. And you want to add \emph{color} argument to the function.
But, let's say, you cannot add another argument to it (a lot of software use your \ac{API} and cannot be recompiled).
And if the old piece of software uses your \ac{API} with color display,
let your function draw a circle in (default) black and white colors.

Another day you add another feature: circle now can be filled, and brush type can be set.

Here is one solution to the problem:

\lstinputlisting[style=customc]{advanced/550_more_structs/src_EN.c}

In other words, \emph{SizeOfStruct} field takes a role of \emph{version of structure} field.
It could be enumerate type (1, 2, 3, etc.), but to set \emph{SizeOfStruct} field to \emph{sizeof(struct...)}
is less prone to mistakes/bugs: we just write \emph{s.SizeOfStruct=sizeof(...)} in caller's code.

In C++, this problem is solved using \emph{inheritance} (\myref{cpp_inheritance}).
You just extend your base class (let's call it \emph{Circle}),
and then you will have \emph{ColoredCircle} and then \emph{FilledColoredCircle}, and so on.
A current \emph{version} of an object (or, more precisely, current \emph{type}) will be determined using C++ \ac{RTTI}.

So when you see \emph{SizeOfStruct} somewhere in \ac{MSDN}---perhaps this structure was extended at least once in past.

