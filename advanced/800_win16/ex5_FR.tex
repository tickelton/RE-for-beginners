\subsection{\Example{} \#5}
\label{win16_near_far_pointers}

\lstinputlisting[style=customc]{\CURPATH/ex5.c}

\lstinputlisting[style=customasmx86]{\CURPATH/ex5.lst}

\myindex{Intel!8086!Modèle de mémoire}
Nous voyons ici une différence entre les pointeurs appelés \q{near} et \q{far}: un
autre effet bizarre de la mémoire segmentée en 16-bit 8086.

Vous pouvez en lire plus à ce sujet ici: \myref{8086_memory_mode}.

Les pointeurs \q{near} sont ceux qui pointent dans le segment de données courant.
C'est pourquoi la fonction \TT{string\_compare()} prend seulement deux pointeurs 16-bit,
et accède des données dans le segment sur lequel \TT{DS} pointe (L'instruction \TT{mov al, [bx]}
fonctionne en fait comme \TT{mov al, ds:[bx]}---\TT{DS} est implicite ici).

Les pointeurs \q{far} sont ceux qui pointent sur des données dans un autre segment
de mémoire.
C'est pourquoi \TT{string\_compare\_far()} prend la paire de 16-bit comme un pointeur,
charge la partie haute dans le registre de segment \TT{ES} et accède aux données
à travers lui (\TT{mov al, es:[bx]}).
Les pointeurs \q{far} sont aussi utilisés dans mon exemple win16
\TT{MessageBox()}: \myref{win16_messagebox}.
En effet, le noyau de Windows n'est pas au courant du segment de données qui doit être
utilisé pour accéder aux chaînes de texte, donc il a besoin de l'information complète.
La raison de cette distinction est qu'un programme compact peut n'utiliser qu'un
segment de données de 64kb, donc il n'a pas besoin de passer la partie haute de l'adresse,
qui est toujours la même.
Un programme plus gros peut utiliser plusieurs segments de données de 64kb, donc il
doit spécifier le segment de données à chaque fois.

C'est la même histoire avec les segments de code.
Un programme compact peut avoir tout son code exécutable dans un seul segment de 64kb,
donc toutes les fonctions y seront appelées en utilisant l'instruction \TT{CALL NEAR},
et le contrôle du flux sera renvoyé en utilisant \TT{RETN}.
Mais si il y a plusieurs segments de code, alors l'adresse d'une fonction devra être
spécifiée par une paire, et sera appelée en utilisant l'instruction \TT{CALL FAR},
et le contrôle du flux renvoyé en utilisant \TT{RETF}.

Ceci est ce qui est mis dans le compilateur en spécifiant le \q{modèle de mémoire}.

Les compilateurs qui ciblent MS-DOS et Win16 ont des bibliothèques spécifiques pour
chaque modèle de mémoire: elles diffèrent par le type de pointeurs pour le code et
les données.

