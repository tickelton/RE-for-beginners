\subsection{Optimisation étrange de boucle}

Ceci est une des fonctions memcpy() les plus simple jamais implémentée:

\begin{lstlisting}[style=customc]
void memcpy (unsigned char* dst, unsigned char* src, size_t cnt)
{
	size_t i;
	for (i=0; i<cnt; i++)
		dst[i]=src[i];
};
\end{lstlisting}

Au moins MSVC 6.0 de la fin des années 1990 jusqu'à MSVC 2013 peuvent produire du
code vraiment étrange (ce listing est généré par MSVC 2013 x86):

\lstinputlisting[style=customasmx86]{advanced/500_loop_optimizations/1_1_FR.lst}

Ceci est étrange, car comment travaille les humains avec deux pointeurs? Ils stockent
les deux adresses dans deux registres ou deux emplacements mémoire.
Dans ce cas, le compilateur MSVC stocke les deux pointeurs comme un pointeur (\emph{dst glissant} dans \EAX)
et la différence entre les pointeurs \emph{src} et \emph{dst} (qui reste inchangée lors
de l'exécution du corps de la boucle) dans \ESI.
\myindex{\CLanguageElements!ptrdiff\_t}
(À propos, ceci est un des rare cas où le type de donnée ptrdiff\_t peut être utilisé.)
Lorsqu'il doit charger un octet depuis \emph{src}, il le charge en \emph{diff + dst glissant}
et stocke l'octet juste en \emph{dst glissant}.

Ceci est plus une astuce d'optimisation. Mais j'ai récrit cette fonction en:

\lstinputlisting[style=customasmx86]{advanced/500_loop_optimizations/1_2.lst}

\dots et ça fonctionne aussi efficacement que la version \emph{optimisée} sur mon Intel
Xeon E31220 @ 3.10GHz.
Peut-être que cette optimisation ciblait des vieux CPUs x86 des années 1990, puisque
ce truc est utilisé au moins par l'ancien MS VC 6.0?

Une idée?

\myindex{Hex-Rays}
Hex-Rays 2.2 a du mal à reconnaître des schémas comme ça (avec de la chance, temporairement?):

\begin{lstlisting}[style=customc]
void __cdecl f1(char *dst, char *src, size_t size)
{
  size_t counter; // edx@1
  char *sliding_dst; // eax@2
  char tmp; // cl@3

  counter = size;
  if ( size )
  {
    sliding_dst = dst;
    do
    {
      tmp = (sliding_dst++)[src - dst];         // difference (src-dst) is calculated once, at the beginnning
      *(sliding_dst - 1) = tmp;
      --counter;
    }
    while ( counter );
  }
}
\end{lstlisting}

Néanmoins, cette astuce d'optimisation est souvent utilisée par MSVC (pas uniquement
dans des routines \emph{memcpy()} \ac{DIY} maison, mais dans de nombreuses boucles
qui utilisent deux tableaux ou plus).
Donc, ça vaut le coup pour les rétro-ingénieurs de la garder à l'esprit.

% <!-- As of why writting occurred after <b>dst</b> incrementing? -->

