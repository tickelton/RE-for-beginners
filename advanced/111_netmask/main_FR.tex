\mysection{Exemple de calcul d'adresse réseau}

Comme nous le savons, une adresse (IPv4) consiste en quatre nombres dans l'intervalle
$0 \ldots 255$, i.e., quatre octets.

Quatre octets peuvent être stockés facilement dans une variable 32-bit, donc une
adresse IPv4 d'hôte, de masque réseau ou d'adresse de réseau peuvent toutes être
un entier 32-bit.

Du point de vue de l'utilisateur, le masque réseau est défini par quatre nombres et est
formaté comme 255.255.255.0, mais les ingénieurs réseaux (sysadmins) utilisent une
notation plus compacte comme \q{/8}, \q{/16}, etc.

Cette notation défini simplement le nombre de bits qu'a le masque, en commençant
par le \ac{MSB}.

\small
\begin{center}
\begin{tabular}{ | l | l | l | l | l | l | }
\hline
\HeaderColor Masque &
\HeaderColor Hôte &
\HeaderColor Utilisable &
\HeaderColor Masque de réseau &
\HeaderColor Masque hexadécimal &
\HeaderColor \\
\hline
/30  & 4        & 2        & 255.255.255.252  & 0xfffffffc  & \\
\hline
/29  & 8        & 6        & 255.255.255.248  & 0xfffffff8  & \\
\hline
/28  & 16       & 14       & 255.255.255.240  & 0xfffffff0  & \\
\hline
/27  & 32       & 30       & 255.255.255.224  & 0xffffffe0  & \\
\hline
/26  & 64       & 62       & 255.255.255.192  & 0xffffffc0  & \\
\hline
/24  & 256      & 254      & 255.255.255.0    & 0xffffff00  & réseau de classe C \\
\hline
/23  & 512      & 510      & 255.255.254.0    & 0xfffffe00  & \\
\hline
/22  & 1024     & 1022     & 255.255.252.0    & 0xfffffc00  & \\
\hline
/21  & 2048     & 2046     & 255.255.248.0    & 0xfffff800  & \\
\hline
/20  & 4096     & 4094     & 255.255.240.0    & 0xfffff000  & \\
\hline
/19  & 8192     & 8190     & 255.255.224.0    & 0xffffe000  & \\
\hline
/18  & 16384    & 16382    & 255.255.192.0    & 0xffffc000  & \\
\hline
/17  & 32768    & 32766    & 255.255.128.0    & 0xffff8000  & \\
\hline
/16  & 65536    & 65534    & 255.255.0.0      & 0xffff0000  & réseau de classe B \\
\hline
/8   & 16777216 & 16777214 & 255.0.0.0        & 0xff000000  & réseau de classe A \\
\hline
\end{tabular}
\end{center}
\normalsize

Voici un petit exemple, qui calcule l'adresse du réseau en appliquant le masque réseau
à l'adresse de l'hôte.

\lstinputlisting[style=customc]{\CURPATH/netmask.c}

\subsection{calc\_network\_address()}

La fonction \TT{calc\_network\_address()} est la plus simple:
elle effectue simplement un AND entre l'adresse de l'hôte et le masque de réseau,
dont le résultat est l'adresse du réseau.

\lstinputlisting[caption=MSVC 2012 \Optimizing\ /Ob0,numbers=left,style=customasmx86]{\CURPATH/calc_network_address_MSVC_2012_Ox.asm}

À la ligne 22, nous voyons le plus important \AND---ici l'adresse du réseau est calculée.

\subsection{form\_IP()}

La fonction \TT{form\_IP()} met juste les 4 octets dans une valeur 32-bit.

Voici comment cela est fait habituellement:

\begin{itemize}
\item Allouer une variable pour la valeur de retour.  La mettre à 0.

\item Prendre le 4ème octet (de poids le plus faible), appliquer l'opération OR à
cet octet et renvoyer la valeur.

\item Prendre le troisième octet, le décaler à gauche de 8 bits.
Vous obtenez une valeur comme \TT{0x0000bb00} où \TT{bb} est votre troisième octet.
Appliquer l'opération OR à la valeur résultante.
La valeur de retour contenait \TT{0x000000aa} jusqu'à présent, donc effectuer un
OU logique des valeurs produira une valeur comme \TT{0x0000bbaa}.

\item Prendre le second octet, le décaler à gauche de 16 bits.
Vous obtenez une valeur comme \TT{0x00cc0000} où \TT{cc} est votre deuxième octet.
Appliquer l'opération OR à la valeur résultante.
La valeur de retour contenait \TT{0x0000bbaa} jusqu'à présent, donc effectuer un
OU logique des valeurs produira une valeur comme \TT{0x00ccbbaa}.

\item Prendre le premier octet, le décaler à gauche de 24 bits.
Vous obtenez une valeur comme \TT{0xdd000000} où \TT{dd} est votre premier octet.
Appliquer l'opération OR à la valeur résultante.
La valeur de retour contenait \TT{0x00ccbbaa} jusqu'à présent, donc effectuer un
OU logique des valeurs produira une valeur comme \TT{0xddccbbaa}.

\end{itemize}

Voici comment c'est fait par MSVC 2012 sans optimisation:

\lstinputlisting[caption=MSVC 2012 \NonOptimizing,style=customasmx86]{\CURPATH/form_IP_MSVC_2012_FR.asm}

Certes, l'ordre est différent, mais, bien sûr, l'ordre des opérations n'a pas d'importance.

MSVC 2012 \Optimizing produit en fait la même chose, mais d'une façon différente:

\lstinputlisting[caption=MSVC 2012 \Optimizing\ /Ob0,style=customasmx86]{\CURPATH/form_IP_MSVC_2012_Ox_FR.asm}

Nous pourrions dire que chaque octet est écrit dans les 8 bits inférieurs de la valeur
de retour, et qu'elle est ensuite décalée à gauche d'un octet à chaque étape.

Répéter 4 fois pour chaque octet en entrée.

\par
C'est tout! Malheureusement, il n'y sans doute pas d'autre moyen de le faite.

Il n'y a pas de \ac{CPU}s ou d'\ac{ISA}s répandues qui possède une instruction pour
composer une valeur à partir de bits ou d'octets.

C'est d'habitude fait par décalage de bit et OU logique.

\subsection{print\_as\_IP()}

La fonction \TT{print\_as\_IP()} effectue l'inverse: séparer une valeur 32-bit en
4 octets.

Le découpage fonctionne un peu plus simplement: il suffit de décaler la valeur en
entrée de 24, 16, 8 ou 0 bits, prendre les 8 bits d'indice 0 à 7 (octet de poids
faible), et c'est fait:

\lstinputlisting[caption=MSVC 2012 \NonOptimizing,style=customasmx86]{\CURPATH/print_as_IP_MSVC_2012_FR.asm}

MSVC 2012 \Optimizing fait presque la même chose, mais sans recharger inutilement
la valeur en entrée:

\lstinputlisting[caption=MSVC 2012 \Optimizing\ /Ob0,style=customasmx86]{\CURPATH/print_as_IP_MSVC_2012_Ox.asm}

\subsection{form\_netmask() et set\_bit()}

La fonction \TT{form\_netmask()} produit un masque de réseau à partir de la notation
\ac{CIDR}.
Bien sûr, il serait plus efficace d'utiliser une sorte de table pré-calculée, mais
nous utilisons cette façon de faire intentionnellement, afin d'illustrer les décalages
de bit.

Nous allons aussi écrire une fonction séparées \TT{set\_bit()}.
Ce n'est pas une très bonne idée de créer un fonction pour une telle opération primitive,
mais cela facilite la compréhension du fonctionnement.

\lstinputlisting[caption=MSVC 2012 \Optimizing\ /Ob0,style=customasmx86]{\CURPATH/form_netmask_MSVC_2012_Ox.asm}

\TT{set\_bit()} est primitive: elle décale juste 1 à gauche du nombre de bits dont
nous avons besoin et puis effectue un OU logique avec la valeur \q{input}.
\TT{form\_netmask()} a une boucle: elle met autant de bits (en partant du \ac{MSB})
que demandé dans l'argument \TT{netmask\_bits}.

\subsection{Résumé}

C'est tout!
Nous le lançons et obtenons:

\begin{lstlisting}
netmask=255.255.255.0
network address=10.1.2.0
netmask=255.0.0.0
network address=10.0.0.0
netmask=255.255.255.128
network address=10.1.2.0
netmask=255.255.255.192
network address=10.1.2.64
\end{lstlisting}
