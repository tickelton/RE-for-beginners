\mysection{C99 restrict}
\myindex{\CLanguageElements!C99!restrict}
\myindex{Fortran}

Voici une raison pour laquelle les programmes en Fortran, dans certains cas, fonctionnent
plus vite que ceux en \CCpp.

\begin{lstlisting}[style=customc]
void f1 (int* x, int* y, int* sum, int* product, int* sum_product, int* update_me, size_t s)
{
	for (int i=0; i<s; i++)
	{
		sum[i]=x[i]+y[i];
		product[i]=x[i]*y[i];
		update_me[i]=i*123; // some dummy value
		sum_product[i]=sum[i]+product[i];	
	};
};
\end{lstlisting}

C'est un exemple très simple, qui contient une spécificité:
le pointeur sur le tableau \TT{update\_me} peut-être un pointeur sur le tableau
\TT{sum}, le tableau \TT{product} ou même le tableau \TT{sum\_product}---rien
ne l'interdit, n'est-ce pas?

Le compilateur est parfaitement conscient de ceci, donc il génère du code avec quatre
étapes dans le corps de la boucle:
\begin{itemize}
\item calcule le \TT{sum[i]} suivant
\item calcule le \TT{product[i]} suivant
\item calcule le \TT{update\_me[i]} suivant
\item calcule le \TT{sum\_product[i]} suivant---à cette étape, nous devons charger
depuis la mémoire les valeurs \TT{sum[i]} et \TT{product[i]} déjà calculées
\end{itemize}

Et-il possible d'optimiser la dernière étape?
Puisque nous avons déjà calculé \TT{sum[i]} et \TT{product[i]}, il n'est pas nécessaire
de les charger à nouveau depuis la mémoire.

Oui, mais le compilateurs n'est pas sûr que rien n'a été récris à la 3ème étape!
Ceci est appelé \q{pointer aliasing},
une situation dans laquelle le compilateur ne peut pas être sûr que la mémoire sur
laquelle le pointeur pointe n'a pas été modifiée.

\emph{restrict} dans le standard C99 \InSqBrackets{\CNineNineStd{} 6.7.3/1}
est une promesse faite par le programmeur au compilateur que les arguments de la
fonction marqués par ce mot-clef vont toujours pointer vers des case mémoire différentes
et ne vont jamais se recouper.

Pour être plus précis et décrire ceci formellement, \emph{restrict} indique que seul
ce pointeur est utilisé pour accéder un objet, et qu'aucun autre pointeur ne sera
utilisé pour ceci.

On peut même dire que l'objet ne sera accéder que par un seul pointeur, si il est
marqué comme \emph{restrict}.

Ajoutons ce mot-clef à chaque argument pointeur:

\begin{lstlisting}[style=customc]
void f2 (int* restrict x, int* restrict y, int* restrict sum, int* restrict product, int* restrict sum_product, 
	int* restrict update_me, size_t s)
{
	for (int i=0; i<s; i++)
	{
		sum[i]=x[i]+y[i];
		product[i]=x[i]*y[i];
		update_me[i]=i*123; // some dummy value
		sum_product[i]=sum[i]+product[i];	
	};
};
\end{lstlisting}

Regardons le résultat:

\lstinputlisting[caption=GCC x64: f1(),style=customasmx86]{\CURPATH/f1_FR.asm}

\lstinputlisting[caption=GCC x64: f2(),style=customasmx86]{\CURPATH/f2_FR.asm}

La différence entre les fonctions \TT{f1()} et \TT{f2()} compilées est la suivante:
dans \TT{f1()}, \TT{sum[i]} et \TT{product[i]}
sont rechargés au milieu de la boucle, et il n'y a rien de tel dans \TT{f2()}, les
valeurs déjà calculées sont utilisées, puisque nous avons \q{promis} au compilateur
que rien ni personne ne changera les valeurs pendant l'exécution du corps de la boucle,
donc il est \q{certain} qu'il n'y a pas besoin de recharger la valeur depuis la mémoire.

Étonnamment, le second exemple est plus rapide.

Mais que se passe-t-il si les pointeurs dans les arguments de la fonction se
modifient d'une manière ou d'une autre?

Ceci est du ressort de la conscience du programmeur, et le résultat sera incorrect.

Retournons au Fortran.

Les compilateurs de ce langage traitent tous les pointeurs de cette façon, donc lorsqu'il
n'est pas possible d'utiliser \emph{restrict} en C, Fortran peut générer du code plus
rapide dans ces cas.

À quel point est-ce pratique?

Dans les cas où le fonction travaille avec des gros blocs en mémoire.

C'est le cas en algèbre linéaire, par exemple.

Les superordinateurs/\ac{HPC} utilisent beaucoup d'algèbre linéaire, c'est probablement
pourquoi,\\
traditionnellement, Fortran y est encore utilisé [Eugene Loh, \emph{The Ideal HPC Programming Language}, (2010)].

Mais lorsque le nombre d'itérations n'est pas très important, certainement, le gain
en vitesse ne doit pas être significatif.
