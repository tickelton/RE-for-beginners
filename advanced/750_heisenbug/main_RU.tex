\mysection{Еще одна heisenbug-а}
\label{GlobalArraysOverflowHeisenbug}

Иногда, переполнение массива (или буфера) может привести к \emph{ошибке заборного столба (fencepost error)}:

\lstinputlisting[style=customc]{advanced/750_heisenbug/1.c}

Вот что выводится в моем случае (неоптимизирующий GCC 5.4 x86 на Linux):

\begin{lstlisting}
important_var1=1
important_var2=456
important_var3=3
important_var4=4
important_var5=5
\end{lstlisting}

Как бывает, \TT{important\_var2} может быть расположена компилятором сразу за \TT{array1[]}:

\begin{lstlisting}[caption=objdump -x]
0804a040 g     O .bss   00000200              array1
...
0804a240 g     O .bss   00000004              important_var2
0804a244 g     O .bss   00000004              important_var4
...
0804a248 g     O .bss   00000004              important_var1
0804a24c g     O .bss   00000004              important_var3
0804a250 g     O .bss   00000004              important_var5
\end{lstlisting}

Другой компилятор может расположить переменные в другом порядке, и другая переменная затрется.
\myindex{Heisenbug}
Это также \emph{heisenbug}-а (\myref{heisenbug}) --- ошибка может появится или может оставаться незамеченной в зависимости
от версии компилятора и флагов оптимизации.

\myindex{Valgrind}
Если все переменные и массивы расположены в локальном стеке, защита стека может сработать, а может и нет.
Хотя, Valgrind может находить такие ошибки.

Еще один пример в этой книге (игра Angband): \myref{Angband}.

