\subsection{Cas Pin}

\myindex{Pin}
Il est intéressant de noter que certaines fonctions du framework \ac{DBI} Pin prennent
un nombre variable d'arguments:

\begin{lstlisting}[style=customc]
            INS_InsertPredicatedCall(
                ins, IPOINT_BEFORE, (AFUNPTR)RecordMemRead,
                IARG_INST_PTR,
                IARG_MEMORYOP_EA, memOp,
                IARG_END);
\end{lstlisting}
( \TT{pinatrace.cpp} )

Et voici comment la fonction \TT{INS\_InsertPredicatedCall()} est déclarée:

\begin{lstlisting}[style=customc]
extern VOID INS_InsertPredicatedCall(INS ins, IPOINT ipoint, AFUNPTR funptr, ...);
\end{lstlisting}
( \TT{pin\_client.PH} )

Ainsi, les constantes avec un nom débutant par \TT{IARG\_} sont des sortes d'arguments
pour la fonction, qui sont manipulés à l'intérieur de \TT{INS\_InsertPredicatedCall()}.
Vous pouvez passer autant d'arguments que vous en avez besoin.
Certaines commandes ont des arguments additionnels, d'autres non.
Liste complète des arguments:
\url{https://software.intel.com/sites/landingpage/pintool/docs/58423/Pin/html/group__INST__ARGS.html}.
Et il faut un moyen pour détecter la fin de la liste des arguments, donc la liste
doit être terminée par la constante \TT{IARG\_END}, sans laquelle, la fonction essayerait
de traiter les données indéterminées dans la pile locale comme des arguments additionnels.

\index{Python}
Aussi, dans [\RobPikePractice] nous pouvons trouver un bel exemple de routines C/C++
très similaires à \emph{pack/unpack}\footnote{\url{https://docs.python.org/3/library/struct.html}}
en Python.

