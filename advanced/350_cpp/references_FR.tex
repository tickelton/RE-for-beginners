\subsection{Références}
\myindex{\Cpp!References}
\label{cpp_references}

En \Cpp, les références sont aussi des pointeurs (\myref{label_pointers}), mais
elles sont dites \emph{sûre}, car il est plus difficile de faire une erreur en les
utilisant (C++11 8.3.2). % FIXME link

Par exemple, les références doivent toujours pointer sur un objet de type correspondant
et ne peuvent pas être NULL
\InSqBrackets{\ParashiftCPPFAQ 8.6}.

Encore mieux que ça, les références ne peuvent être changées, il est impossible de
les faire pointer sur un autre objet (réassigner)
\InSqBrackets{\ParashiftCPPFAQ 8.5}.

Si nous essayons de modifier l'exemple avec des pointeurs (\myref{label_pointers})
pour utiliser des références à la place \dots

\begin{lstlisting}[style=customc]
void f2 (int x, int y, int & sum, int & product)
{
	sum=x+y;
	product=x*y;
};
\end{lstlisting}

\dots alors nous pouvons voir que le code compilé est simplement le même que dans
l'exemple avec les pointeurs (\myref{label_pointers}):

\begin{lstlisting}[caption=MSVC 2010 \Optimizing,style=customasmx86]
_x$ = 8		; size = 4
_y$ = 12	; size = 4
_sum$ = 16	; size = 4
_product$ = 20	; size = 4
?f2@@YAXHHAAH0@Z PROC	; f2
	mov	ecx, DWORD PTR _y$[esp-4]
	mov	eax, DWORD PTR _x$[esp-4]
	lea	edx, DWORD PTR [eax+ecx]
	imul eax, ecx
	mov ecx, DWORD PTR _product$[esp-4]
	push esi
	mov	esi, DWORD PTR _sum$[esp]
	mov	DWORD PTR [esi], edx
	mov	DWORD PTR [ecx], eax
	pop	esi
	ret	0
?f2@@YAXHHAAH0@Z ENDP	; f2
\end{lstlisting}

(La raison pour laquelle les fonctions \Cpp ont des noms aussi étranges est expliquée ici: \myref{namemangling}.)

De ce fait, les références C++ sont bien plus efficientes que les pointeurs usuels.
