\subsubsection{Héritage multiple}


L'héritage multiple est la création d'une classe qui hérite des champs et méthodes
de deux classes ou plus.

Écrivons à nouveau un exemple simple:

\lstinputlisting[style=customc]{\CURPATH/classes/classes3_multiple.cpp}


Compilons-le avec MSVC 2008 avec les options \Ox et \Obzero et regardons le code de \TT{box::dump()},\\
\TT{solid\_object::dump()} et \TT{solid\_box::dump()}:

\lstinputlisting[caption=MSVC 2008 \Optimizing\ /Ob0,style=customasmx86]{\CURPATH/classes/classes3_1.asm}

\lstinputlisting[caption=MSVC 2008 \Optimizing\ /Ob0,style=customasmx86]{\CURPATH/classes/classes3_2.asm}

\lstinputlisting[caption=MSVC 2008 \Optimizing\ /Ob0,style=customasmx86]{\CURPATH/classes/classes3_3.asm}


Donc, la disposition de la mémoire pour ces trois classes est:

Classe \emph{box}:

\begin{center}
\begin{tabular}{ | l | l | }
\hline
  \tableheader{} \\
\hline
  +0x0 & width \\
\hline
  +0x4 & height \\
\hline
  +0x8 & depth \\
\hline
\end{tabular}
\end{center}

Classe \emph{solid\_object}:

\begin{center}
\begin{tabular}{ | l | l | }
\hline
  \tableheader{} \\
\hline
  +0x0 & density \\
\hline
\end{tabular}
\end{center}


On peut dire que la disposition de la mémoire de la classe \emph{solid\_box} est \emph{unifiée}:

Classe \emph{solid\_box}:

\begin{center}
\begin{tabular}{ | l | l | }
\hline
  \tableheader{} \\
\hline
  +0x0 & width \\
\hline
  +0x4 & height \\
\hline
  +0x8 & depth \\
\hline
  +0xC & density \\
\hline
\end{tabular}
\end{center}


Le code des méthodes \TT{box::get\_volume()} et \TT{solid\_object::get\_density()} est trivial:

\lstinputlisting[caption=MSVC 2008 \Optimizing\ /Ob0,style=customasmx86]{\CURPATH/classes/classes3_4.asm}

\lstinputlisting[caption=MSVC 2008 \Optimizing\ /Ob0,style=customasmx86]{\CURPATH/classes/classes3_5.asm}


Mais le code de la méthode \TT{solid\_box::get\_weight()} est bien plus intéressant:

\lstinputlisting[caption=MSVC 2008 \Optimizing\ /Ob0,style=customasmx86]{\CURPATH/classes/classes3_6.asm}


\TT{get\_weight()} appelle juste deux méthodes, mais pour \TT{get\_volume()} il passe
simplement un pointeur sur \TT{this}, et pour \TT{get\_density()} il passe un pointeur
sur \TT{this} incrémenté de 12 (ou \TT{0xC}) octets, et ici, dans la disposition
de la mémoire de la classe \TT{solid\_box}, les champs de la classe \TT{solid\_object}
commencent.


Ainsi, la méthode \TT{solid\_object::get\_density()} croira qu'elle traite une classe
\TT{solid\_object} usuelle, et la méthode \TT{box::get\_volume()} fonctionnera avec
ses trois champs, croyant que c'est juste un objet usuel de la classe \TT{box}.


Ainsi, on peut dire, un objet d'une classe, qui hérite de plusieurs autres classes,
est représenté en mémoire comme une classe \emph{unifiée}, qui contient tous les champs
hérités.
Et chaque méthode héritée est appelée avec un pointeur sur la partie correspondante
de la structure.

