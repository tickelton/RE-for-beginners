\subsection{x86}

\dots est compilée de manière très prédictive:

\lstinputlisting[caption=MSVC,style=customasmx86]{\CURPATH/11_1_msvc_FR.asm}

\myindex{x86!\Instructions!IDIV}

\IDIV divise le nombre 64-bit stocké dans la paire de registres \TT{EDX:EAX} par
la valeur dans \ECX.
Comme résultat, \EAX contiendra le \gls{quotient}, et \EDX --- le reste.
Le résultat de la fonction \ttf est renvoyé dans le registre \EAX, donc la valeur
n'est pas déplacée après la division, elle est déjà à la bonne place.

Puisque \IDIV utilise la valeur dans la paire de registres \TT{EDX:EAX}, l'instruction
\TT{CDQ} (avant \IDIV) étend la valeur dans \EAX en une valeur 64-bit, en tenant
compte du signe, tout comme \MOVSX le fait.

Si nous mettons l'optimisation (\Ox), nous obtenons:

\lstinputlisting[caption=MSVC \Optimizing,style=customasmx86]{\CURPATH/11_1_msvc_Ox.asm}

Ceci est la division par la multiplication. L'opération de multiplication est bien
plus rapide.
Et il possible d'utiliser cette astuce
\footnote{En savoir plus sur la division par la multiplication dans \InSqBrackets{\HenryWarren 10-3}}
pour produire du code effectivement équivalent et plus rapide.

Ceci est aussi appelé \q{strength reduction} dans les optimisations du compilateur.

GCC 4.4.1 génère presque le même code, même sans flag d'optimisation, tout comme
MSVC avec l'optimisation:

\lstinputlisting[caption=GCC 4.4.1 \NonOptimizing,style=customasmx86]{\CURPATH/11_2_gcc.asm}
