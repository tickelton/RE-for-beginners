\subsection{Трюк с переключением значений}

... найдено в Jorg Arndt --- Matters Computational / Ideas, Algorithms, Source Code
\footnote{\url{https://www.jjj.de/fxt/fxtbook.pdf}}.

Вам нужно, чтобы некая переменная переключалась между 123 и 456.
Вы напишете что-то вроде:

\begin{lstlisting}
if (a==123)
    a=456;
else
    a=123;
\end{lstlisting}

Оказывается, это можно делать одной операцией:

\lstinputlisting[style=customc]{fundamentals/XOR_switch.c}

Это работает потому что $123 \oplus 123 \oplus 456=0 \oplus 456=456$ и
$456 \oplus 123 \oplus 456=456 \oplus 456 \oplus 123=0 \oplus 123=123$.

Стоит так писать или нет, а особенно думая о читабельность кода, можно спорить.
Но это еще одна хорошая иллюстрация свойств исключающего ИЛИ.

