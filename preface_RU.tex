\section*{Предисловие}

У термина \q{\gls{reverse engineering}} несколько популярных значений:
1) исследование скомпилированных
программ; 2) сканирование трехмерной модели для последующего копирования;
3) восстановление структуры СУБД. Настоящий сборник заметок
связан с первым значением.

\subsection*{Желательные знания перед началом чтения}

Очень желательно базовое знание \ac{PL} Си.
Рекомендуемые материалы: \myref{CCppBooks}.

\subsection*{Упражнения и задачи}

\dots 
все перемещены на отдельный сайт: \url{http://challenges.re}.

\subsection*{Об авторе}
\begin{tabularx}{\textwidth}{ l X }

\raisebox{-\totalheight}{
\includegraphics[scale=0.60]{Dennis_Yurichev.jpg}
}

&
Денис Юричев~--- опытный reverse engineer и программист.
С ним можно контактировать по емейлу: \textbf{\EMAIL{}}. % или по Skype: \textbf{dennis.yurichev}.

% FIXME: no link. \tablefootnote doesn't work
\end{tabularx}

% subsections:
\subsection*{%
	\RU{Отзывы об этой книге}%
	\EN{Praise for this book}%
	\ES{Elogios para}% TBT
	\PTBRph{}%
	\DE{Lob für}% TBT
	\PLph{}%
	\IT{Elogi per questo libro}%
	\FR{Éloges de ce livre}%
	\JA{賛辞}% TBT
}

\begin{itemize}

\item \q{Now that Dennis Yurichev has made this book free (libre), it is a contribution to the world of free knowledge and free education.} Richard M. Stallman,
\EN{GNU founder, software freedom activist.}
\RU{Основатель GNU, активист в области свободного ПО.}
\FR{Fondateur de GNU, militant pour la liberté des logiciels}
\IT{Fondatore di GNU, attivista del software libero.}
\JA{GNU創設者、自由なソフトウェアの活動家}

% expanded URLs to make it more robust for printouts. In electronic editions people will click anyway, so tracking will keep working
\item \q{It's very well done .. and for free .. amazing.}\footnote{\href{http://go.yurichev.com/17095}{twitter.com/daniel\_bilar/status/436578617221742593}} Daniel Bilar, Siege Technologies, LLC.

\item \q{... excellent and free}\footnote{\href{http://go.yurichev.com/17096}{twitter.com/petefinnigan/status/400551705797869568}} Pete Finnigan,%
	\RU{гуру по безопасности}%
	\ES{gur\'u de seguridad en}%
	\PTBRph{}%
	\DE{Security-Guru}%
	\PLph{}%
	\IT{Guru della sicurezza}
	\FR{gourou de la sécurité}
    \JA{セキュリティグル}
\oracle
	\EN{security guru}.

\item \q{... [the] book is interesting, great job!} Michael Sikorski,
	\RU{автор книги}%
	\EN{author of}%
	\ES{autor de}%
	\PTBRph{}%
	\DE{Autor von}%
	\PLph{}%
	\IT{Autore di}
	\FR{auteur de}
	\JA{以下の著作の著者です}
\emph{Practical Malware Analysis: The Hands-On Guide to Dissecting Malicious Software}.

\item \q{... my compliments for the very nice tutorial!} Herbert Bos,
	\RU{профессор университета}%
	\EN{full professor at the}%
	\ES{catedr\'atico de tiempo completo en la}%
	\PTBRph{}%
	\DE{Professor an der}%
	\PLph{}%
	\IT{professore presso la}
	\FR{professeur à temps complet à}
	\JA{教授}
Vrije Universiteit Amsterdam,
	\RU{соавтор}%
	\EN{co-author of}%
	\ES{coautor de}%
	\PTBRph{}%
	\DE{Co-Autor von}%
	\PLph{}%
	\IT{coautore di}
	\FR{co-auteur de}
	\JA{共著者}
\emph{Modern Operating Systems (4th Edition)}.

\item \q{... It is amazing and unbelievable.} Luis Rocha, CISSP / ISSAP, Technical Manager, Network \& Information Security at Verizon Business.

\item \q{Thanks for the great work and your book.} Joris van de Vis,
	\RU{специалист по}%
	\ES{especialista en}%
	\PTBRph{}%
	\DE{Spezialist bei}%
	\PLph{}%
	\IT{specialista di}
	\FR{spécialiste}
SAP Netweaver \& Security
	\EN{specialist}.
	\JA{スペシャリスト}

\item \q{... [a] reasonable intro to some of the techniques.}\footnote{\href{http://go.yurichev.com/17099}{reddit}} Mike Stay,
	\RU{преподаватель в}%
	\EN{teacher at the}%
	\ES{profesor en el}%
	\PTBRph{}%
	\DE{Professor an der}%
	\PLph{}%
	\IT{docente presso}
	\FR{professeur au}
	\JA{教授}
Federal Law Enforcement Training Center, Georgia, US.

\item \q{I love this book! I have several students reading it at the moment, [and] plan to use it in graduate course.}\footnote{\href{http://go.yurichev.com/17097}{twitter.com/sergeybratus/status/505590326560833536}}
	\RU{Сергей Братусь}%
	\EN{Sergey Bratus}%
	\ES{Sergey Bratus}%
	\PTBRph{}%
	\DE{Sergey Bratus}%
	\PLph{}%
	\IT{Sergey Bratus}
	\FR{Sergey Bratus},
	\JA{セルゲイブラウス}
Research Assistant Professor
	\RU{в отделе Computer Science в}%
	\EN{at the Computer Science Department at}%
	\ES{en el Departamento de Ciencias de la Computaci\'on en}%
	\PTBRph{}%
	\DE{an der Fakultät für Computer Science}
	\PLph{}%
	\IT{presso il dipartimento di Informatica del}
	\FR{dans le Département Informatique du}
Dartmouth College
	\JA{コンピュータサイエンス学部}

\item \q{Dennis @Yurichev has published an impressive (and free!) book on reverse engineering}\footnote{\href{http://go.yurichev.com/17098}{twitter.com/TanelPoder/status/524668104065159169}} Tanel Poder,
	\RU{эксперт по настройке производительности Oracle RDBMS}%
	\EN{Oracle RDBMS performance tuning expert}%
	\ES{experto en afinaci\'on de rendimiento de Oracle RDBMS}%
	\PTBRph{}%
	\DE{Oracle RDBMS Performacence-Tuning Experte}%
	\PLph{}
	\IT{esperto di ottimizzazione di Oracle RDBMS}
	\FR{expert en optimisation des performances Oracle RDBMS}
	\JA{オラクルRDBMSパフォーマンスチューニングエキスパート}.

\item \q{This book is a kind of Wikipedia to beginners...} Archer, Chinese Translator, IT Security Researcher.

\RU{\item \q{Прочел Вашу книгу~--- отличная работа, рекомендую на своих курсах студентам
в качестве учебного пособия}. Николай Ильин, преподаватель в ФТИ НТУУ \q{КПИ} и DefCon-UA}

\item \q{[A] first-class reference for people wanting to learn reverse engineering. And it's free for all.} Mikko Hyppönen, F-Secure.

\end{itemize}

\input{thanks}
\input{FAQ_RU}

\subsection*{Как научиться программированию}

Многие люди спрашивают об этом.

Легких путей нет, но есть пути быстрые и эффективные.

Из моего опыта, это просто: решать задачи из:

\begin{itemize}
\item \KRBook
\item Харольд Абельсон, Джеральд Сассман -- Структура и интерпретация компьютерных программ
\item \TAOCP
\item Книги Никласа Вирта
\item \RobPikePractice
\end{itemize}

... на чистом Си и Лиспе.
В будущем, возможно, вы не будете использовать эти языки вообще.
Большинство коммерческих программистов не используют. Но опыт программирования на Си и Лиспе помогает очень сильно в 
долгосрочной перспективе.

Также, вы можете не читать сами книги, просто листайте, тогда, когда вы чувствуете, что вы не можете чего-то
понять для решения задачи.

В лучшем случае, это занимает годы, или всю жизнь, но это все же намного быстрее, чем метаться между очередными модными
штуками.

Успех этих книг вероятно связан с тем, что их авторы -- преподаватели, и весь материал был в начале отточен на студентах.

Насчет Листа, я бы рекомендовал Racket (диалект Scheme). Но это дело вкуса, так или иначе.

Некоторые люди говорят, что понимание ассемблера очень помогает, даже если вы не будете использовать его.
Это верно.
Но это уже путь для наиболее посвященных гиков, и для начала это можно отложить на будущее.

Также, очень частая проблема самоучек (включая автора этих строк), это то, что они слишком часто хватаются
за слишком трудные проблемы, пропуская более легкие.
Это большая ошибка.
Сравните со спортом или музыкой -- никто не начинает с попыток поднимать сразу 100 кг,
или же сразу пытаться играть Капризы Паганини.
Я бы сказал так -- браться за задачу можно тогда, когда вы зараннее, в уме, можете примерно прикинуть решение.

\begin{framed}
\begin{quotation}                                                                                                               I think the art of doing research consists largely of asking questions,
and sometimes answering them. Learn how to repeatedly pose miniquestions
that represent special cases of the big questions you are hoping to solve.

When you begin to explore some area, you take baby steps at first, building
intuition about that territory. Play with many small examples, trying to
get a complete understanding of particular parts of the general situation.

In that way you learn many properties that are true and many properties
that are false. That gives guidance about directions that are fruitful
versus directions to avoid.

Eventually your brain will have learned how to take larger and larger steps.
And shazam, you’ll be ready to take some giant steps and solve the big problem.

But don’t stop there! At this point you’ll be one of very few people in the
world who have ever understood your problem area so well. It will therefore
be your responsibility to discover what else is true, in the neighborhood
of that problem, using the same or similar methods to what your brain
can now envision. Take your results to their “natural boundary” (in a sense
analogous to the natural boundary where a function of a complex variable
ceases to be analytic).

My little book Surreal Numbers provides an authentic example of research
as it is happening. The characters in that story make false starts and
useful discoveries in exactly the same order as I myself made those false starts
and useful discoveries, when I first studied John Conway’s fascinating
axioms about number systems — his amazingly simple axioms that go
significantly beyond real-valued numbers.

(One of the characters in that book tends to succeed or fail by brute force
and patience; the other is more introspective, and able to see a bigger
picture. Both of them represent aspects of my own activities while doing
research. With that book I hoped to teach research skills “by osmosis”,
as readers observe a detailed case study.)

Surreal Numbers deals with a purely mathematical topic, not especially close
to computer science; it features algebra and logic, not algorithms.
When algorithms become part of the research, a beautiful new dimension
also comes into play: Algorithms can be implemented on computers!

I strongly recommend that you look for every opportunity to write programs
that carry out all or a part of whatever algorithms relate to your research.
In my experience the very act of writing such a program never fails to
deepen my understanding of the problem area.
\end{quotation}
\end{framed}
( Дональд Э. Кнут -- \url{https://theorydish.blog/2018/02/01/donald-knuth-on-doing-research/} )

Удачи!

\subsection*{О переводе на корейский язык}

В январе 2015, издательство Acorn в Южной Корее сделало много работы в переводе 
и издании моей книги (по состоянию на август 2014) на корейский язык.
Она теперь доступна на \href{http://go.yurichev.com/17343}{их сайте}.

\iffalse
\begin{figure}[H]
\centering
\includegraphics[scale=0.3]{acorn_cover.jpg}
\end{figure}
\fi

Переводил Byungho Min (\href{http://go.yurichev.com/17344}{twitter/tais9}).
Обложку нарисовал мой хороший знакомый художник Андрей Нечаевский
\href{http://go.yurichev.com/17023}{facebook/andydinka}.
Они также имеют права на издание книги на корейском языке.
Так что если вы хотите иметь \IT{настоящую} книгу на полке на корейском языке и
хотите поддержать мою работу, вы можете купить её.

\subsection*{О переводе на персидский язык (фарси)}

В 2016 году книга была переведена Mohsen Mostafa Jokar (который также известен иранскому сообществу по переводу руководства Radare\footnote{\url{http://rada.re/get/radare2book-persian.pdf}}).
Книга доступна на сайте издательства\footnote{\url{http://goo.gl/2Tzx0H}} (Pendare Pars).

Первые 40 страниц: \url{https://beginners.re/farsi.pdf}.

Регистрация книги в Национальной Библиотеке Ирана: \url{http://opac.nlai.ir/opac-prod/bibliographic/4473995}.

\subsection*{О переводе на китайский язык}

В апреле 2017, перевод на китайский был закончен китайским издательством PTPress. Они также имеют права на издание книги на китайском языке.

Она доступна для заказа здесь: \url{http://www.epubit.com.cn/book/details/4174}. Что-то вроде рецензии и история о переводе: \url{http://www.cptoday.cn/news/detail/3155}.

Основным переводчиком был Archer, перед которым я теперь в долгу.
Он был крайне дотошным (в хорошем смысле) и сообщил о большинстве известных ошибок и баг, что крайне важно для литературы вроде этой книги.
Я буду рекомендовать его услуги всем остальным авторам!

Ребята из \href{http://www.antiy.net/}{Antiy Labs} также помогли с переводом. \href{http://www.epubit.com.cn/book/onlinechapter/51413}{Здесь предисловие} написанное ими.

