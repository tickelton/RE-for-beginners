\section*{Préface}

\subsection*{C'est quoi ces deux titres?}
\label{TwoTitles}

Le livre a été appelé ``Reverse Engineering for Beginners'' en 2014-2018, mais j'ai
toujours suspecté que ça rendait son audience trop réduite.

Les gens de l'infosec connaissent le ``reverse engineering'', mais j'ai rarement
entendu le mot ``assembleur'' de leur part.

De même, le terme ``reverse engineering'' est quelque peu cryptique pour une audience
générale de programmeurs, mais qui ont des connaissances à propos de l'``assembleur''.

En juillet 2018, à titre d'expérience, j'ai changé le titre en ``Assembly Language for Beginners''
et publié le lien sur le site Hacker News\footnote{\url{https://news.ycombinator.com/item?id=17549050}},
et le livre a été plutôt bien accueilli.

Donc, c'est ainsi que le livre a maintenant deux titres.

Toutefois, j'ai changé le second titre à ``Understanding Assembly Language'', car
quelqu'un a déjà écrit le livre ``Assembly Language for Beginners''.
De même, des gens disent que ``for Beginners'' sonne sarcastique pour un livre de
\textasciitilde{}1000 pages.

Les deux livres diffèrent seulement par le titre, le nom du fichier (UAL-XX.pdf versus
RE4B-XX.pdf), l'URL et quelques-une des première pages.

\subsection*{À propos de la rétro-ingénierie}

Il existe plusieurs définitions pour l'expression \q{ingénierie inverse ou rétro-ingénierie \gls{reverse engineering}} :

1) L'ingénierie inverse de logiciels : examiner des programmes compilés;

2) Le balayage des structures en 3D et la manipulation numérique nécessaire afin de les reproduire;

3) Recréer une structure de base de données.

Ce livre concerne la première définition.

\subsection*{Prérequis}

Connaissance basique du C \ac{PL}.
Il est recommandé de lire: \myref{CCppBooks}.

\subsection*{Exercices et tâches}

\dots
ont été déplacés sur un site différent : \url{http://challenges.re}.

\subsection*{A propos de l'auteur}
\begin{tabularx}{\textwidth}{ l X }

\raisebox{-\totalheight}{
\includegraphics[scale=0.60]{Dennis_Yurichev.jpg}
}

&
Dennis Yurichev est un ingénieur expérimenté en rétro-ingénierie et un programmeur.
Il peut être contacté par email : \textbf{\EMAIL{}}.

% FIXME: no link. \tablefootnote doesn't work
\end{tabularx}

% subsections:
\subsection*{%
	\RU{Отзывы об этой книге}%
	\EN{Praise for this book}%
	\ES{Elogios para}% TBT
	\PTBRph{}%
	\DE{Lob für}% TBT
	\PLph{}%
	\IT{Elogi per questo libro}%
	\FR{Éloges de ce livre}%
	\JA{賛辞}% TBT
}

\begin{itemize}

\item \q{Now that Dennis Yurichev has made this book free (libre), it is a contribution to the world of free knowledge and free education.} Richard M. Stallman,
\EN{GNU founder, software freedom activist.}
\RU{Основатель GNU, активист в области свободного ПО.}
\FR{Fondateur de GNU, militant pour la liberté des logiciels}
\IT{Fondatore di GNU, attivista del software libero.}
\JA{GNU創設者、自由なソフトウェアの活動家}

% expanded URLs to make it more robust for printouts. In electronic editions people will click anyway, so tracking will keep working
\item \q{It's very well done .. and for free .. amazing.}\footnote{\href{http://go.yurichev.com/17095}{twitter.com/daniel\_bilar/status/436578617221742593}} Daniel Bilar, Siege Technologies, LLC.

\item \q{... excellent and free}\footnote{\href{http://go.yurichev.com/17096}{twitter.com/petefinnigan/status/400551705797869568}} Pete Finnigan,%
	\RU{гуру по безопасности}%
	\ES{gur\'u de seguridad en}%
	\PTBRph{}%
	\DE{Security-Guru}%
	\PLph{}%
	\IT{Guru della sicurezza}
	\FR{gourou de la sécurité}
    \JA{セキュリティグル}
\oracle
	\EN{security guru}.

\item \q{... [the] book is interesting, great job!} Michael Sikorski,
	\RU{автор книги}%
	\EN{author of}%
	\ES{autor de}%
	\PTBRph{}%
	\DE{Autor von}%
	\PLph{}%
	\IT{Autore di}
	\FR{auteur de}
	\JA{以下の著作の著者です}
\emph{Practical Malware Analysis: The Hands-On Guide to Dissecting Malicious Software}.

\item \q{... my compliments for the very nice tutorial!} Herbert Bos,
	\RU{профессор университета}%
	\EN{full professor at the}%
	\ES{catedr\'atico de tiempo completo en la}%
	\PTBRph{}%
	\DE{Professor an der}%
	\PLph{}%
	\IT{professore presso la}
	\FR{professeur à temps complet à}
	\JA{教授}
Vrije Universiteit Amsterdam,
	\RU{соавтор}%
	\EN{co-author of}%
	\ES{coautor de}%
	\PTBRph{}%
	\DE{Co-Autor von}%
	\PLph{}%
	\IT{coautore di}
	\FR{co-auteur de}
	\JA{共著者}
\emph{Modern Operating Systems (4th Edition)}.

\item \q{... It is amazing and unbelievable.} Luis Rocha, CISSP / ISSAP, Technical Manager, Network \& Information Security at Verizon Business.

\item \q{Thanks for the great work and your book.} Joris van de Vis,
	\RU{специалист по}%
	\ES{especialista en}%
	\PTBRph{}%
	\DE{Spezialist bei}%
	\PLph{}%
	\IT{specialista di}
	\FR{spécialiste}
SAP Netweaver \& Security
	\EN{specialist}.
	\JA{スペシャリスト}

\item \q{... [a] reasonable intro to some of the techniques.}\footnote{\href{http://go.yurichev.com/17099}{reddit}} Mike Stay,
	\RU{преподаватель в}%
	\EN{teacher at the}%
	\ES{profesor en el}%
	\PTBRph{}%
	\DE{Professor an der}%
	\PLph{}%
	\IT{docente presso}
	\FR{professeur au}
	\JA{教授}
Federal Law Enforcement Training Center, Georgia, US.

\item \q{I love this book! I have several students reading it at the moment, [and] plan to use it in graduate course.}\footnote{\href{http://go.yurichev.com/17097}{twitter.com/sergeybratus/status/505590326560833536}}
	\RU{Сергей Братусь}%
	\EN{Sergey Bratus}%
	\ES{Sergey Bratus}%
	\PTBRph{}%
	\DE{Sergey Bratus}%
	\PLph{}%
	\IT{Sergey Bratus}
	\FR{Sergey Bratus},
	\JA{セルゲイブラウス}
Research Assistant Professor
	\RU{в отделе Computer Science в}%
	\EN{at the Computer Science Department at}%
	\ES{en el Departamento de Ciencias de la Computaci\'on en}%
	\PTBRph{}%
	\DE{an der Fakultät für Computer Science}
	\PLph{}%
	\IT{presso il dipartimento di Informatica del}
	\FR{dans le Département Informatique du}
Dartmouth College
	\JA{コンピュータサイエンス学部}

\item \q{Dennis @Yurichev has published an impressive (and free!) book on reverse engineering}\footnote{\href{http://go.yurichev.com/17098}{twitter.com/TanelPoder/status/524668104065159169}} Tanel Poder,
	\RU{эксперт по настройке производительности Oracle RDBMS}%
	\EN{Oracle RDBMS performance tuning expert}%
	\ES{experto en afinaci\'on de rendimiento de Oracle RDBMS}%
	\PTBRph{}%
	\DE{Oracle RDBMS Performacence-Tuning Experte}%
	\PLph{}
	\IT{esperto di ottimizzazione di Oracle RDBMS}
	\FR{expert en optimisation des performances Oracle RDBMS}
	\JA{オラクルRDBMSパフォーマンスチューニングエキスパート}.

\item \q{This book is a kind of Wikipedia to beginners...} Archer, Chinese Translator, IT Security Researcher.

\RU{\item \q{Прочел Вашу книгу~--- отличная работа, рекомендую на своих курсах студентам
в качестве учебного пособия}. Николай Ильин, преподаватель в ФТИ НТУУ \q{КПИ} и DefCon-UA}

\item \q{[A] first-class reference for people wanting to learn reverse engineering. And it's free for all.} Mikko Hyppönen, F-Secure.

\end{itemize}

\input{thanks}
\input{FAQ_FR}

\subsection*{Comment apprendre à programmer}

Beaucoup de gens me l'ont demandé.

Il n'y a pas de ``voie royale'', mais il y a quelques chemins efficaces.

De ma propre expérience, ceci est juste: résoudre les exercices de:

\begin{itemize}
\item \KRBook
\item Harold Abelson, Gerald Jay Sussman, Julie Sussman -- Structure and Interpretation of Computer Programs
\item \TAOCP
\item Niklaus Wirth's books
\item \RobPikePractice
\end{itemize}

... en pur C et LISP.
Vous pourriez ne jamais utiliser du tout ces langages de programmation dans le futur.
Presqu'aucun des programmeurs de métier le les utilisent. Mais l'expérience du développement
en C et en LISP aide énormément sur la long terme.

Vous pouvez également passer leur lecture elle-même, parcourez-les seulement lorsque
vous sentez que vous devez comprendre quelque chose que vous ne comprenez pas dans
l'exercice que vous faites.

Ceci peu prendre des années, ou une vie entière, mais c'est toujours plus rapide
que de se précipiter entre les modes.

Le succès de ces livres est sans doute lié au fait que les auteurs sont des enseignants
et tous ce matériel a d'abord été amélioré avec les étudiants.

Pour LISP, je recommande personnellement Racket (dialecte Scheme). Mais c'est une
affaire de goût.

Certaines personnes disent que comprendre le langage d'assemblage est très utile,
même si vous ne l'utiliserez jamais.
Ceci est vrai.
Mais ceci est une voie pour les geeks les plus persévérants, et ça peut être mis de
côté au début.

De même, les autodidactes (incluant l'auteur de ces lignes) ont souvent le problème
de travailler dur sur des problèmes difficiles, passant sur les plus faciles.
Ceci est une grosse erreur.
Comparez avec le sport ou le musique -- personne ne commence à un poids de 100kg,
ou Paganini's Caprices.
Je dirais -- essayez de résoudre un problème si vous pouvez en esquisser la solution
dans votre tête.

\begin{framed}
\begin{quotation}
I think the art of doing research consists largely of asking questions,
and sometimes answering them. Learn how to repeatedly pose miniquestions
that represent special cases of the big questions you are hoping to solve.

When you begin to explore some area, you take baby steps at first, building
intuition about that territory. Play with many small examples, trying to
get a complete understanding of particular parts of the general situation.

In that way you learn many properties that are true and many properties
that are false. That gives guidance about directions that are fruitful
versus directions to avoid.

Eventually your brain will have learned how to take larger and larger steps.
And shazam, you’ll be ready to take some giant steps and solve the big problem.

But don’t stop there! At this point you’ll be one of very few people in the
world who have ever understood your problem area so well. It will therefore
be your responsibility to discover what else is true, in the neighborhood
of that problem, using the same or similar methods to what your brain
can now envision. Take your results to their “natural boundary” (in a sense
analogous to the natural boundary where a function of a complex variable
ceases to be analytic).

My little book Surreal Numbers provides an authentic example of research
as it is happening. The characters in that story make false starts and
useful discoveries in exactly the same order as I myself made those false starts
and useful discoveries, when I first studied John Conway’s fascinating
axioms about number systems — his amazingly simple axioms that go
significantly beyond real-valued numbers.

(One of the characters in that book tends to succeed or fail by brute force
and patience; the other is more introspective, and able to see a bigger
picture. Both of them represent aspects of my own activities while doing
research. With that book I hoped to teach research skills “by osmosis”,
as readers observe a detailed case study.)

Surreal Numbers deals with a purely mathematical topic, not especially close
to computer science; it features algebra and logic, not algorithms.
When algorithms become part of the research, a beautiful new dimension
also comes into play: Algorithms can be implemented on computers!

I strongly recommend that you look for every opportunity to write programs
that carry out all or a part of whatever algorithms relate to your research.
In my experience the very act of writing such a program never fails to
deepen my understanding of the problem area.
\end{quotation}
\end{framed}

( Donald E. Knuth -- \url{https://theorydish.blog/2018/02/01/donald-knuth-on-doing-research/} )

Bonne chance!

\subsection*{À propos de la traduction en Coréen}

En Janvier 2015, la maison d'édition Acorn (\href{http://www.acornpub.co.kr}{www.acornpub.co.kr}) en Corée du Sud a réalisé un énorme travail en traduisant et en publiant mon livre (dans son état en Août 2014) en Coréen.

Il est désormais disponible sur \href{http://go.yurichev.com/17343}{leur site web}.

\iffalse
\begin{figure}[H]
\centering
\includegraphics[scale=0.3]{acorn_cover.jpg}
\end{figure}
\fi

Le traducteur est Byungho Min (\href{http://go.yurichev.com/17344}{twitter/tais9}).
L'illustration de couverture a été réalisée l'artiste, Andy Nechaevsky, un ami de l'auteur:
\href{http://go.yurichev.com/17023}{facebook/andydinka}.
Ils détiennent également les droits d'auteurs sur la traduction coréenne.

Donc si vous souhaitez avoir un livre \emph{réel} en coréen sur votre étagère et que vous souhaitez soutenir ce travail, il est désormais disponible à l'achat.

\subsection*{Á propos de la traduction en Farsi/Perse}

En 2016, ce livre a été traduit par Mohsen Mostafa Jokar (qui est aussi connu dans
la communauté iranienne pour sa traduction du manuel de Radare\footnote{\url{http://rada.re/get/radare2book-persian.pdf}}).
Il est disponible sur le site web de l'éditeur\footnote{\url{http://goo.gl/2Tzx0H}}
(Pendare Pars).

Extrait de 40 pages: \url{https://beginners.re/farsi.pdf}.

Enregistrement du livre à la Bibliothèque Nationale d'Iran: \url{http://opac.nlai.ir/opac-prod/bibliographic/4473995}.

\subsection*{Á propos de la traduction en Chinois}

En avril 2017, la traduction en Chinois a été terminée par Chinese PTPress. Ils sont
également les détenteurs des droits de la traduction en Chinois.

La version chinoise est disponible à l'achat ici: \url{http://www.epubit.com.cn/book/details/4174}.
Une revue partielle et l'historique de la traduction peut être trouvé ici: \url{http://www.cptoday.cn/news/detail/3155}.


Le traducteur principal est Archer, à qui je dois beaucoup. Il a été très méticuleux
(dans le bon sens du terme) et a signalé la plupart des erreurs et bugs connus, ce
qui est très important dans le genre de littérature de ce livre.
Je recommanderais ses services à tout autre auteur!

Les gens de \href{http://www.antiy.net/}{Antiy Labs} ont aussi aidé pour la traduction.
\href{http://www.epubit.com.cn/book/onlinechapter/51413}{Voici la préface} écrite par eux.

