\section*{Preface}

There are several popular meanings of the term \q{\gls{reverse engineering}}:

1) The reverse engineering of software; researching compiled programs

2) The scanning of 3D structures and the subsequent digital manipulation required in order to duplicate them

3) Recreating \ac{DBMS} structure

This book is about the first meaning.

\subsection*{Prerequisites}

Basic knowledge of the C \ac{PL}.
Recommended reading: \myref{CCppBooks}.

\subsection*{Exercises and tasks}

\dots
can be found at: \url{http://challenges.re}.

\subsection*{About the author}
\begin{tabularx}{\textwidth}{ l X }

\raisebox{-\totalheight}{
\includegraphics[scale=0.60]{Dennis_Yurichev.jpg}
}

&
Dennis Yurichev is an experienced reverse engineer and programmer.
He can be contacted by email: \textbf{\EMAIL{}}. % or Skype: \textbf{dennis.yurichev}.

% FIXME: no link. \tablefootnote doesn't work
\end{tabularx}

% subsections:
\subsection*{%
	\RU{Отзывы об этой книге}%
	\EN{Praise for this book}%
	\ES{Elogios para}% TBT
	\PTBRph{}%
	\DE{Lob für}% TBT
	\PLph{}%
	\IT{Elogi per questo libro}%
	\FR{Éloges de ce livre}%
	\JA{賛辞}% TBT
}

\begin{itemize}

\item \q{Now that Dennis Yurichev has made this book free (libre), it is a contribution to the world of free knowledge and free education.} Richard M. Stallman,
\EN{GNU founder, software freedom activist.}
\RU{Основатель GNU, активист в области свободного ПО.}
\FR{Fondateur de GNU, militant pour la liberté des logiciels}
\IT{Fondatore di GNU, attivista del software libero.}
\JA{GNU創設者、自由なソフトウェアの活動家}

% expanded URLs to make it more robust for printouts. In electronic editions people will click anyway, so tracking will keep working
\item \q{It's very well done .. and for free .. amazing.}\footnote{\href{http://go.yurichev.com/17095}{twitter.com/daniel\_bilar/status/436578617221742593}} Daniel Bilar, Siege Technologies, LLC.

\item \q{... excellent and free}\footnote{\href{http://go.yurichev.com/17096}{twitter.com/petefinnigan/status/400551705797869568}} Pete Finnigan,%
	\RU{гуру по безопасности}%
	\ES{gur\'u de seguridad en}%
	\PTBRph{}%
	\DE{Security-Guru}%
	\PLph{}%
	\IT{Guru della sicurezza}
	\FR{gourou de la sécurité}
    \JA{セキュリティグル}
\oracle
	\EN{security guru}.

\item \q{... [the] book is interesting, great job!} Michael Sikorski,
	\RU{автор книги}%
	\EN{author of}%
	\ES{autor de}%
	\PTBRph{}%
	\DE{Autor von}%
	\PLph{}%
	\IT{Autore di}
	\FR{auteur de}
	\JA{以下の著作の著者です}
\emph{Practical Malware Analysis: The Hands-On Guide to Dissecting Malicious Software}.

\item \q{... my compliments for the very nice tutorial!} Herbert Bos,
	\RU{профессор университета}%
	\EN{full professor at the}%
	\ES{catedr\'atico de tiempo completo en la}%
	\PTBRph{}%
	\DE{Professor an der}%
	\PLph{}%
	\IT{professore presso la}
	\FR{professeur à temps complet à}
	\JA{教授}
Vrije Universiteit Amsterdam,
	\RU{соавтор}%
	\EN{co-author of}%
	\ES{coautor de}%
	\PTBRph{}%
	\DE{Co-Autor von}%
	\PLph{}%
	\IT{coautore di}
	\FR{co-auteur de}
	\JA{共著者}
\emph{Modern Operating Systems (4th Edition)}.

\item \q{... It is amazing and unbelievable.} Luis Rocha, CISSP / ISSAP, Technical Manager, Network \& Information Security at Verizon Business.

\item \q{Thanks for the great work and your book.} Joris van de Vis,
	\RU{специалист по}%
	\ES{especialista en}%
	\PTBRph{}%
	\DE{Spezialist bei}%
	\PLph{}%
	\IT{specialista di}
	\FR{spécialiste}
SAP Netweaver \& Security
	\EN{specialist}.
	\JA{スペシャリスト}

\item \q{... [a] reasonable intro to some of the techniques.}\footnote{\href{http://go.yurichev.com/17099}{reddit}} Mike Stay,
	\RU{преподаватель в}%
	\EN{teacher at the}%
	\ES{profesor en el}%
	\PTBRph{}%
	\DE{Professor an der}%
	\PLph{}%
	\IT{docente presso}
	\FR{professeur au}
	\JA{教授}
Federal Law Enforcement Training Center, Georgia, US.

\item \q{I love this book! I have several students reading it at the moment, [and] plan to use it in graduate course.}\footnote{\href{http://go.yurichev.com/17097}{twitter.com/sergeybratus/status/505590326560833536}}
	\RU{Сергей Братусь}%
	\EN{Sergey Bratus}%
	\ES{Sergey Bratus}%
	\PTBRph{}%
	\DE{Sergey Bratus}%
	\PLph{}%
	\IT{Sergey Bratus}
	\FR{Sergey Bratus},
	\JA{セルゲイブラウス}
Research Assistant Professor
	\RU{в отделе Computer Science в}%
	\EN{at the Computer Science Department at}%
	\ES{en el Departamento de Ciencias de la Computaci\'on en}%
	\PTBRph{}%
	\DE{an der Fakultät für Computer Science}
	\PLph{}%
	\IT{presso il dipartimento di Informatica del}
	\FR{dans le Département Informatique du}
Dartmouth College
	\JA{コンピュータサイエンス学部}

\item \q{Dennis @Yurichev has published an impressive (and free!) book on reverse engineering}\footnote{\href{http://go.yurichev.com/17098}{twitter.com/TanelPoder/status/524668104065159169}} Tanel Poder,
	\RU{эксперт по настройке производительности Oracle RDBMS}%
	\EN{Oracle RDBMS performance tuning expert}%
	\ES{experto en afinaci\'on de rendimiento de Oracle RDBMS}%
	\PTBRph{}%
	\DE{Oracle RDBMS Performacence-Tuning Experte}%
	\PLph{}
	\IT{esperto di ottimizzazione di Oracle RDBMS}
	\FR{expert en optimisation des performances Oracle RDBMS}
	\JA{オラクルRDBMSパフォーマンスチューニングエキスパート}.

\item \q{This book is a kind of Wikipedia to beginners...} Archer, Chinese Translator, IT Security Researcher.

\RU{\item \q{Прочел Вашу книгу~--- отличная работа, рекомендую на своих курсах студентам
в качестве учебного пособия}. Николай Ильин, преподаватель в ФТИ НТУУ \q{КПИ} и DefCon-UA}

\item \q{[A] first-class reference for people wanting to learn reverse engineering. And it's free for all.} Mikko Hyppönen, F-Secure.

\end{itemize}

\input{thanks}
\input{FAQ_EN}

\subsection*{How to learn programming}

Many people keep asking about it.

There is no ``royal road'', but there are quite efficient ways.

From my own experience, this is just: solving exercises from:

\begin{itemize}
\item \KRBook
\item Harold Abelson, Gerald Jay Sussman, Julie Sussman -- Structure and Interpretation of Computer Programs
\item \TAOCP
\item Niklaus Wirth's books
\item \RobPikePractice
\end{itemize}

... in pure C and LISP.
You may never use these programming languages in future at all.
Almost all commercial programmers don’t. But C and LISP coding experience will help enormously in long run.

Also, you can skip reading these books itselves,
just skim them whenever you feel you need to understand something you missing for the exercise you currently solve.

This may take years at best, or a lifetime, but still this is way faster than to rush between fads.

The success of these books probably related to the fact that their authors are teachers
and all this material has been honed on students first.

As of LISP, I personally would recommend Racket (Scheme dialect). But this is matter of taste, anyway.

Some people say assembly language understanding is also very helpful, even if you will never use it.
This is true.
But this is a way for the most dedicated geeks, and it can be postponed at start.

Also, self-taught people (including author of these lines) often has the problem of trying too hard on hard problems,
skipping easy ones.
This is a great mistake.
Compare to sport or music -- no one starts at 100kg weights, or Paganini's Caprices.
I would say -- you can try to tackle a problem if you can outline its solution in your mind.

\begin{framed}
\begin{quotation}                                                                                                               I think the art of doing research consists largely of asking questions,
and sometimes answering them. Learn how to repeatedly pose miniquestions
that represent special cases of the big questions you are hoping to solve.

When you begin to explore some area, you take baby steps at first, building
intuition about that territory. Play with many small examples, trying to
get a complete understanding of particular parts of the general situation.

In that way you learn many properties that are true and many properties
that are false. That gives guidance about directions that are fruitful
versus directions to avoid.

Eventually your brain will have learned how to take larger and larger steps.
And shazam, you’ll be ready to take some giant steps and solve the big problem.

But don’t stop there! At this point you’ll be one of very few people in the
world who have ever understood your problem area so well. It will therefore
be your responsibility to discover what else is true, in the neighborhood
of that problem, using the same or similar methods to what your brain
can now envision. Take your results to their “natural boundary” (in a sense
analogous to the natural boundary where a function of a complex variable
ceases to be analytic).

My little book Surreal Numbers provides an authentic example of research
as it is happening. The characters in that story make false starts and
useful discoveries in exactly the same order as I myself made those false starts
and useful discoveries, when I first studied John Conway’s fascinating
axioms about number systems — his amazingly simple axioms that go
significantly beyond real-valued numbers.

(One of the characters in that book tends to succeed or fail by brute force
and patience; the other is more introspective, and able to see a bigger
picture. Both of them represent aspects of my own activities while doing
research. With that book I hoped to teach research skills “by osmosis”,
as readers observe a detailed case study.)

Surreal Numbers deals with a purely mathematical topic, not especially close
to computer science; it features algebra and logic, not algorithms.
When algorithms become part of the research, a beautiful new dimension
also comes into play: Algorithms can be implemented on computers!

I strongly recommend that you look for every opportunity to write programs
that carry out all or a part of whatever algorithms relate to your research.
In my experience the very act of writing such a program never fails to
deepen my understanding of the problem area.
\end{quotation}
\end{framed}
( Donald E. Knuth -- \url{https://theorydish.blog/2018/02/01/donald-knuth-on-doing-research/} )

Good luck!

\subsection*{About the Korean translation}

In January 2015, the Acorn publishing company (\href{http://www.acornpub.co.kr}{www.acornpub.co.kr}) in South Korea did a huge amount of work in translating and publishing
this book (as it was in August 2014) into Korean.

It's available now at \href{http://go.yurichev.com/17343}{their website}.

\iffalse
\begin{figure}[H]
\centering
\includegraphics[scale=0.3]{acorn_cover.jpg}
\end{figure}
\fi

The translator is Byungho Min (\href{http://go.yurichev.com/17344}{twitter/tais9}).
The cover art was done by the artistic Andy Nechaevsky, a friend of the author:
\href{http://go.yurichev.com/17023}{facebook/andydinka}.
Acorn also holds the copyright to the Korean translation.

So, if you want to have a \IT{real} book on your shelf in Korean and
want to support this work, it is now available for purchase.

\subsection*{About the Persian/Farsi translation}

In 2016 the book was translated by Mohsen Mostafa Jokar (who is also known to Iranian community for his translation of Radare manual\footnote{\url{http://rada.re/get/radare2book-persian.pdf}}).
It is available on the publisher’s website\footnote{\url{http://goo.gl/2Tzx0H}} (Pendare Pars).

Here is a link to a 40-page excerpt: \url{https://beginners.re/farsi.pdf}.

National Library of Iran registration information: \url{http://opac.nlai.ir/opac-prod/bibliographic/4473995}.

\subsection*{About the Chinese translation}

In April 2017, translation to Chinese was completed by Chinese PTPress. They are also the Chinese translation copyright holders.

 The Chinese version is available for order here: \url{http://www.epubit.com.cn/book/details/4174}. A partial review and history behind the translation can be found here: \url{http://www.cptoday.cn/news/detail/3155}.

The principal translator is Archer, to whom the author owes very much. He was extremely meticulous (in a good sense) and reported most of the known mistakes and bugs, which is very important in literature such as this book.
The author would recommend his services to any other author!

The guys from \href{http://www.antiy.net/}{Antiy Labs} has also helped with translation. \href{http://www.epubit.com.cn/book/onlinechapter/51413}{Here is preface} written by them.

I was happy to work with PTPress at start, however, they never paid me.

