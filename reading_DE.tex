\chapter{Bücher / Lesenswerte Blogs}

\mysection{Bücher und andere Materialien}

\subsection{Reverse Engineering}

\begin{itemize}
\item Eldad Eilam, \emph{Reversing: Secrets of Reverse Engineering}, (2005)

\item Bruce Dang, Alexandre Gazet, Elias Bachaalany, Sebastien Josse, \emph{Practical Reverse Engineering: x86, x64, ARM, Windows Kernel, Reversing Tools, and Obfuscation}, (2014)

\item Michael Sikorski, Andrew Honig, \emph{Practical Malware Analysis: The Hands-On Guide to Dissecting Malicious Software}, (2012)

\item Chris Eagle, \emph{IDA Pro Book}, (2011)

\item Reginald Wong, \emph{Mastering Reverse Engineering: Re-engineer your ethical hacking skills}, (2018)

\end{itemize}


Ebenfalls das Buch von Kris Kaspersky.

\subsection{Windows}

\input{Win_reading}

\subsection{\CCpp}

\input{CCppBooks}

\subsection{x86 / x86-64}

\label{x86_manuals}
\begin{itemize}
\item Intel Handbücher\footnote{\AlsoAvailableAs \url{http://www.intel.com/content/www/us/en/processors/architectures-software-developer-manuals.html}}

\item AMD Handbücher\footnote{\AlsoAvailableAs \url{http://developer.amd.com/resources/developer-guides-manuals/}}

\item \AgnerFog{}\footnote{\AlsoAvailableAs \url{http://agner.org/optimize/microarchitecture.pdf}}

\item \AgnerFogCC{}\footnote{\AlsoAvailableAs \url{http://www.agner.org/optimize/calling_conventions.pdf}}

\item \IntelOptimization

\item \AMDOptimization
\end{itemize}

Etwas veraltet aber immer noch interessant zu lesen:

\MAbrash\footnote{\AlsoAvailableAs \url{https://github.com/jagregory/abrash-black-book}}
(Er ist bekannt für seine Arbeiten auf dem Gebiet der Low-Level Optimierung in Projekten wie Windows NT 3.1 und id Quake).

\subsection{ARM}

\begin{itemize}
\item ARM Handbücher\footnote{\AlsoAvailableAs \url{http://infocenter.arm.com/help/index.jsp?topic=/com.arm.doc.subset.architecture.reference/index.html}}

\item \ARMSevenRef

\item \ARMSixFourRefURL

\item \ARMCookBook\footnote{\AlsoAvailableAs \url{http://go.yurichev.com/17273}}
\end{itemize}

\subsection{Assembler}

Richard Blum --- Professional Assembly Language.

\subsection{Java}

\JavaBook.

\subsection{UNIX}

\TAOUP

\subsection{Programmierung Allgemein}

\begin{itemize}

	\item \RobPikePractice
	
	\item \HenryWarren.
	Einige Leute sagen, die Tricks und Hacks aus diesem Buch sind heute nicht mehr relevant und haben die eigentliche Bedeutung für \ac{RISC} \ac{CPU}s, bei denen Verzweigungsbefehle teuer sind.
	Nichtsdestotrotz können diese immens hilfreich sein um Bool'sche Algebra und die damit zusammenhängende Mathematik zu verstehen.
	
	\item (Für Hardcore-Geeks mit Informatik- und / oder Mathematik-Hintergrund) \TAOCP.
	Einge argumentieren, dass es sich für mittelmäßige Programmierer lohnt sich auch diese recht anspruchsvolle Grundlagenliteratur zu erarbeiten.
	Ich würde vorschlagen sie zumindest zu überfliegen, um den Inhalt der Informatik kennenzulernen.
	
\end{itemize}

% subsection:
\subsection{\EN{Cryptography}\ES{Criptograf\'ia}\IT{Crittografia}\RU{Криптография}\FR{Cryptographie}\DE{Kryptografie}\JA{暗号学}}
\label{crypto_books}

\begin{itemize}
\item \Schneier{}

\item (Free) lvh, \emph{Crypto 101}\footnote{\AlsoAvailableAs \url{https://www.crypto101.io/}}

\item (Free) Dan Boneh, Victor Shoup, \emph{A Graduate Course in Applied Cryptography}\footnote{\AlsoAvailableAs \url{https://crypto.stanford.edu/~dabo/cryptobook/}}.
\end{itemize}


